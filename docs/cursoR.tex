\documentclass[]{book}
\usepackage{lmodern}
\usepackage{amssymb,amsmath}
\usepackage{ifxetex,ifluatex}
\usepackage{fixltx2e} % provides \textsubscript
\ifnum 0\ifxetex 1\fi\ifluatex 1\fi=0 % if pdftex
  \usepackage[T1]{fontenc}
  \usepackage[utf8]{inputenc}
\else % if luatex or xelatex
  \ifxetex
    \usepackage{mathspec}
  \else
    \usepackage{fontspec}
  \fi
  \defaultfontfeatures{Ligatures=TeX,Scale=MatchLowercase}
\fi
% use upquote if available, for straight quotes in verbatim environments
\IfFileExists{upquote.sty}{\usepackage{upquote}}{}
% use microtype if available
\IfFileExists{microtype.sty}{%
\usepackage{microtype}
\UseMicrotypeSet[protrusion]{basicmath} % disable protrusion for tt fonts
}{}
\usepackage[margin=1in]{geometry}
\usepackage{hyperref}
\hypersetup{unicode=true,
            pdftitle={Curso de R para Meteorologia IAG/USP},
            pdfauthor={Sergio Ibarra-Espinosa, Amanda Rehbein, Daniel Schuch, Camila Lopes, e possivelmente outros (u r invited to collaborate)},
            pdfborder={0 0 0},
            breaklinks=true}
\urlstyle{same}  % don't use monospace font for urls
\usepackage{natbib}
\bibliographystyle{apalike}
\usepackage{color}
\usepackage{fancyvrb}
\newcommand{\VerbBar}{|}
\newcommand{\VERB}{\Verb[commandchars=\\\{\}]}
\DefineVerbatimEnvironment{Highlighting}{Verbatim}{commandchars=\\\{\}}
% Add ',fontsize=\small' for more characters per line
\usepackage{framed}
\definecolor{shadecolor}{RGB}{248,248,248}
\newenvironment{Shaded}{\begin{snugshade}}{\end{snugshade}}
\newcommand{\KeywordTok}[1]{\textcolor[rgb]{0.13,0.29,0.53}{\textbf{#1}}}
\newcommand{\DataTypeTok}[1]{\textcolor[rgb]{0.13,0.29,0.53}{#1}}
\newcommand{\DecValTok}[1]{\textcolor[rgb]{0.00,0.00,0.81}{#1}}
\newcommand{\BaseNTok}[1]{\textcolor[rgb]{0.00,0.00,0.81}{#1}}
\newcommand{\FloatTok}[1]{\textcolor[rgb]{0.00,0.00,0.81}{#1}}
\newcommand{\ConstantTok}[1]{\textcolor[rgb]{0.00,0.00,0.00}{#1}}
\newcommand{\CharTok}[1]{\textcolor[rgb]{0.31,0.60,0.02}{#1}}
\newcommand{\SpecialCharTok}[1]{\textcolor[rgb]{0.00,0.00,0.00}{#1}}
\newcommand{\StringTok}[1]{\textcolor[rgb]{0.31,0.60,0.02}{#1}}
\newcommand{\VerbatimStringTok}[1]{\textcolor[rgb]{0.31,0.60,0.02}{#1}}
\newcommand{\SpecialStringTok}[1]{\textcolor[rgb]{0.31,0.60,0.02}{#1}}
\newcommand{\ImportTok}[1]{#1}
\newcommand{\CommentTok}[1]{\textcolor[rgb]{0.56,0.35,0.01}{\textit{#1}}}
\newcommand{\DocumentationTok}[1]{\textcolor[rgb]{0.56,0.35,0.01}{\textbf{\textit{#1}}}}
\newcommand{\AnnotationTok}[1]{\textcolor[rgb]{0.56,0.35,0.01}{\textbf{\textit{#1}}}}
\newcommand{\CommentVarTok}[1]{\textcolor[rgb]{0.56,0.35,0.01}{\textbf{\textit{#1}}}}
\newcommand{\OtherTok}[1]{\textcolor[rgb]{0.56,0.35,0.01}{#1}}
\newcommand{\FunctionTok}[1]{\textcolor[rgb]{0.00,0.00,0.00}{#1}}
\newcommand{\VariableTok}[1]{\textcolor[rgb]{0.00,0.00,0.00}{#1}}
\newcommand{\ControlFlowTok}[1]{\textcolor[rgb]{0.13,0.29,0.53}{\textbf{#1}}}
\newcommand{\OperatorTok}[1]{\textcolor[rgb]{0.81,0.36,0.00}{\textbf{#1}}}
\newcommand{\BuiltInTok}[1]{#1}
\newcommand{\ExtensionTok}[1]{#1}
\newcommand{\PreprocessorTok}[1]{\textcolor[rgb]{0.56,0.35,0.01}{\textit{#1}}}
\newcommand{\AttributeTok}[1]{\textcolor[rgb]{0.77,0.63,0.00}{#1}}
\newcommand{\RegionMarkerTok}[1]{#1}
\newcommand{\InformationTok}[1]{\textcolor[rgb]{0.56,0.35,0.01}{\textbf{\textit{#1}}}}
\newcommand{\WarningTok}[1]{\textcolor[rgb]{0.56,0.35,0.01}{\textbf{\textit{#1}}}}
\newcommand{\AlertTok}[1]{\textcolor[rgb]{0.94,0.16,0.16}{#1}}
\newcommand{\ErrorTok}[1]{\textcolor[rgb]{0.64,0.00,0.00}{\textbf{#1}}}
\newcommand{\NormalTok}[1]{#1}
\usepackage{longtable,booktabs}
\usepackage{graphicx,grffile}
\makeatletter
\def\maxwidth{\ifdim\Gin@nat@width>\linewidth\linewidth\else\Gin@nat@width\fi}
\def\maxheight{\ifdim\Gin@nat@height>\textheight\textheight\else\Gin@nat@height\fi}
\makeatother
% Scale images if necessary, so that they will not overflow the page
% margins by default, and it is still possible to overwrite the defaults
% using explicit options in \includegraphics[width, height, ...]{}
\setkeys{Gin}{width=\maxwidth,height=\maxheight,keepaspectratio}
\IfFileExists{parskip.sty}{%
\usepackage{parskip}
}{% else
\setlength{\parindent}{0pt}
\setlength{\parskip}{6pt plus 2pt minus 1pt}
}
\setlength{\emergencystretch}{3em}  % prevent overfull lines
\providecommand{\tightlist}{%
  \setlength{\itemsep}{0pt}\setlength{\parskip}{0pt}}
\setcounter{secnumdepth}{5}
% Redefines (sub)paragraphs to behave more like sections
\ifx\paragraph\undefined\else
\let\oldparagraph\paragraph
\renewcommand{\paragraph}[1]{\oldparagraph{#1}\mbox{}}
\fi
\ifx\subparagraph\undefined\else
\let\oldsubparagraph\subparagraph
\renewcommand{\subparagraph}[1]{\oldsubparagraph{#1}\mbox{}}
\fi

%%% Use protect on footnotes to avoid problems with footnotes in titles
\let\rmarkdownfootnote\footnote%
\def\footnote{\protect\rmarkdownfootnote}

%%% Change title format to be more compact
\usepackage{titling}

% Create subtitle command for use in maketitle
\newcommand{\subtitle}[1]{
  \posttitle{
    \begin{center}\large#1\end{center}
    }
}

\setlength{\droptitle}{-2em}
  \title{Curso de R para Meteorologia IAG/USP}
  \pretitle{\vspace{\droptitle}\centering\huge}
  \posttitle{\par}
  \author{Sergio Ibarra-Espinosa, Amanda Rehbein, Daniel Schuch, Camila Lopes, e
possivelmente outros (u r invited to collaborate)}
  \preauthor{\centering\large\emph}
  \postauthor{\par}
  \predate{\centering\large\emph}
  \postdate{\par}
  \date{2018-05-17}

\usepackage{booktabs}
\usepackage{amsthm}
\makeatletter
\def\thm@space@setup{%
  \thm@preskip=8pt plus 2pt minus 4pt
  \thm@postskip=\thm@preskip
}
\makeatother

\begin{document}
\maketitle

{
\setcounter{tocdepth}{1}
\tableofcontents
}
\chapter{Pre-requisitos do sistema}\label{primero}

Em Windows, instale além do R, Rtools
\url{https://cran.r-project.org/bin/windows/Rtools/}

Em MAC instale netcdf e:

\begin{Shaded}
\begin{Highlighting}[]
\ExtensionTok{brew}\NormalTok{ unlink gdal}
\ExtensionTok{brew}\NormalTok{ tap osgeo/osgeo4mac }\KeywordTok{&&} \ExtensionTok{brew}\NormalTok{ tap --repair}
\ExtensionTok{brew}\NormalTok{ install proj}
\ExtensionTok{brew}\NormalTok{ install geos}
\ExtensionTok{brew}\NormalTok{ install udunits}
\ExtensionTok{brew}\NormalTok{ install gdal2 --with-armadillo --with-complete --with-libkml --with-unsupported}
\ExtensionTok{brew}\NormalTok{ link --force gdal2}
\end{Highlighting}
\end{Shaded}

Em Ubuntu:

\begin{Shaded}
\begin{Highlighting}[]
  \ExtensionTok{-}\NormalTok{ sudo add-apt-repository ppa:ubuntugis/ubuntugis-unstable --yes}
  \ExtensionTok{-}\NormalTok{ sudo apt-get --yes --force-yes update -qq}
  \CommentTok{# install tmap dependencies}
  \ExtensionTok{-}\NormalTok{ sudo apt-get install --yes libprotobuf-dev protobuf-compiler libv8-3.14-dev}
  \CommentTok{# install tmap dependencies; for 16.04 libjq-dev this ppa is needed:}
  \ExtensionTok{-}\NormalTok{ sudo add-apt-repository -y ppa:opencpu/jq}
  \ExtensionTok{-}\NormalTok{ sudo apt-get --yes --force-yes update -qq}
  \ExtensionTok{-}\NormalTok{ sudo apt-get install libjq-dev}
  \CommentTok{# units/udunits2 dependency:}
  \ExtensionTok{-}\NormalTok{ sudo apt-get install --yes libudunits2-dev}
  \CommentTok{# sf dependencies:}
  \ExtensionTok{-}\NormalTok{ sudo apt-get install --yes libproj-dev libgeos-dev libgdal-dev libnetcdf-dev  netcdf-bin gdal-bin}
\end{Highlighting}
\end{Shaded}

\section{Pacotes usados neste curso}\label{pacotes-usados-neste-curso}

Para fazer este curso instale os seguintes pacotes como indicado:

\begin{Shaded}
\begin{Highlighting}[]
\KeywordTok{install.packages}\NormalTok{(}\StringTok{"devtools"}\NormalTok{)}
\NormalTok{devtools}\OperatorTok{::}\KeywordTok{install_github}\NormalTok{(}\StringTok{"tidyverse/tidyverse"}\NormalTok{)}
\NormalTok{devtools}\OperatorTok{::}\KeywordTok{install_github}\NormalTok{(}\StringTok{"r-spatial/sf"}\NormalTok{)}
\NormalTok{devtools}\OperatorTok{::}\KeywordTok{install_github}\NormalTok{(}\StringTok{"r-spatial/mapview"}\NormalTok{)}
\NormalTok{devtools}\OperatorTok{::}\KeywordTok{install_github}\NormalTok{(}\StringTok{"r-spatial/stars"}\NormalTok{)}
\KeywordTok{install.packages}\NormalTok{(}\KeywordTok{c}\NormalTok{(}\StringTok{"raster"}\NormalTok{, }\StringTok{"sp"}\NormalTok{, }\StringTok{"rgdal"}\NormalTok{, }\StringTok{"maptools"}\NormalTok{, }\StringTok{"ncdf4"}\NormalTok{))}
\KeywordTok{install.packages}\NormalTok{(}\KeywordTok{c}\NormalTok{(}\StringTok{"cptcity"}\NormalTok{, }\StringTok{"data.table"}\NormalTok{, }\StringTok{"openair"}\NormalTok{))}
\end{Highlighting}
\end{Shaded}

\begin{itemize}
\tightlist
\item
  \href{https://CRAN.R-project.org/package=devtools}{devtools} é um
  pacote para instalar pacotes de diferentes repositórios
\item
  \href{https://github.com/tidyverse}{tidyverse} é o universo de pacotes
  do Hadley Wickham. A instalação tem que ser usando devtools, pois
  precisamos plotar os objetos espacias sf usando
  \href{https://www.isgeomsfinggplot2yet.site/}{geom\_sf}.
\item
  \href{https://github.com/r-spatial/sf}{sf} e
  \href{https://github.com/r-spatial/mapbiew}{mapview},
  \href{https://github.com/r-spatial/stars}{stars}, raster, sp, rgdal e
  maptools são para a parte espacial. Lembrar que os objetos em
  meteorologias são espaço-temporais.
\item
  ncdf4 é um pacote para manipular arquivos NetCDF.
\item
  \href{https://ibarraespinosa.github.io/cptcity/}{cptcity} é um pacote
  que tem 7140 paletas de cores do arquivo web cpt-city
  (\url{http://soliton.vm.bytemark.co.uk/pub/cpt-city/index.html}).
\item
  \href{http://davidcarslaw.github.io/openair/}{openair} é um pacote
  para trabalhar com dados de qualidade do ar e meteorologia.
\end{itemize}

Se faltarem dependencias de sistema, instale elas e instale os pacotes.

\section{Colaborar}\label{colaborar}

A forma preferida de colaboração é com pull-requests em
\url{https://github.com/ibarraespinosa/cursoR/pull/new/master}. Lembre
de aplicar a Guia de Estilo de R de Google
(\url{https://google.github.io/styleguide/Rguide.xml}) ou com o formato
de formatR \url{https://yihui.name/formatr/}. Em poucas palavras, lembre
que seu código vai ser lido por seres humanos. Se quiser tem acesso no
repositório deste curso, me contate. Tem um botão para editar qualquer
página.

\section{Aportar com dados}\label{aportar-com-dados}

Se você tem dados para fazer este curso mais legal, por favor, edite
este aquivo e com pull request, eu vou fazer um merge para poder.

\begin{enumerate}
\def\labelenumi{\arabic{enumi}.}
\item
  NCEP: \url{ftp://nomads.ncdc.noaa.gov/GFS/analysis_only/}
\item
\item
\end{enumerate}

\chapter{Intro}\label{intro}

Este curso é para pos, então vamos ver conteúdo rapidamente e se não da
tempo, este curso esta online no sitio
\url{https://github.com/atmoschem/cursorIAG}.

Eu tento usar
\href{http://stat.ethz.ch/R-manual/R-devel/library/base/html/00Index.html}{BASE}
sempre que posso, e se não da ai vou para outros paradigmas.

Outros pacotes de BASE:
\href{http://stat.ethz.ch/R-manual/R-devel/library/utils/html/00Index.html}{utils},
\href{http://stat.ethz.ch/R-manual/R-devel/library/stats/html/00Index.html}{stats},
\href{http://stat.ethz.ch/R-manual/R-devel/library/datasets/html/00Index.html}{datasets},
\href{http://stat.ethz.ch/R-manual/R-devel/library/graphics/html/00Index.html}{graphics},
\href{https://stat.ethz.ch/R-manual/R-devel/library/grDevices/html/00Index.html}{grDevices},
\href{https://stat.ethz.ch/R-manual/R-devel/library/grid/html/00Index.html}{grid},
\href{https://stat.ethz.ch/R-manual/R-devel/library/methods/html/00Index.html}{methods},
\href{https://stat.ethz.ch/R-manual/R-devel/library/tools/html/00Index.html}{tools},
\href{https://stat.ethz.ch/R-manual/R-devel/library/parallel/html/00Index.html}{parallel},
\href{https://stat.ethz.ch/R-manual/R-devel/library/compiler/html/00Index.html}{compiler},
\href{https://stat.ethz.ch/R-manual/R-devel/library/splines/html/00Index.html}{splines},
\href{https://stat.ethz.ch/R-manual/R-devel/library/tcltk/html/00Index.html}{tcltk}
,
\href{https://stat.ethz.ch/R-manual/R-devel/library/stats4/html/00Index.html}{stats4}.

Veja
\href{https://cran.r-project.org/web/packages/available_packages_by_name.html}{outros}
pacotes.

Este curso esta baseado no livro
\href{https://leanpub.com/rprogramming}{R Programming for Data Science}.

Vamos usar \href{https://www.rstudio.com/}{Rstudio}

\textbf{Dica:}

\begin{itemize}
\tightlist
\item
  Se não sabe como usar uma função, escreva: \texttt{?função}.
\item
  As funções tem argumentos, use \textbf{TAB} para ver eles numa função.
\end{itemize}

\section{IMPORTANTE}\label{importante}

teu novo melhor amigo, besti friendi, BFF, parceiro, mano, tabarish,
komrade, compaheiro, colega, buisiness partner amd whatever meanningful
is

\begin{itemize}
\tightlist
\item
  \textbf{TAB} no \textbf{RSTUDIO}.
\end{itemize}

Esta combinação é tão boa, como o cafe com leite, pizza e abacaxi,
vitamina de acabate com amendoim Manaus, a melhor combinação.

\includegraphics[width=5.88in]{figuras/tab-key-}

Porque quando se tu não lembra os argumentos da função, e não quer ver o
help \emph{?} de cada função, so clica \textbf{TAB} e RSTUDIO te
mostrara a lista de argumentos.

Vamos lá!

\chapter{R!}\label{r}

\begin{itemize}
\tightlist
\item
  Quase em qualquer sistema operacional mas eu vou focar em Linux.
\item
  Muita documentação:
\item
  \href{http://cran.r-project.org/doc/manuals/r-release/R-intro.html}{Intro}.
\item
  \href{http://cran.r-project.org/doc/manuals/r-release/R-data.html}{I/O}.
\item
  Quer fazer um pacote?
  \href{http://cran.r-project.org/doc/manuals/r-release/R-exts.html}{Veja},
  \href{http://cran.r-project.org/doc/manuals/r-release/R-ints.html}{aqui}
  e
  \href{http://cran.r-project.org/doc/manuals/r-release/R-lang.html}{aqui}.
\item
  \href{https://stackoverflow.com/questions/tagged/r}{Stackoverflow}
  provides a great source of resources.
\end{itemize}

\section{Objetos de R}\label{objetos-de-r}

\begin{itemize}
\tightlist
\item
  Character a
\item
  numeric 1
\item
  integer 1
\item
  complex 0+1i
\item
  logical TRUE
\end{itemize}

\section{Classe}\label{classe}

\texttt{class} função permite ver a classe dos objetos

\section{Vetores}\label{vetores}

\begin{itemize}
\tightlist
\item
  c(``A'', ``C'', ``D'')
\item
  1:5 = c(1, 2, 3, 4, 5)
\item
  c(TRUE, FALSE)
\item
  c(1i, -1i)
\item
  c(1, ``C'', ``D'') qual é a classe???
\item
  c(1, NA, ``D'') qual é a classe???
\item
  c(1, NA, NaN) qual é a classe???
\end{itemize}

\section{\texorpdfstring{Convertir objetos com
\texttt{as}}{Convertir objetos com as}}\label{convertir-objetos-com-as}

\begin{Shaded}
\begin{Highlighting}[]
\KeywordTok{as.numeric}\NormalTok{(}\KeywordTok{c}\NormalTok{(}\DecValTok{1}\NormalTok{, }\StringTok{"C"}\NormalTok{, }\StringTok{"D"}\NormalTok{))}
\end{Highlighting}
\end{Shaded}

\begin{verbatim}
## Warning: NAs introduzidos por coerção
\end{verbatim}

\begin{verbatim}
## [1]  1 NA NA
\end{verbatim}

\section{\texorpdfstring{Matrices e a função
\texttt{matrix}}{Matrices e a função matrix}}\label{matrices-e-a-funcao-matrix}

\textbf{{[}linhas, colunas{]}}

\begin{itemize}
\tightlist
\item
  permitidos elementos \textbf{da mesma clase}!
\end{itemize}

vamos ver os argumentos da função \texttt{matrix}

\begin{Shaded}
\begin{Highlighting}[]
\KeywordTok{args}\NormalTok{(matrix)}
\end{Highlighting}
\end{Shaded}

\begin{verbatim}
## function (data = NA, nrow = 1, ncol = 1, byrow = FALSE, dimnames = NULL) 
## NULL
\end{verbatim}

usando TAB

\begin{Shaded}
\begin{Highlighting}[]
\NormalTok{(m <-}\StringTok{ }\KeywordTok{matrix}\NormalTok{(}\DataTypeTok{data =} \DecValTok{0}\NormalTok{, }\DataTypeTok{nrow =} \DecValTok{4}\NormalTok{, }\DataTypeTok{ncol =} \DecValTok{4}\NormalTok{))}
\end{Highlighting}
\end{Shaded}

\begin{verbatim}
##      [,1] [,2] [,3] [,4]
## [1,]    0    0    0    0
## [2,]    0    0    0    0
## [3,]    0    0    0    0
## [4,]    0    0    0    0
\end{verbatim}

\begin{Shaded}
\begin{Highlighting}[]
\NormalTok{(m1 <-}\StringTok{ }\KeywordTok{matrix}\NormalTok{(}\DataTypeTok{data =} \DecValTok{1}\OperatorTok{:}\NormalTok{(}\DecValTok{4}\OperatorTok{*}\DecValTok{4}\NormalTok{), }\DataTypeTok{nrow =} \DecValTok{4}\NormalTok{, }\DataTypeTok{ncol =} \DecValTok{4}\NormalTok{))}
\end{Highlighting}
\end{Shaded}

\begin{verbatim}
##      [,1] [,2] [,3] [,4]
## [1,]    1    5    9   13
## [2,]    2    6   10   14
## [3,]    3    7   11   15
## [4,]    4    8   12   16
\end{verbatim}

\begin{Shaded}
\begin{Highlighting}[]
\KeywordTok{dim}\NormalTok{(m1)}
\end{Highlighting}
\end{Shaded}

\begin{verbatim}
## [1] 4 4
\end{verbatim}

\begin{Shaded}
\begin{Highlighting}[]
\NormalTok{(m2 <-}\StringTok{ }\KeywordTok{matrix}\NormalTok{(}\DataTypeTok{data =} \DecValTok{1}\OperatorTok{:}\NormalTok{(}\DecValTok{4}\OperatorTok{*}\DecValTok{4}\NormalTok{), }\DataTypeTok{nrow =} \DecValTok{4}\NormalTok{, }\DataTypeTok{ncol =} \DecValTok{4}\NormalTok{, }\DataTypeTok{byrow =} \OtherTok{TRUE}\NormalTok{))}
\end{Highlighting}
\end{Shaded}

\begin{verbatim}
##      [,1] [,2] [,3] [,4]
## [1,]    1    2    3    4
## [2,]    5    6    7    8
## [3,]    9   10   11   12
## [4,]   13   14   15   16
\end{verbatim}

\section{Array}\label{array}

É como uma matriz de matrizes de matrizes de matrizes\ldots{}\ldots{}
and so on.

\begin{Shaded}
\begin{Highlighting}[]
\KeywordTok{args}\NormalTok{(array)}
\end{Highlighting}
\end{Shaded}

\begin{verbatim}
## function (data = NA, dim = length(data), dimnames = NULL) 
## NULL
\end{verbatim}

lembre usar TAB

\begin{Shaded}
\begin{Highlighting}[]
\NormalTok{(a <-}\StringTok{ }\KeywordTok{array}\NormalTok{(}\DataTypeTok{data =} \DecValTok{0}\NormalTok{, }\DataTypeTok{dim =} \KeywordTok{c}\NormalTok{(}\DecValTok{1}\NormalTok{,}\DecValTok{1}\NormalTok{)))}
\end{Highlighting}
\end{Shaded}

\begin{verbatim}
##      [,1]
## [1,]    0
\end{verbatim}

\begin{Shaded}
\begin{Highlighting}[]
\KeywordTok{class}\NormalTok{(a)}
\end{Highlighting}
\end{Shaded}

\begin{verbatim}
## [1] "matrix"
\end{verbatim}

\begin{Shaded}
\begin{Highlighting}[]
\NormalTok{(a <-}\StringTok{ }\KeywordTok{array}\NormalTok{(}\DataTypeTok{data =} \DecValTok{0}\NormalTok{, }\DataTypeTok{dim =} \KeywordTok{c}\NormalTok{(}\DecValTok{1}\NormalTok{,}\DecValTok{1}\NormalTok{,}\DecValTok{1}\NormalTok{)))}
\end{Highlighting}
\end{Shaded}

\begin{verbatim}
## , , 1
## 
##      [,1]
## [1,]    0
\end{verbatim}

\begin{Shaded}
\begin{Highlighting}[]
\KeywordTok{class}\NormalTok{(a)}
\end{Highlighting}
\end{Shaded}

\begin{verbatim}
## [1] "array"
\end{verbatim}

\begin{Shaded}
\begin{Highlighting}[]
\NormalTok{(a <-}\StringTok{ }\KeywordTok{array}\NormalTok{(}\DataTypeTok{data =} \DecValTok{0}\NormalTok{, }\DataTypeTok{dim =} \KeywordTok{c}\NormalTok{(}\DecValTok{2}\NormalTok{,}\DecValTok{2}\NormalTok{,}\DecValTok{2}\NormalTok{)))}
\end{Highlighting}
\end{Shaded}

\begin{verbatim}
## , , 1
## 
##      [,1] [,2]
## [1,]    0    0
## [2,]    0    0
## 
## , , 2
## 
##      [,1] [,2]
## [1,]    0    0
## [2,]    0    0
\end{verbatim}

\begin{Shaded}
\begin{Highlighting}[]
\NormalTok{(a <-}\StringTok{ }\KeywordTok{array}\NormalTok{(}\DataTypeTok{data =} \DecValTok{0}\NormalTok{, }\DataTypeTok{dim =} \KeywordTok{c}\NormalTok{(}\DecValTok{2}\NormalTok{,}\DecValTok{4}\NormalTok{,}\DecValTok{4}\NormalTok{)))}
\end{Highlighting}
\end{Shaded}

\begin{verbatim}
## , , 1
## 
##      [,1] [,2] [,3] [,4]
## [1,]    0    0    0    0
## [2,]    0    0    0    0
## 
## , , 2
## 
##      [,1] [,2] [,3] [,4]
## [1,]    0    0    0    0
## [2,]    0    0    0    0
## 
## , , 3
## 
##      [,1] [,2] [,3] [,4]
## [1,]    0    0    0    0
## [2,]    0    0    0    0
## 
## , , 4
## 
##      [,1] [,2] [,3] [,4]
## [1,]    0    0    0    0
## [2,]    0    0    0    0
\end{verbatim}

\begin{Shaded}
\begin{Highlighting}[]
\KeywordTok{dim}\NormalTok{(a)}
\end{Highlighting}
\end{Shaded}

\begin{verbatim}
## [1] 2 4 4
\end{verbatim}

\begin{Shaded}
\begin{Highlighting}[]
\NormalTok{(a <-}\StringTok{ }\KeywordTok{array}\NormalTok{(}\DataTypeTok{data =} \DecValTok{0}\NormalTok{, }\DataTypeTok{dim =} \KeywordTok{c}\NormalTok{(}\DecValTok{2}\NormalTok{, }\DecValTok{2}\NormalTok{,}\DecValTok{2}\NormalTok{,}\DecValTok{2}\NormalTok{)))}
\end{Highlighting}
\end{Shaded}

\section{\texorpdfstring{\texttt{list}}{list}}\label{list}

As listas são como sacolas, e dentro delas, tu pode colocar mais
sacolas\ldots{} então, tu pode ter sacolas, dentro de sacolas, dentro de
sacolas\ldots{} ou

\begin{Shaded}
\begin{Highlighting}[]
\KeywordTok{list}\NormalTok{(}\KeywordTok{list}\NormalTok{(}\KeywordTok{list}\NormalTok{(}\KeywordTok{list}\NormalTok{(}\DecValTok{1}\NormalTok{))))}
\end{Highlighting}
\end{Shaded}

\begin{verbatim}
## [[1]]
## [[1]][[1]]
## [[1]][[1]][[1]]
## [[1]][[1]][[1]][[1]]
## [1] 1
\end{verbatim}

a diferença das matrices, tu pode colocar cualquer coisa nas listas, por
exemplo: funções, characters, etc.

\begin{Shaded}
\begin{Highlighting}[]
\NormalTok{(x <-}\StringTok{ }\KeywordTok{list}\NormalTok{(}\DecValTok{1}\NormalTok{, }\StringTok{"a"}\NormalTok{, }\OtherTok{TRUE}\NormalTok{, }\DecValTok{1} \OperatorTok{+}\StringTok{ }\NormalTok{4i))}
\end{Highlighting}
\end{Shaded}

\begin{verbatim}
## [[1]]
## [1] 1
## 
## [[2]]
## [1] "a"
## 
## [[3]]
## [1] TRUE
## 
## [[4]]
## [1] 1+4i
\end{verbatim}

\section{Tempo e Data}\label{tempo-e-data}

R tem classes de tempo e data:

\begin{Shaded}
\begin{Highlighting}[]
\NormalTok{(a <-}\StringTok{ }\KeywordTok{ISOdate}\NormalTok{(}\DataTypeTok{year =} \DecValTok{2018}\NormalTok{, }\DataTypeTok{month =} \DecValTok{4}\NormalTok{, }\DataTypeTok{day =} \DecValTok{5}\NormalTok{))}
\end{Highlighting}
\end{Shaded}

\begin{verbatim}
## [1] "2018-04-05 12:00:00 GMT"
\end{verbatim}

\begin{Shaded}
\begin{Highlighting}[]
\KeywordTok{class}\NormalTok{(a)}
\end{Highlighting}
\end{Shaded}

\begin{verbatim}
## [1] "POSIXct" "POSIXt"
\end{verbatim}

\begin{Shaded}
\begin{Highlighting}[]
\NormalTok{(b <-}\StringTok{ }\KeywordTok{ISOdate}\NormalTok{(}\DataTypeTok{year =} \DecValTok{2018}\NormalTok{, }\DataTypeTok{month =} \DecValTok{4}\NormalTok{, }\DataTypeTok{day =} \DecValTok{5}\NormalTok{, }\DataTypeTok{tz =} \StringTok{"Americas/Sao_Paulo"}\NormalTok{))}
\end{Highlighting}
\end{Shaded}

\begin{verbatim}
## [1] "2018-04-05 12:00:00 Americas"
\end{verbatim}

tempo

\begin{Shaded}
\begin{Highlighting}[]
\NormalTok{(d <-}\StringTok{ }\KeywordTok{ISOdatetime}\NormalTok{(}\DataTypeTok{year =} \DecValTok{2018}\NormalTok{, }\DataTypeTok{month =} \DecValTok{4}\NormalTok{, }\DataTypeTok{day =} \DecValTok{5}\NormalTok{, }\DataTypeTok{hour =} \DecValTok{0}\NormalTok{, }\DataTypeTok{min =} \DecValTok{0}\NormalTok{, }\DataTypeTok{sec =} \DecValTok{0}\NormalTok{,}
                  \DataTypeTok{tz =} \StringTok{"Americas/Sao_Paulo"}\NormalTok{))}
\end{Highlighting}
\end{Shaded}

\begin{verbatim}
## [1] "2018-04-05 Americas"
\end{verbatim}

O pacote \href{https://github.com/eddelbuettel/nanotime}{nanotime}
permite trabalhar com nano segundos.

Da pra fazer secuencias:

\begin{Shaded}
\begin{Highlighting}[]
\NormalTok{hoje <-}\StringTok{ }\KeywordTok{Sys.time}\NormalTok{()}
\NormalTok{(a <-}\StringTok{ }\KeywordTok{seq.POSIXt}\NormalTok{(}\DataTypeTok{from =}\NormalTok{ hoje, }\DataTypeTok{by =} \DecValTok{3600}\NormalTok{, }\DataTypeTok{length.out =} \DecValTok{24}\NormalTok{))}
\end{Highlighting}
\end{Shaded}

\begin{verbatim}
##  [1] "2018-05-17 19:15:05 -03" "2018-05-17 20:15:05 -03"
##  [3] "2018-05-17 21:15:05 -03" "2018-05-17 22:15:05 -03"
##  [5] "2018-05-17 23:15:05 -03" "2018-05-18 00:15:05 -03"
##  [7] "2018-05-18 01:15:05 -03" "2018-05-18 02:15:05 -03"
##  [9] "2018-05-18 03:15:05 -03" "2018-05-18 04:15:05 -03"
## [11] "2018-05-18 05:15:05 -03" "2018-05-18 06:15:05 -03"
## [13] "2018-05-18 07:15:05 -03" "2018-05-18 08:15:05 -03"
## [15] "2018-05-18 09:15:05 -03" "2018-05-18 10:15:05 -03"
## [17] "2018-05-18 11:15:05 -03" "2018-05-18 12:15:05 -03"
## [19] "2018-05-18 13:15:05 -03" "2018-05-18 14:15:05 -03"
## [21] "2018-05-18 15:15:05 -03" "2018-05-18 16:15:05 -03"
## [23] "2018-05-18 17:15:05 -03" "2018-05-18 18:15:05 -03"
\end{verbatim}

funções bacana: \textbf{weekdays}, \textbf{month}, \textbf{julian}

\begin{Shaded}
\begin{Highlighting}[]
\KeywordTok{weekdays}\NormalTok{(a)}
\end{Highlighting}
\end{Shaded}

\begin{verbatim}
##  [1] "Thursday" "Thursday" "Thursday" "Thursday" "Thursday" "Friday"  
##  [7] "Friday"   "Friday"   "Friday"   "Friday"   "Friday"   "Friday"  
## [13] "Friday"   "Friday"   "Friday"   "Friday"   "Friday"   "Friday"  
## [19] "Friday"   "Friday"   "Friday"   "Friday"   "Friday"   "Friday"
\end{verbatim}

\begin{Shaded}
\begin{Highlighting}[]
\KeywordTok{months}\NormalTok{(a)}
\end{Highlighting}
\end{Shaded}

\begin{verbatim}
##  [1] "May" "May" "May" "May" "May" "May" "May" "May" "May" "May" "May"
## [12] "May" "May" "May" "May" "May" "May" "May" "May" "May" "May" "May"
## [23] "May" "May"
\end{verbatim}

\begin{Shaded}
\begin{Highlighting}[]
\KeywordTok{julian}\NormalTok{(a) }\CommentTok{#olha ?julian... dias desde origin}
\end{Highlighting}
\end{Shaded}

\begin{verbatim}
## Time differences in days
##  [1] 17668.93 17668.97 17669.01 17669.05 17669.09 17669.14 17669.18
##  [8] 17669.22 17669.26 17669.30 17669.34 17669.39 17669.43 17669.47
## [15] 17669.51 17669.55 17669.59 17669.64 17669.68 17669.72 17669.76
## [22] 17669.80 17669.84 17669.89
## attr(,"origin")
## [1] "1970-01-01 GMT"
\end{verbatim}

olha \url{https://en.wikipedia.org/wiki/Julian_day}:

\section{Fatores}\label{fatores}

Os \texttt{factors} podem ser um pouco infernais. Olha
\href{http://www.burns-stat.com/documents/books/the-r-inferno/}{R
INFERNO}

Usados para representar categorias, ejemplo clasico para nos, dias da
semana.

\begin{Shaded}
\begin{Highlighting}[]
\NormalTok{a <-}\StringTok{ }\KeywordTok{seq.POSIXt}\NormalTok{(}\DataTypeTok{from =}\NormalTok{ hoje, }\DataTypeTok{by =} \DecValTok{3600}\NormalTok{, }\DataTypeTok{length.out =} \DecValTok{24}\OperatorTok{*}\DecValTok{7}\NormalTok{)}
\NormalTok{aa <-}\StringTok{ }\KeywordTok{weekdays}\NormalTok{(a)}
\KeywordTok{class}\NormalTok{(aa)}
\end{Highlighting}
\end{Shaded}

\begin{verbatim}
## [1] "character"
\end{verbatim}

\begin{Shaded}
\begin{Highlighting}[]
\KeywordTok{factor}\NormalTok{(aa)}
\end{Highlighting}
\end{Shaded}

\begin{verbatim}
##   [1] Thursday  Thursday  Thursday  Thursday  Thursday  Friday    Friday   
##   [8] Friday    Friday    Friday    Friday    Friday    Friday    Friday   
##  [15] Friday    Friday    Friday    Friday    Friday    Friday    Friday   
##  [22] Friday    Friday    Friday    Friday    Friday    Friday    Friday   
##  [29] Friday    Saturday  Saturday  Saturday  Saturday  Saturday  Saturday 
##  [36] Saturday  Saturday  Saturday  Saturday  Saturday  Saturday  Saturday 
##  [43] Saturday  Saturday  Saturday  Saturday  Saturday  Saturday  Saturday 
##  [50] Saturday  Saturday  Saturday  Saturday  Sunday    Sunday    Sunday   
##  [57] Sunday    Sunday    Sunday    Sunday    Sunday    Sunday    Sunday   
##  [64] Sunday    Sunday    Sunday    Sunday    Sunday    Sunday    Sunday   
##  [71] Sunday    Sunday    Sunday    Sunday    Sunday    Sunday    Sunday   
##  [78] Monday    Monday    Monday    Monday    Monday    Monday    Monday   
##  [85] Monday    Monday    Monday    Monday    Monday    Monday    Monday   
##  [92] Monday    Monday    Monday    Monday    Monday    Monday    Monday   
##  [99] Monday    Monday    Monday    Tuesday   Tuesday   Tuesday   Tuesday  
## [106] Tuesday   Tuesday   Tuesday   Tuesday   Tuesday   Tuesday   Tuesday  
## [113] Tuesday   Tuesday   Tuesday   Tuesday   Tuesday   Tuesday   Tuesday  
## [120] Tuesday   Tuesday   Tuesday   Tuesday   Tuesday   Tuesday   Wednesday
## [127] Wednesday Wednesday Wednesday Wednesday Wednesday Wednesday Wednesday
## [134] Wednesday Wednesday Wednesday Wednesday Wednesday Wednesday Wednesday
## [141] Wednesday Wednesday Wednesday Wednesday Wednesday Wednesday Wednesday
## [148] Wednesday Wednesday Thursday  Thursday  Thursday  Thursday  Thursday 
## [155] Thursday  Thursday  Thursday  Thursday  Thursday  Thursday  Thursday 
## [162] Thursday  Thursday  Thursday  Thursday  Thursday  Thursday  Thursday 
## Levels: Friday Monday Saturday Sunday Thursday Tuesday Wednesday
\end{verbatim}

olha os \textbf{Levels}

Então:

\begin{Shaded}
\begin{Highlighting}[]
\NormalTok{ab <-}\StringTok{ }\KeywordTok{factor}\NormalTok{(}\DataTypeTok{x =}\NormalTok{ aa,}
             \DataTypeTok{levels =} \KeywordTok{c}\NormalTok{(}\StringTok{"Monday"}\NormalTok{, }\StringTok{"Tuesday"}\NormalTok{,  }\StringTok{"Wednesday"}\NormalTok{,  }\StringTok{"Thursday"}\NormalTok{,}
                        \StringTok{"Friday"}\NormalTok{, }\StringTok{"Saturday"}\NormalTok{, }\StringTok{"Sunday"}\NormalTok{))}
\KeywordTok{levels}\NormalTok{(ab)}
\end{Highlighting}
\end{Shaded}

\begin{verbatim}
## [1] "Monday"    "Tuesday"   "Wednesday" "Thursday"  "Friday"    "Saturday" 
## [7] "Sunday"
\end{verbatim}

\section{\texorpdfstring{\texttt{data.frames}}{data.frames}}\label{data.frames}

\emph{lembre ?data.frame}

São como planilha EXCEL\ldots{}. mais o menos

É uma classe bem especial, tem elementos de matriz mas o modo é lista

\begin{Shaded}
\begin{Highlighting}[]
\NormalTok{(df <-}\StringTok{ }\KeywordTok{data.frame}\NormalTok{(}\DataTypeTok{a =} \DecValTok{1}\OperatorTok{:}\DecValTok{3}\NormalTok{))}
\end{Highlighting}
\end{Shaded}

\begin{verbatim}
##   a
## 1 1
## 2 2
## 3 3
\end{verbatim}

\begin{Shaded}
\begin{Highlighting}[]
\KeywordTok{names}\NormalTok{(df)}
\end{Highlighting}
\end{Shaded}

\begin{verbatim}
## [1] "a"
\end{verbatim}

\begin{Shaded}
\begin{Highlighting}[]
\KeywordTok{class}\NormalTok{(df)}
\end{Highlighting}
\end{Shaded}

\begin{verbatim}
## [1] "data.frame"
\end{verbatim}

\begin{Shaded}
\begin{Highlighting}[]
\KeywordTok{mode}\NormalTok{(df)}
\end{Highlighting}
\end{Shaded}

\begin{verbatim}
## [1] "list"
\end{verbatim}

Então

\begin{Shaded}
\begin{Highlighting}[]
\KeywordTok{nrow}\NormalTok{(df)}
\end{Highlighting}
\end{Shaded}

\begin{verbatim}
## [1] 3
\end{verbatim}

\begin{Shaded}
\begin{Highlighting}[]
\KeywordTok{ncol}\NormalTok{(df)}
\end{Highlighting}
\end{Shaded}

\begin{verbatim}
## [1] 1
\end{verbatim}

\begin{Shaded}
\begin{Highlighting}[]
\KeywordTok{dim}\NormalTok{(df)}
\end{Highlighting}
\end{Shaded}

\begin{verbatim}
## [1] 3 1
\end{verbatim}

\chapter{Importando e exportando dados em
R}\label{importando-e-exportando-dados-em-r}

\section{data-frames}\label{data-frames}

Probabelmente um dos promeiros objetos que vamos usar quando começamos
usar R. Pensa num data-frame como uma planilha de Libreoffice (o excel).
Os data-frame pode ser criaos como foi visto na seção anterior. O
principal, é que temos varias funções para ler data-frames no R, entre
elas

\begin{itemize}
\tightlist
\item
  read.csv
\item
  read.csv2
\item
  read.table
\end{itemize}

Agora vamos a ler dados do repositorio usando read.table, mas primeiro
vamos lembrar que se tu precisar ver a ajuda da função, tem que escrever
no R \texttt{?read.table}. Então, agora vamos ver os argumentos da
função:

\begin{Shaded}
\begin{Highlighting}[]
\KeywordTok{args}\NormalTok{(read.table)}
\end{Highlighting}
\end{Shaded}

\begin{verbatim}
## function (file, header = FALSE, sep = "", quote = "\"'", dec = ".", 
##     numerals = c("allow.loss", "warn.loss", "no.loss"), row.names, 
##     col.names, as.is = !stringsAsFactors, na.strings = "NA", 
##     colClasses = NA, nrows = -1, skip = 0, check.names = TRUE, 
##     fill = !blank.lines.skip, strip.white = FALSE, blank.lines.skip = TRUE, 
##     comment.char = "#", allowEscapes = FALSE, flush = FALSE, 
##     stringsAsFactors = default.stringsAsFactors(), fileEncoding = "", 
##     encoding = "unknown", text, skipNul = FALSE) 
## NULL
\end{verbatim}

Aqui vem-se os valores default dos argumentos da função
\texttt{read.table}. O terceiro argumento é sep, com valores por default
= ``''.

\begin{Shaded}
\begin{Highlighting}[]
\NormalTok{df <-}\StringTok{ }\KeywordTok{read.table}\NormalTok{(}\StringTok{"https://raw.githubusercontent.com/ibarraespinosa/cursoR/master/dados/NOXIPEN2014.txt"}\NormalTok{)}
\end{Highlighting}
\end{Shaded}

Agora vamos usar a funções \texttt{head} and \texttt{tail} para ver as
primeiras e as ultimas 6 linhas do data-frame.

\begin{Shaded}
\begin{Highlighting}[]
\KeywordTok{head}\NormalTok{(df)}
\end{Highlighting}
\end{Shaded}

\begin{verbatim}
##   TipodeRede TipodeMonitoramento            Tipo       Data  Hora
## 2 Automático              CETESB Dados Primários 01/01/2014 01:00
## 3 Automático              CETESB Dados Primários 01/01/2014 02:00
## 4 Automático              CETESB Dados Primários 01/01/2014 03:00
## 5 Automático              CETESB Dados Primários 01/01/2014 04:00
## 6 Automático              CETESB Dados Primários 01/01/2014 05:00
## 7 Automático              CETESB Dados Primários 01/01/2014 06:00
##   CodigoEstação                NomeEstação              NomeParâmetro
## 2            95 Cid.Universitária-USP-Ipen NOx (Óxidos de Nitrogênio)
## 3            95 Cid.Universitária-USP-Ipen NOx (Óxidos de Nitrogênio)
## 4            95 Cid.Universitária-USP-Ipen NOx (Óxidos de Nitrogênio)
## 5            95 Cid.Universitária-USP-Ipen NOx (Óxidos de Nitrogênio)
## 6            95 Cid.Universitária-USP-Ipen NOx (Óxidos de Nitrogênio)
## 7            95 Cid.Universitária-USP-Ipen NOx (Óxidos de Nitrogênio)
##   UnidadedeMedida MediaHoraria MediaMovel Valido
## 2             ppb            9          -    Não
## 3             ppb            9          -    Sim
## 4             ppb            5          -    Sim
## 5             ppb            4          -    Sim
## 6             ppb            5          -    Sim
## 7             ppb            5          -    Sim
\end{verbatim}

\begin{Shaded}
\begin{Highlighting}[]
\KeywordTok{tail}\NormalTok{(df)}
\end{Highlighting}
\end{Shaded}

\begin{verbatim}
##      TipodeRede TipodeMonitoramento            Tipo       Data  Hora
## 8577 Automático              CETESB Dados Primários 01/01/2015 19:00
## 8578 Automático              CETESB Dados Primários 01/01/2015 20:00
## 8579 Automático              CETESB Dados Primários 01/01/2015 21:00
## 8580 Automático              CETESB Dados Primários 01/01/2015 22:00
## 8581 Automático              CETESB Dados Primários 01/01/2015 23:00
## 8582 Automático              CETESB Dados Primários 01/01/2015 24:00
##      CodigoEstação                NomeEstação              NomeParâmetro
## 8577            95 Cid.Universitária-USP-Ipen NOx (Óxidos de Nitrogênio)
## 8578            95 Cid.Universitária-USP-Ipen NOx (Óxidos de Nitrogênio)
## 8579            95 Cid.Universitária-USP-Ipen NOx (Óxidos de Nitrogênio)
## 8580            95 Cid.Universitária-USP-Ipen NOx (Óxidos de Nitrogênio)
## 8581            95 Cid.Universitária-USP-Ipen NOx (Óxidos de Nitrogênio)
## 8582            95 Cid.Universitária-USP-Ipen NOx (Óxidos de Nitrogênio)
##      UnidadedeMedida MediaHoraria MediaMovel Valido
## 8577             ppb            3          -    Sim
## 8578             ppb            8          -    Sim
## 8579             ppb           11          -    Sim
## 8580             ppb           11          -    Sim
## 8581             ppb           16          -    Sim
## 8582             ppb           NA          -    Sim
\end{verbatim}

Agora vamos ler os mesmos dados com outro formato e testar e read.table
funciona do mesmo jeito

\begin{Shaded}
\begin{Highlighting}[]
\NormalTok{df2 <-}\StringTok{ }\KeywordTok{read.table}\NormalTok{(}\StringTok{"https://raw.githubusercontent.com/ibarraespinosa/cursoR/master/dados/NOXIPEN2014v2.txt"}\NormalTok{)}
\CommentTok{# Error in scan(file = file, what = what, sep = sep, quote = quote, dec = dec, : }
\CommentTok{# linha 1 não tinha 6 elementos}
\end{Highlighting}
\end{Shaded}

Vemos a mensagem de error, mas o que quer dizer.

\textbf{Se tu recever um banco de dados tipo .txt e quer abrir no
R\ldots{} ABRE ELE COM BLOCO DE NOTAS PRIMEIRO!!!}

O primeiro arquivo:

\includegraphics[width=18.47in]{figuras/f1}

O segundo arquivo é:

\includegraphics[width=15.33in]{figuras/f2}

qual é a diferença?

Como vemos o segundo arquivo tem separação de ``;'', entao, temos que
lero arquivo assim:

\begin{Shaded}
\begin{Highlighting}[]
\NormalTok{df2 <-}\StringTok{ }\KeywordTok{read.table}\NormalTok{(}\StringTok{"https://raw.githubusercontent.com/ibarraespinosa/cursoR/master/dados/NOXIPEN2014v2.txt"}\NormalTok{, }\DataTypeTok{sep =} \StringTok{";"}\NormalTok{)}
\KeywordTok{head}\NormalTok{(df2)}
\end{Highlighting}
\end{Shaded}

\begin{verbatim}
##   TipodeRede TipodeMonitoramento            Tipo       Data  Hora
## 2 Automático              CETESB Dados Primários 01/01/2014 01:00
## 3 Automático              CETESB Dados Primários 01/01/2014 02:00
## 4 Automático              CETESB Dados Primários 01/01/2014 03:00
## 5 Automático              CETESB Dados Primários 01/01/2014 04:00
## 6 Automático              CETESB Dados Primários 01/01/2014 05:00
## 7 Automático              CETESB Dados Primários 01/01/2014 06:00
##   CodigoEstação                NomeEstação              NomeParâmetro
## 2            95 Cid.Universitária-USP-Ipen NOx (Óxidos de Nitrogênio)
## 3            95 Cid.Universitária-USP-Ipen NOx (Óxidos de Nitrogênio)
## 4            95 Cid.Universitária-USP-Ipen NOx (Óxidos de Nitrogênio)
## 5            95 Cid.Universitária-USP-Ipen NOx (Óxidos de Nitrogênio)
## 6            95 Cid.Universitária-USP-Ipen NOx (Óxidos de Nitrogênio)
## 7            95 Cid.Universitária-USP-Ipen NOx (Óxidos de Nitrogênio)
##   UnidadedeMedida MediaHoraria MediaMovel Valido
## 2             ppb            9          -    Não
## 3             ppb            9          -    Sim
## 4             ppb            5          -    Sim
## 5             ppb            4          -    Sim
## 6             ppb            5          -    Sim
## 7             ppb            5          -    Sim
\end{verbatim}

\begin{Shaded}
\begin{Highlighting}[]
\KeywordTok{tail}\NormalTok{(df2)}
\end{Highlighting}
\end{Shaded}

\begin{verbatim}
##      TipodeRede TipodeMonitoramento            Tipo       Data  Hora
## 8577 Automático              CETESB Dados Primários 01/01/2015 19:00
## 8578 Automático              CETESB Dados Primários 01/01/2015 20:00
## 8579 Automático              CETESB Dados Primários 01/01/2015 21:00
## 8580 Automático              CETESB Dados Primários 01/01/2015 22:00
## 8581 Automático              CETESB Dados Primários 01/01/2015 23:00
## 8582 Automático              CETESB Dados Primários 01/01/2015 24:00
##      CodigoEstação                NomeEstação              NomeParâmetro
## 8577            95 Cid.Universitária-USP-Ipen NOx (Óxidos de Nitrogênio)
## 8578            95 Cid.Universitária-USP-Ipen NOx (Óxidos de Nitrogênio)
## 8579            95 Cid.Universitária-USP-Ipen NOx (Óxidos de Nitrogênio)
## 8580            95 Cid.Universitária-USP-Ipen NOx (Óxidos de Nitrogênio)
## 8581            95 Cid.Universitária-USP-Ipen NOx (Óxidos de Nitrogênio)
## 8582            95 Cid.Universitária-USP-Ipen NOx (Óxidos de Nitrogênio)
##      UnidadedeMedida MediaHoraria MediaMovel Valido
## 8577             ppb            3          -    Sim
## 8578             ppb            8          -    Sim
## 8579             ppb           11          -    Sim
## 8580             ppb           11          -    Sim
## 8581             ppb           16          -    Sim
## 8582             ppb           NA          -    Sim
\end{verbatim}

\subsection{Qua dificultades tu já enfrentou importando
dados?}\label{qua-dificultades-tu-ja-enfrentou-importando-dados}

\section{\texorpdfstring{Exportando texto com
\texttt{write.table}}{Exportando texto com write.table}}\label{exportando-texto-com-write.table}

Exportar é bem facil, mas se sabemos os argumentos das funções, vai ser
mais eficiente ainda. Vamos \texttt{write.table}

\begin{Shaded}
\begin{Highlighting}[]
\KeywordTok{args}\NormalTok{(write.table)}
\end{Highlighting}
\end{Shaded}

\begin{verbatim}
## function (x, file = "", append = FALSE, quote = TRUE, sep = " ", 
##     eol = "\n", na = "NA", dec = ".", row.names = TRUE, col.names = TRUE, 
##     qmethod = c("escape", "double"), fileEncoding = "") 
## NULL
\end{verbatim}

Se temos um data-frame com colunas de classe character,
\texttt{quote\ =\ TRUE} quer dizer que o arquivo de texto resultante vai
ter aspas nas colunas de caracter.

\texttt{sep} é como vão ser separadas as colunas. Se tu quer abrir o
arquivo com Excel, poderia separar com ``,'', ``;'', " ``,''\t``\ldots{}
Depende como tu quer.

eol quer dizer \emph{end of line}, e é para ver a forma de colocar o
``end of line''

\textbf{row.names}.. esta TRUE mas SEMPRE SEMPRE SEMPRE COLOCA:

\textbf{row.names = FALSE}. Se não, R vai adiiconar uma coluna com os
indices das linhas\ldots{}.

col.names se tu quer o nome nas colunas\ldots{}

\textbf{PRATICA!}

\section{\texorpdfstring{Exportando objetos com
\texttt{save}}{Exportando objetos com save}}\label{exportando-objetos-com-save}

\begin{Shaded}
\begin{Highlighting}[]
\KeywordTok{args}\NormalTok{(save)}
\end{Highlighting}
\end{Shaded}

\begin{verbatim}
## function (..., list = character(), file = stop("'file' must be specified"), 
##     ascii = FALSE, version = NULL, envir = parent.frame(), compress = isTRUE(!ascii), 
##     compression_level, eval.promises = TRUE, precheck = TRUE) 
## NULL
\end{verbatim}

\texttt{save} salva o objeto com a extensão .rda. Para carregar de volta
o objeto, tem que ser feito com a função \texttt{load}

\begin{Shaded}
\begin{Highlighting}[]
\KeywordTok{args}\NormalTok{(load)}
\end{Highlighting}
\end{Shaded}

\begin{verbatim}
## function (file, envir = parent.frame(), verbose = FALSE) 
## NULL
\end{verbatim}

O que pode ser ruim, porque as vezes tu esqueceu o nome do objeto no
ambiente de R. Por exemplo, tu salvou o arquivo

\begin{Shaded}
\begin{Highlighting}[]
\KeywordTok{save}\NormalTok{(frenteFria, }\DataTypeTok{file =} \StringTok{"FrenteQuente.rda"}\NormalTok{)}
\end{Highlighting}
\end{Shaded}

logo tu carrega

\begin{Shaded}
\begin{Highlighting}[]
\KeywordTok{load}\NormalTok{(}\StringTok{"FrenteQuente.rda"}\NormalTok{)}
\end{Highlighting}
\end{Shaded}

acreditando que vai ter tua frente quente, mas o nome do objeto no
ambiente de R é frenteDria\ldots{} então, tem que ficar de olho, e como
somos imperfeito, vai dar merda\ldots{}.

O melhor da função é que permite salvar com tipos de compressão, por
exemplo compress = ``xz''.

\section{\texorpdfstring{Exportando objetos com
\texttt{saveRDS}}{Exportando objetos com saveRDS}}\label{exportando-objetos-com-saverds}

Esta é uma das minhas funçoes favoritas no R

\begin{Shaded}
\begin{Highlighting}[]
\KeywordTok{args}\NormalTok{(saveRDS)}
\end{Highlighting}
\end{Shaded}

\begin{verbatim}
## function (object, file = "", ascii = FALSE, version = NULL, compress = TRUE, 
##     refhook = NULL) 
## NULL
\end{verbatim}

e

\begin{Shaded}
\begin{Highlighting}[]
\KeywordTok{args}\NormalTok{(readRDS)}
\end{Highlighting}
\end{Shaded}

\begin{verbatim}
## function (file, refhook = NULL) 
## NULL
\end{verbatim}

Tu consegue salvar o objeto R de forma serializada e compactada com o
argumento \texttt{compress} mas o melhor é quando vai chamar o objeto de
volta ao R. Agora tu usa o \texttt{readRDS} e coloca o nome que tu
quiser.

\begin{Shaded}
\begin{Highlighting}[]
\KeywordTok{saveRDS}\NormalTok{(FrenteQuente, }\StringTok{"FrenteQuente.rds"}\NormalTok{)}
\end{Highlighting}
\end{Shaded}

\begin{Shaded}
\begin{Highlighting}[]
\NormalTok{frenteQ <-}\StringTok{ }\KeywordTok{readRDS}\NormalTok{(}\StringTok{"FremteQuente.rds"}\NormalTok{)}
\end{Highlighting}
\end{Shaded}

\section{Processando nossa
data-frame}\label{processando-nossa-data-frame}

Tem numeroas formas e pacotes para ordenar, arrangiar (Arrange), mutar e
cambiar as data-frames. As mais conhecidas são provablemente do universe
\emph{tidyverse} com o famoso pacote \emph{dplyr}. Mas, nesta curso
vamos focar em \textbf{base}.

Vamos então revisar a classe de cada columna do nosso data-frame com a
função \texttt{sapply}, apresentada em outro capitulo, mas se quiser, da
uma olhada em \texttt{?sapply}.

\begin{Shaded}
\begin{Highlighting}[]
\KeywordTok{sapply}\NormalTok{(df, class)}
\end{Highlighting}
\end{Shaded}

\begin{verbatim}
##          TipodeRede TipodeMonitoramento                Tipo 
##            "factor"            "factor"            "factor" 
##                Data                Hora       CodigoEstação 
##            "factor"            "factor"           "integer" 
##         NomeEstação       NomeParâmetro     UnidadedeMedida 
##            "factor"            "factor"            "factor" 
##        MediaHoraria          MediaMovel              Valido 
##           "integer"            "factor"            "factor"
\end{verbatim}

Quando nos trabalhamos com series de tempo, é importante ter a variabel
de tempo reconhecida como ``tempo'', especificamente como classe
``POSIXct''. Mas, a classe de Data é ``factor'' e de Hora tambem
``factor'', o que é ruim. Então, vamos criar uma variabel de tempo mais
standard com formato 2018-05-17 19:15:05.

Para isso temos que grudar as variabel Data e Hora. Faremios isso numa
nova varaibel chamada tempo\_char, adicionando ela diretamente no
\texttt{df} com o cifrão DOLLAR \$. O grude pode ser feito com as
funções \texttt{paste} ou \texttt{paste0}.

\begin{Shaded}
\begin{Highlighting}[]
\NormalTok{df}\OperatorTok{$}\NormalTok{tempo_char <-}\StringTok{ }\KeywordTok{paste}\NormalTok{(df}\OperatorTok{$}\NormalTok{Data, df}\OperatorTok{$}\NormalTok{Hora)}
\KeywordTok{head}\NormalTok{(df}\OperatorTok{$}\NormalTok{tempo_char)}
\end{Highlighting}
\end{Shaded}

\begin{verbatim}
## [1] "01/01/2014 01:00" "01/01/2014 02:00" "01/01/2014 03:00"
## [4] "01/01/2014 04:00" "01/01/2014 05:00" "01/01/2014 06:00"
\end{verbatim}

\begin{Shaded}
\begin{Highlighting}[]
\KeywordTok{class}\NormalTok{(df}\OperatorTok{$}\NormalTok{tempo_char)}
\end{Highlighting}
\end{Shaded}

\begin{verbatim}
## [1] "character"
\end{verbatim}

Esta melhorando mas ainda tem clase character.

Para convertir a nossa classe POSIXct podemos usar a função
\texttt{as.POSIXct} (olha \texttt{as.POSIXct}). Seus argumentos são:

\begin{Shaded}
\begin{Highlighting}[]
\KeywordTok{args}\NormalTok{(as.POSIXct)}
\end{Highlighting}
\end{Shaded}

\begin{verbatim}
## function (x, tz = "", ...) 
## NULL
\end{verbatim}

Então, vamos criar outra variabel tempo o formato POSIXct

\begin{Shaded}
\begin{Highlighting}[]
\NormalTok{df}\OperatorTok{$}\NormalTok{tempo <-}\StringTok{ }\KeywordTok{as.POSIXct}\NormalTok{(}\DataTypeTok{x =}\NormalTok{ df}\OperatorTok{$}\NormalTok{tempo_char, }\DataTypeTok{tz =} \StringTok{"Americas/Sao_Paulo"}\NormalTok{, }
                       \DataTypeTok{format =} \StringTok{"%d/%m/%Y %H:%M"}\NormalTok{)}
\KeywordTok{head}\NormalTok{(df}\OperatorTok{$}\NormalTok{tempo)}
\end{Highlighting}
\end{Shaded}

\begin{verbatim}
## [1] "2014-01-01 01:00:00 Americas" "2014-01-01 02:00:00 Americas"
## [3] "2014-01-01 03:00:00 Americas" "2014-01-01 04:00:00 Americas"
## [5] "2014-01-01 05:00:00 Americas" "2014-01-01 06:00:00 Americas"
\end{verbatim}

\begin{Shaded}
\begin{Highlighting}[]
\KeywordTok{class}\NormalTok{(df}\OperatorTok{$}\NormalTok{tempo)}
\end{Highlighting}
\end{Shaded}

\begin{verbatim}
## [1] "POSIXct" "POSIXt"
\end{verbatim}

Agora, vamos a extraer os dias da semana do tempo, mes e dia juliano:

\begin{Shaded}
\begin{Highlighting}[]
\NormalTok{df}\OperatorTok{$}\NormalTok{weekdays <-}\StringTok{ }\KeywordTok{format}\NormalTok{(df}\OperatorTok{$}\NormalTok{tempo, }\StringTok{"%A"}\NormalTok{)}
\KeywordTok{head}\NormalTok{(df}\OperatorTok{$}\NormalTok{weekdays)}
\end{Highlighting}
\end{Shaded}

\begin{verbatim}
## [1] "Wednesday" "Wednesday" "Wednesday" "Wednesday" "Wednesday" "Wednesday"
\end{verbatim}

\begin{Shaded}
\begin{Highlighting}[]
\NormalTok{df}\OperatorTok{$}\NormalTok{mes <-}\StringTok{ }\KeywordTok{format}\NormalTok{(df}\OperatorTok{$}\NormalTok{tempo, }\StringTok{"%B"}\NormalTok{)}
\KeywordTok{head}\NormalTok{(df}\OperatorTok{$}\NormalTok{mes)}
\end{Highlighting}
\end{Shaded}

\begin{verbatim}
## [1] "January" "January" "January" "January" "January" "January"
\end{verbatim}

\begin{Shaded}
\begin{Highlighting}[]
\NormalTok{df}\OperatorTok{$}\NormalTok{diajuliano <-}\StringTok{ }\KeywordTok{julian}\NormalTok{(df}\OperatorTok{$}\NormalTok{tempo)}
\KeywordTok{head}\NormalTok{(df}\OperatorTok{$}\NormalTok{diajuliano)}
\end{Highlighting}
\end{Shaded}

\begin{verbatim}
## Time differences in days
## [1] 16071.04 16071.08 16071.12 16071.17 16071.21 16071.25
\end{verbatim}

\begin{Shaded}
\begin{Highlighting}[]
\NormalTok{df}\OperatorTok{$}\NormalTok{ano <-}\StringTok{ }\KeywordTok{format}\NormalTok{(df}\OperatorTok{$}\NormalTok{tempo, }\StringTok{"%Y"}\NormalTok{)}
\end{Highlighting}
\end{Shaded}

\section{aggregate}\label{aggregate}

Vamos a carregar a nossa data.frame. Primero uma olhada

\begin{Shaded}
\begin{Highlighting}[]
\KeywordTok{head}\NormalTok{(df)}
\end{Highlighting}
\end{Shaded}

\begin{verbatim}
##   TipodeRede TipodeMonitoramento            Tipo       Data  Hora
## 2 Automático              CETESB Dados Primários 01/01/2014 01:00
## 3 Automático              CETESB Dados Primários 01/01/2014 02:00
## 4 Automático              CETESB Dados Primários 01/01/2014 03:00
## 5 Automático              CETESB Dados Primários 01/01/2014 04:00
## 6 Automático              CETESB Dados Primários 01/01/2014 05:00
## 7 Automático              CETESB Dados Primários 01/01/2014 06:00
##   CodigoEstação                NomeEstação              NomeParâmetro
## 2            95 Cid.Universitária-USP-Ipen NOx (Óxidos de Nitrogênio)
## 3            95 Cid.Universitária-USP-Ipen NOx (Óxidos de Nitrogênio)
## 4            95 Cid.Universitária-USP-Ipen NOx (Óxidos de Nitrogênio)
## 5            95 Cid.Universitária-USP-Ipen NOx (Óxidos de Nitrogênio)
## 6            95 Cid.Universitária-USP-Ipen NOx (Óxidos de Nitrogênio)
## 7            95 Cid.Universitária-USP-Ipen NOx (Óxidos de Nitrogênio)
##   UnidadedeMedida MediaHoraria MediaMovel Valido       tempo_char
## 2             ppb            9          -    Não 01/01/2014 01:00
## 3             ppb            9          -    Sim 01/01/2014 02:00
## 4             ppb            5          -    Sim 01/01/2014 03:00
## 5             ppb            4          -    Sim 01/01/2014 04:00
## 6             ppb            5          -    Sim 01/01/2014 05:00
## 7             ppb            5          -    Sim 01/01/2014 06:00
##                 tempo  weekdays     mes    diajuliano  ano
## 2 2014-01-01 01:00:00 Wednesday January 16071.04 days 2014
## 3 2014-01-01 02:00:00 Wednesday January 16071.08 days 2014
## 4 2014-01-01 03:00:00 Wednesday January 16071.12 days 2014
## 5 2014-01-01 04:00:00 Wednesday January 16071.17 days 2014
## 6 2014-01-01 05:00:00 Wednesday January 16071.21 days 2014
## 7 2014-01-01 06:00:00 Wednesday January 16071.25 days 2014
\end{verbatim}

Poderiamos calcular a media horaria por dia da semana. Então:

\begin{Shaded}
\begin{Highlighting}[]
\NormalTok{dff <-}\StringTok{ }\KeywordTok{aggregate}\NormalTok{(df}\OperatorTok{$}\NormalTok{MediaHoraria, }\DataTypeTok{by =} \KeywordTok{list}\NormalTok{(df}\OperatorTok{$}\NormalTok{weekdays), sum, }\DataTypeTok{na.rm =}\NormalTok{ T)}
\NormalTok{dff}
\end{Highlighting}
\end{Shaded}

\begin{verbatim}
##     Group.1     x
## 1    Friday 42558
## 2    Monday 34057
## 3  Saturday 32298
## 4    Sunday 20327
## 5  Thursday 41199
## 6   Tuesday 37904
## 7 Wednesday 40180
\end{verbatim}

\begin{Shaded}
\begin{Highlighting}[]
\KeywordTok{names}\NormalTok{(dff) <-}\StringTok{ }\KeywordTok{c}\NormalTok{(}\StringTok{"dias"}\NormalTok{, }\StringTok{"MediaHoraria"}\NormalTok{)}
\end{Highlighting}
\end{Shaded}

\begin{Shaded}
\begin{Highlighting}[]
\NormalTok{dff}\OperatorTok{$}\NormalTok{sd <-}\StringTok{ }\KeywordTok{aggregate}\NormalTok{(df}\OperatorTok{$}\NormalTok{MediaHoraria, }
                    \DataTypeTok{by =} \KeywordTok{list}\NormalTok{(df}\OperatorTok{$}\NormalTok{weekdays), }
\NormalTok{                    sum, }\DataTypeTok{na.rm =}\NormalTok{ T)}\OperatorTok{$}\NormalTok{x}
\NormalTok{dff}
\end{Highlighting}
\end{Shaded}

\begin{verbatim}
##        dias MediaHoraria    sd
## 1    Friday        42558 42558
## 2    Monday        34057 34057
## 3  Saturday        32298 32298
## 4    Sunday        20327 20327
## 5  Thursday        41199 41199
## 6   Tuesday        37904 37904
## 7 Wednesday        40180 40180
\end{verbatim}

\section{subset}\label{subset}

Como poderiamos escolher só o mes de janeiro??

\begin{Shaded}
\begin{Highlighting}[]
\CommentTok{#[     LINHAS    ,  COLUNAS   ]}
\KeywordTok{head}\NormalTok{(df[df}\OperatorTok{$}\NormalTok{mes }\OperatorTok{==}\StringTok{ "janeiro"}\NormalTok{, ]) }\CommentTok{#TODAS AS COLUNAS}
\end{Highlighting}
\end{Shaded}

\begin{verbatim}
##  [1] TipodeRede          TipodeMonitoramento Tipo               
##  [4] Data                Hora                CodigoEstação      
##  [7] NomeEstação         NomeParâmetro       UnidadedeMedida    
## [10] MediaHoraria        MediaMovel          Valido             
## [13] tempo_char          tempo               weekdays           
## [16] mes                 diajuliano          ano                
## <0 rows> (or 0-length row.names)
\end{verbatim}

Mes janeiro pero solo o valor mediahoraria, que retorna um vetor
numerico

\begin{Shaded}
\begin{Highlighting}[]
\KeywordTok{names}\NormalTok{(df)}
\end{Highlighting}
\end{Shaded}

\begin{verbatim}
##  [1] "TipodeRede"          "TipodeMonitoramento" "Tipo"               
##  [4] "Data"                "Hora"                "CodigoEstação"      
##  [7] "NomeEstação"         "NomeParâmetro"       "UnidadedeMedida"    
## [10] "MediaHoraria"        "MediaMovel"          "Valido"             
## [13] "tempo_char"          "tempo"               "weekdays"           
## [16] "mes"                 "diajuliano"          "ano"
\end{verbatim}

\begin{Shaded}
\begin{Highlighting}[]
\KeywordTok{head}\NormalTok{(df[df}\OperatorTok{$}\NormalTok{mes }\OperatorTok{==}\StringTok{ "janeiro"}\NormalTok{, }\DecValTok{10}\NormalTok{]) }
\end{Highlighting}
\end{Shaded}

\begin{verbatim}
## integer(0)
\end{verbatim}

\begin{Shaded}
\begin{Highlighting}[]
\KeywordTok{head}\NormalTok{(df[df}\OperatorTok{$}\NormalTok{mes }\OperatorTok{==}\StringTok{ "janeiro"}\NormalTok{, }\StringTok{"MediaHoraria"}\NormalTok{])}
\end{Highlighting}
\end{Shaded}

\begin{verbatim}
## integer(0)
\end{verbatim}

\begin{Shaded}
\begin{Highlighting}[]
\KeywordTok{class}\NormalTok{(df[df}\OperatorTok{$}\NormalTok{mes }\OperatorTok{==}\StringTok{ "janeiro"}\NormalTok{, }\StringTok{"MediaHoraria"}\NormalTok{])}
\end{Highlighting}
\end{Shaded}

\begin{verbatim}
## [1] "integer"
\end{verbatim}

Mas vamos salvar o nosso ``df''

\begin{Shaded}
\begin{Highlighting}[]
\KeywordTok{saveRDS}\NormalTok{(df, }\StringTok{"df.rds"}\NormalTok{)}
\end{Highlighting}
\end{Shaded}

\section{data.table, read\_xl e mais}\label{data.table-read_xl-e-mais}

data.table é um pacote que apresenta a classe \texttt{data.table}, que é
como uma versão melhorada da classe \texttt{data-frame} O termo
especifico é que \texttt{data-table} tem herencia (inherits) da classe
\texttt{data.frame}

Vamos ver como funciona data.table lendo o dois arquivos e comparar
quanto tempo demoram cada um.

\begin{Shaded}
\begin{Highlighting}[]
\NormalTok{df1 <-}\StringTok{ }\KeywordTok{print}\NormalTok{(}\KeywordTok{system.time}\NormalTok{(}\KeywordTok{read.table}\NormalTok{(}\StringTok{"https://raw.githubusercontent.com/ibarraespinosa/cursoR/master/dados/NOXIPEN2014.txt"}\NormalTok{, }\DataTypeTok{h =}\NormalTok{ T)))}
\end{Highlighting}
\end{Shaded}

\begin{verbatim}
##    user  system elapsed 
##   0.065   0.000   0.185
\end{verbatim}

\begin{Shaded}
\begin{Highlighting}[]
\KeywordTok{library}\NormalTok{(data.table)}
\NormalTok{df2 <-}\StringTok{ }\KeywordTok{print}\NormalTok{(}\KeywordTok{system.time}\NormalTok{(}\KeywordTok{fread}\NormalTok{(}\StringTok{"https://raw.githubusercontent.com/ibarraespinosa/cursoR/master/dados/NOXIPEN2014.txt"}\NormalTok{, }\DataTypeTok{h =}\NormalTok{ T)))}
\end{Highlighting}
\end{Shaded}

\begin{verbatim}
##    user  system elapsed 
##   0.017   0.004   0.028
\end{verbatim}

olha que estamos usando a função \texttt{fread}.

read\_xl é mais uma função do universo tidyverse que permite importar
excel no R, diretamente e inteligentemente.

\section{NetCDF}\label{netcdf}

O NetCDF (Network Common Data Form) é um conjunto de bibliotecas de
software e formatos de dados independentes de máquina e autodescritivos
com suporte para criação, acesso e compartilhamento de dados científicos
orientados a matrizes. Arquivos NetCDF (criado por essa biblioteca ou
por programas que utilizam essa biblioteca) são arquivos compostos por
dados, atributos e metadados.

O pacote \texttt{ncdf4} pode ser usado para acessar a essa biblioteca,
os comandos abaixo instalam e carregam esse pacote:

\begin{Shaded}
\begin{Highlighting}[]
\CommentTok{#install.packages("ncdf4") # instala o pacote}
\KeywordTok{library}\NormalTok{(}\StringTok{"ncdf4"}\NormalTok{)          }\CommentTok{# carrega o pacote}
\KeywordTok{nc_version}\NormalTok{()              }\CommentTok{# que retorna a versão da biblioteca}
\end{Highlighting}
\end{Shaded}

\begin{verbatim}
## [1] "ncdf4_1.16_20170401"
\end{verbatim}

Um exmplo de NetCDF:

\begin{Shaded}
\begin{Highlighting}[]
\KeywordTok{download.file}\NormalTok{(}\StringTok{"https://github.com/ibarraespinosa/cursoR/raw/master/dados/met_em.d03.2016-01-10.nc"}\NormalTok{, }\DataTypeTok{destfile =} \StringTok{"~/met_em.d03.2016-01-10.nc"}\NormalTok{)}
\end{Highlighting}
\end{Shaded}

\begin{Shaded}
\begin{Highlighting}[]
\NormalTok{wrf <-}\StringTok{ }\NormalTok{ncdf4}\OperatorTok{::}\KeywordTok{nc_open}\NormalTok{(}\StringTok{"~/met_em.d03.2016-01-10.nc"}\NormalTok{)}
\end{Highlighting}
\end{Shaded}

O objeto \texttt{wrf} contém algumas informações sobre o conteúdo do
arquivo, com um \texttt{print(wrf)} ou simplesmente \texttt{wrf}
visualizamos o conteúdo do arquivo:

\begin{Shaded}
\begin{Highlighting}[]
\KeywordTok{class}\NormalTok{(wrf)}
\end{Highlighting}
\end{Shaded}

\begin{verbatim}
## [1] "ncdf4"
\end{verbatim}

\begin{Shaded}
\begin{Highlighting}[]
\NormalTok{wrf}
\end{Highlighting}
\end{Shaded}

\begin{verbatim}
## File ~/met_em.d03.2016-01-10.nc (NC_FORMAT_64BIT):
## 
##      92 variables (excluding dimension variables):
##         char Times[DateStrLen,Time]   
##         float PRES[west_east,south_north,num_metgrid_levels,Time]   
##             FieldType: 104
##             MemoryOrder: XYZ
##             units: 
##             description: 
##             stagger: M
##             sr_x: 1
##             sr_y: 1
##         float SOIL_LAYERS[west_east,south_north,num_st_layers,Time]   
##             FieldType: 104
##             MemoryOrder: XYZ
##             units: 
##             description: 
##             stagger: M
##             sr_x: 1
##             sr_y: 1
##         float SM[west_east,south_north,num_sm_layers,Time]   
##             FieldType: 104
##             MemoryOrder: XYZ
##             units: 
##             description: 
##             stagger: M
##             sr_x: 1
##             sr_y: 1
##         float ST[west_east,south_north,num_st_layers,Time]   
##             FieldType: 104
##             MemoryOrder: XYZ
##             units: 
##             description: 
##             stagger: M
##             sr_x: 1
##             sr_y: 1
##         float GHT[west_east,south_north,num_metgrid_levels,Time]   
##             FieldType: 104
##             MemoryOrder: XYZ
##             units: m
##             description: Height
##             stagger: M
##             sr_x: 1
##             sr_y: 1
##         float HGTTROP[west_east,south_north,Time]   
##             FieldType: 104
##             MemoryOrder: XY 
##             units: m
##             description: Height of tropopause
##             stagger: M
##             sr_x: 1
##             sr_y: 1
##         float TTROP[west_east,south_north,Time]   
##             FieldType: 104
##             MemoryOrder: XY 
##             units: K
##             description: Temperature at tropopause
##             stagger: M
##             sr_x: 1
##             sr_y: 1
##         float PTROPNN[west_east,south_north,Time]   
##             FieldType: 104
##             MemoryOrder: XY 
##             units: Pa
##             description: PTROP, used for nearest neighbor interp
##             stagger: M
##             sr_x: 1
##             sr_y: 1
##         float PTROP[west_east,south_north,Time]   
##             FieldType: 104
##             MemoryOrder: XY 
##             units: Pa
##             description: Pressure of tropopause
##             stagger: M
##             sr_x: 1
##             sr_y: 1
##         float VTROP[west_east,south_north_stag,Time]   
##             FieldType: 104
##             MemoryOrder: XY 
##             units: m s-1
##             description: V                 at tropopause
##             stagger: V
##             sr_x: 1
##             sr_y: 1
##         float UTROP[west_east_stag,south_north,Time]   
##             FieldType: 104
##             MemoryOrder: XY 
##             units: m s-1
##             description: U                 at tropopause
##             stagger: U
##             sr_x: 1
##             sr_y: 1
##         float HGTMAXW[west_east,south_north,Time]   
##             FieldType: 104
##             MemoryOrder: XY 
##             units: m
##             description: Height of max wind level
##             stagger: M
##             sr_x: 1
##             sr_y: 1
##         float TMAXW[west_east,south_north,Time]   
##             FieldType: 104
##             MemoryOrder: XY 
##             units: K
##             description: Temperature at max wind level
##             stagger: M
##             sr_x: 1
##             sr_y: 1
##         float PMAXWNN[west_east,south_north,Time]   
##             FieldType: 104
##             MemoryOrder: XY 
##             units: Pa
##             description: PMAXW, used for nearest neighbor interp
##             stagger: M
##             sr_x: 1
##             sr_y: 1
##         float PMAXW[west_east,south_north,Time]   
##             FieldType: 104
##             MemoryOrder: XY 
##             units: Pa
##             description: Pressure of max wind level
##             stagger: M
##             sr_x: 1
##             sr_y: 1
##         float VMAXW[west_east,south_north_stag,Time]   
##             FieldType: 104
##             MemoryOrder: XY 
##             units: m s-1
##             description: V                 at max wind
##             stagger: V
##             sr_x: 1
##             sr_y: 1
##         float UMAXW[west_east_stag,south_north,Time]   
##             FieldType: 104
##             MemoryOrder: XY 
##             units: m s-1
##             description: U                 at max wind
##             stagger: U
##             sr_x: 1
##             sr_y: 1
##         float SNOWH[west_east,south_north,Time]   
##             FieldType: 104
##             MemoryOrder: XY 
##             units: m
##             description: Physical Snow Depth
##             stagger: M
##             sr_x: 1
##             sr_y: 1
##         float SNOW[west_east,south_north,Time]   
##             FieldType: 104
##             MemoryOrder: XY 
##             units: kg m-2
##             description: Water equivalent snow depth
##             stagger: M
##             sr_x: 1
##             sr_y: 1
##         float SKINTEMP[west_east,south_north,Time]   
##             FieldType: 104
##             MemoryOrder: XY 
##             units: K
##             description: Skin temperature
##             stagger: M
##             sr_x: 1
##             sr_y: 1
##         float SOILHGT[west_east,south_north,Time]   
##             FieldType: 104
##             MemoryOrder: XY 
##             units: m
##             description: Terrain field of source analysis
##             stagger: M
##             sr_x: 1
##             sr_y: 1
##         float LANDSEA[west_east,south_north,Time]   
##             FieldType: 104
##             MemoryOrder: XY 
##             units: proprtn
##             description: Land/Sea flag (1=land, 0 or 2=sea)
##             stagger: M
##             sr_x: 1
##             sr_y: 1
##         float SEAICE[west_east,south_north,Time]   
##             FieldType: 104
##             MemoryOrder: XY 
##             units: proprtn
##             description: Ice flag
##             stagger: M
##             sr_x: 1
##             sr_y: 1
##         float ST100200[west_east,south_north,Time]   
##             FieldType: 104
##             MemoryOrder: XY 
##             units: K
##             description: T 100-200 cm below ground layer (Bottom)
##             stagger: M
##             sr_x: 1
##             sr_y: 1
##         float ST040100[west_east,south_north,Time]   
##             FieldType: 104
##             MemoryOrder: XY 
##             units: K
##             description: T 40-100 cm below ground layer (Upper)
##             stagger: M
##             sr_x: 1
##             sr_y: 1
##         float ST010040[west_east,south_north,Time]   
##             FieldType: 104
##             MemoryOrder: XY 
##             units: K
##             description: T 10-40 cm below ground layer (Upper)
##             stagger: M
##             sr_x: 1
##             sr_y: 1
##         float ST000010[west_east,south_north,Time]   
##             FieldType: 104
##             MemoryOrder: XY 
##             units: K
##             description: T 0-10 cm below ground layer (Upper)
##             stagger: M
##             sr_x: 1
##             sr_y: 1
##         float SM100200[west_east,south_north,Time]   
##             FieldType: 104
##             MemoryOrder: XY 
##             units: fraction
##             description: Soil Moist 100-200 cm below gr layer
##             stagger: M
##             sr_x: 1
##             sr_y: 1
##         float SM040100[west_east,south_north,Time]   
##             FieldType: 104
##             MemoryOrder: XY 
##             units: fraction
##             description: Soil Moist 40-100 cm below grn layer
##             stagger: M
##             sr_x: 1
##             sr_y: 1
##         float SM010040[west_east,south_north,Time]   
##             FieldType: 104
##             MemoryOrder: XY 
##             units: fraction
##             description: Soil Moist 10-40 cm below grn layer
##             stagger: M
##             sr_x: 1
##             sr_y: 1
##         float SM000010[west_east,south_north,Time]   
##             FieldType: 104
##             MemoryOrder: XY 
##             units: fraction
##             description: Soil Moist 0-10 cm below grn layer (Up)
##             stagger: M
##             sr_x: 1
##             sr_y: 1
##         float PSFC[west_east,south_north,Time]   
##             FieldType: 104
##             MemoryOrder: XY 
##             units: Pa
##             description: Surface Pressure
##             stagger: M
##             sr_x: 1
##             sr_y: 1
##         float RH[west_east,south_north,num_metgrid_levels,Time]   
##             FieldType: 104
##             MemoryOrder: XYZ
##             units: %
##             description: Relative Humidity
##             stagger: M
##             sr_x: 1
##             sr_y: 1
##         float VV[west_east,south_north_stag,num_metgrid_levels,Time]   
##             FieldType: 104
##             MemoryOrder: XYZ
##             units: m s-1
##             description: V
##             stagger: V
##             sr_x: 1
##             sr_y: 1
##         float UU[west_east_stag,south_north,num_metgrid_levels,Time]   
##             FieldType: 104
##             MemoryOrder: XYZ
##             units: m s-1
##             description: U
##             stagger: U
##             sr_x: 1
##             sr_y: 1
##         float TT[west_east,south_north,num_metgrid_levels,Time]   
##             FieldType: 104
##             MemoryOrder: XYZ
##             units: K
##             description: Temperature
##             stagger: M
##             sr_x: 1
##             sr_y: 1
##         float PMSL[west_east,south_north,Time]   
##             FieldType: 104
##             MemoryOrder: XY 
##             units: Pa
##             description: Sea-level Pressure
##             stagger: M
##             sr_x: 1
##             sr_y: 1
##         float URB_PARAM[west_east,south_north,z-dimension0132,Time]   
##             FieldType: 104
##             MemoryOrder: XYZ
##             units: dimensionless
##             description: Urban_Parameters
##             stagger: M
##             sr_x: 1
##             sr_y: 1
##         float LAKE_DEPTH[west_east,south_north,Time]   
##             FieldType: 104
##             MemoryOrder: XY 
##             units: meters MSL
##             description: Topography height
##             stagger: M
##             sr_x: 1
##             sr_y: 1
##         float VAR_SSO[west_east,south_north,Time]   
##             FieldType: 104
##             MemoryOrder: XY 
##             units: meters2 MSL
##             description: Variance of Subgrid Scale Orography
##             stagger: M
##             sr_x: 1
##             sr_y: 1
##         float OL4[west_east,south_north,Time]   
##             FieldType: 104
##             MemoryOrder: XY 
##             units: whoknows
##             description: something
##             stagger: M
##             sr_x: 1
##             sr_y: 1
##         float OL3[west_east,south_north,Time]   
##             FieldType: 104
##             MemoryOrder: XY 
##             units: whoknows
##             description: something
##             stagger: M
##             sr_x: 1
##             sr_y: 1
##         float OL2[west_east,south_north,Time]   
##             FieldType: 104
##             MemoryOrder: XY 
##             units: whoknows
##             description: something
##             stagger: M
##             sr_x: 1
##             sr_y: 1
##         float OL1[west_east,south_north,Time]   
##             FieldType: 104
##             MemoryOrder: XY 
##             units: whoknows
##             description: something
##             stagger: M
##             sr_x: 1
##             sr_y: 1
##         float OA4[west_east,south_north,Time]   
##             FieldType: 104
##             MemoryOrder: XY 
##             units: whoknows
##             description: something
##             stagger: M
##             sr_x: 1
##             sr_y: 1
##         float OA3[west_east,south_north,Time]   
##             FieldType: 104
##             MemoryOrder: XY 
##             units: whoknows
##             description: something
##             stagger: M
##             sr_x: 1
##             sr_y: 1
##         float OA2[west_east,south_north,Time]   
##             FieldType: 104
##             MemoryOrder: XY 
##             units: whoknows
##             description: something
##             stagger: M
##             sr_x: 1
##             sr_y: 1
##         float OA1[west_east,south_north,Time]   
##             FieldType: 104
##             MemoryOrder: XY 
##             units: whoknows
##             description: something
##             stagger: M
##             sr_x: 1
##             sr_y: 1
##         float VAR[west_east,south_north,Time]   
##             FieldType: 104
##             MemoryOrder: XY 
##             units: whoknows
##             description: something
##             stagger: M
##             sr_x: 1
##             sr_y: 1
##         float CON[west_east,south_north,Time]   
##             FieldType: 104
##             MemoryOrder: XY 
##             units: whoknows
##             description: something
##             stagger: M
##             sr_x: 1
##             sr_y: 1
##         float SLOPECAT[west_east,south_north,Time]   
##             FieldType: 104
##             MemoryOrder: XY 
##             units: category
##             description: Dominant category
##             stagger: M
##             sr_x: 1
##             sr_y: 1
##         float SNOALB[west_east,south_north,Time]   
##             FieldType: 104
##             MemoryOrder: XY 
##             units: percent
##             description: Maximum snow albedo
##             stagger: M
##             sr_x: 1
##             sr_y: 1
##         float LAI12M[west_east,south_north,z-dimension0012,Time]   
##             FieldType: 104
##             MemoryOrder: XYZ
##             units: m^2/m^2
##             description: MODIS LAI
##             stagger: M
##             sr_x: 1
##             sr_y: 1
##         float GREENFRAC[west_east,south_north,z-dimension0012,Time]   
##             FieldType: 104
##             MemoryOrder: XYZ
##             units: fraction
##             description: MODIS FPAR
##             stagger: M
##             sr_x: 1
##             sr_y: 1
##         float ALBEDO12M[west_east,south_north,z-dimension0012,Time]   
##             FieldType: 104
##             MemoryOrder: XYZ
##             units: percent
##             description: Monthly surface albedo
##             stagger: M
##             sr_x: 1
##             sr_y: 1
##         float SCB_DOM[west_east,south_north,Time]   
##             FieldType: 104
##             MemoryOrder: XY 
##             units: category
##             description: Dominant category
##             stagger: M
##             sr_x: 1
##             sr_y: 1
##         float SOILCBOT[west_east,south_north,z-dimension0016,Time]   
##             FieldType: 104
##             MemoryOrder: XYZ
##             units: category
##             description: 16-category bottom-layer soil type
##             stagger: M
##             sr_x: 1
##             sr_y: 1
##         float SCT_DOM[west_east,south_north,Time]   
##             FieldType: 104
##             MemoryOrder: XY 
##             units: category
##             description: Dominant category
##             stagger: M
##             sr_x: 1
##             sr_y: 1
##         float SOILCTOP[west_east,south_north,z-dimension0016,Time]   
##             FieldType: 104
##             MemoryOrder: XYZ
##             units: category
##             description: 16-category top-layer soil type
##             stagger: M
##             sr_x: 1
##             sr_y: 1
##         float SOILTEMP[west_east,south_north,Time]   
##             FieldType: 104
##             MemoryOrder: XY 
##             units: Kelvin
##             description: Annual mean deep soil temperature
##             stagger: M
##             sr_x: 1
##             sr_y: 1
##         float HGT_M[west_east,south_north,Time]   
##             FieldType: 104
##             MemoryOrder: XY 
##             units: meters MSL
##             description: GMTED2010 30-arc-second topography height
##             stagger: M
##             sr_x: 1
##             sr_y: 1
##         float LU_INDEX[west_east,south_north,Time]   
##             FieldType: 104
##             MemoryOrder: XY 
##             units: category
##             description: Dominant category
##             stagger: M
##             sr_x: 1
##             sr_y: 1
##         float LANDUSEF[west_east,south_north,z-dimension0024,Time]   
##             FieldType: 104
##             MemoryOrder: XYZ
##             units: category
##             description: 24-category USGS landuse
##             stagger: M
##             sr_x: 1
##             sr_y: 1
##         float COSALPHA_V[west_east,south_north_stag,Time]   
##             FieldType: 104
##             MemoryOrder: XY 
##             units: none
##             description: Cosine of rotation angle on V grid
##             stagger: V
##             sr_x: 1
##             sr_y: 1
##         float SINALPHA_V[west_east,south_north_stag,Time]   
##             FieldType: 104
##             MemoryOrder: XY 
##             units: none
##             description: Sine of rotation angle on V grid
##             stagger: V
##             sr_x: 1
##             sr_y: 1
##         float COSALPHA_U[west_east_stag,south_north,Time]   
##             FieldType: 104
##             MemoryOrder: XY 
##             units: none
##             description: Cosine of rotation angle on U grid
##             stagger: U
##             sr_x: 1
##             sr_y: 1
##         float SINALPHA_U[west_east_stag,south_north,Time]   
##             FieldType: 104
##             MemoryOrder: XY 
##             units: none
##             description: Sine of rotation angle on U grid
##             stagger: U
##             sr_x: 1
##             sr_y: 1
##         float XLONG_C[west_east_stag,south_north_stag,Time]   
##             FieldType: 104
##             MemoryOrder: XY 
##             units: degrees longitude
##             description: Longitude at grid cell corners
##             stagger: CORNER
##             sr_x: 1
##             sr_y: 1
##         float XLAT_C[west_east_stag,south_north_stag,Time]   
##             FieldType: 104
##             MemoryOrder: XY 
##             units: degrees latitude
##             description: Latitude at grid cell corners
##             stagger: CORNER
##             sr_x: 1
##             sr_y: 1
##         float LANDMASK[west_east,south_north,Time]   
##             FieldType: 104
##             MemoryOrder: XY 
##             units: none
##             description: Landmask : 1=land, 0=water
##             stagger: M
##             sr_x: 1
##             sr_y: 1
##         float COSALPHA[west_east,south_north,Time]   
##             FieldType: 104
##             MemoryOrder: XY 
##             units: none
##             description: Cosine of rotation angle
##             stagger: M
##             sr_x: 1
##             sr_y: 1
##         float SINALPHA[west_east,south_north,Time]   
##             FieldType: 104
##             MemoryOrder: XY 
##             units: none
##             description: Sine of rotation angle
##             stagger: M
##             sr_x: 1
##             sr_y: 1
##         float F[west_east,south_north,Time]   
##             FieldType: 104
##             MemoryOrder: XY 
##             units: -
##             description: Coriolis F parameter
##             stagger: M
##             sr_x: 1
##             sr_y: 1
##         float E[west_east,south_north,Time]   
##             FieldType: 104
##             MemoryOrder: XY 
##             units: -
##             description: Coriolis E parameter
##             stagger: M
##             sr_x: 1
##             sr_y: 1
##         float MAPFAC_UY[west_east_stag,south_north,Time]   
##             FieldType: 104
##             MemoryOrder: XY 
##             units: none
##             description: Mapfactor (y-dir) on U grid
##             stagger: U
##             sr_x: 1
##             sr_y: 1
##         float MAPFAC_VY[west_east,south_north_stag,Time]   
##             FieldType: 104
##             MemoryOrder: XY 
##             units: none
##             description: Mapfactor (y-dir) on V grid
##             stagger: V
##             sr_x: 1
##             sr_y: 1
##         float MAPFAC_MY[west_east,south_north,Time]   
##             FieldType: 104
##             MemoryOrder: XY 
##             units: none
##             description: Mapfactor (y-dir) on mass grid
##             stagger: M
##             sr_x: 1
##             sr_y: 1
##         float MAPFAC_UX[west_east_stag,south_north,Time]   
##             FieldType: 104
##             MemoryOrder: XY 
##             units: none
##             description: Mapfactor (x-dir) on U grid
##             stagger: U
##             sr_x: 1
##             sr_y: 1
##         float MAPFAC_VX[west_east,south_north_stag,Time]   
##             FieldType: 104
##             MemoryOrder: XY 
##             units: none
##             description: Mapfactor (x-dir) on V grid
##             stagger: V
##             sr_x: 1
##             sr_y: 1
##         float MAPFAC_MX[west_east,south_north,Time]   
##             FieldType: 104
##             MemoryOrder: XY 
##             units: none
##             description: Mapfactor (x-dir) on mass grid
##             stagger: M
##             sr_x: 1
##             sr_y: 1
##         float MAPFAC_U[west_east_stag,south_north,Time]   
##             FieldType: 104
##             MemoryOrder: XY 
##             units: none
##             description: Mapfactor on U grid
##             stagger: U
##             sr_x: 1
##             sr_y: 1
##         float MAPFAC_V[west_east,south_north_stag,Time]   
##             FieldType: 104
##             MemoryOrder: XY 
##             units: none
##             description: Mapfactor on V grid
##             stagger: V
##             sr_x: 1
##             sr_y: 1
##         float MAPFAC_M[west_east,south_north,Time]   
##             FieldType: 104
##             MemoryOrder: XY 
##             units: none
##             description: Mapfactor on mass grid
##             stagger: M
##             sr_x: 1
##             sr_y: 1
##         float CLONG[west_east,south_north,Time]   
##             FieldType: 104
##             MemoryOrder: XY 
##             units: degrees longitude
##             description: Computational longitude on mass grid
##             stagger: M
##             sr_x: 1
##             sr_y: 1
##         float CLAT[west_east,south_north,Time]   
##             FieldType: 104
##             MemoryOrder: XY 
##             units: degrees latitude
##             description: Computational latitude on mass grid
##             stagger: M
##             sr_x: 1
##             sr_y: 1
##         float XLONG_U[west_east_stag,south_north,Time]   
##             FieldType: 104
##             MemoryOrder: XY 
##             units: degrees longitude
##             description: Longitude on U grid
##             stagger: U
##             sr_x: 1
##             sr_y: 1
##         float XLAT_U[west_east_stag,south_north,Time]   
##             FieldType: 104
##             MemoryOrder: XY 
##             units: degrees latitude
##             description: Latitude on U grid
##             stagger: U
##             sr_x: 1
##             sr_y: 1
##         float XLONG_V[west_east,south_north_stag,Time]   
##             FieldType: 104
##             MemoryOrder: XY 
##             units: degrees longitude
##             description: Longitude on V grid
##             stagger: V
##             sr_x: 1
##             sr_y: 1
##         float XLAT_V[west_east,south_north_stag,Time]   
##             FieldType: 104
##             MemoryOrder: XY 
##             units: degrees latitude
##             description: Latitude on V grid
##             stagger: V
##             sr_x: 1
##             sr_y: 1
##         float XLONG_M[west_east,south_north,Time]   
##             FieldType: 104
##             MemoryOrder: XY 
##             units: degrees longitude
##             description: Longitude on mass grid
##             stagger: M
##             sr_x: 1
##             sr_y: 1
##         float XLAT_M[west_east,south_north,Time]   
##             FieldType: 104
##             MemoryOrder: XY 
##             units: degrees latitude
##             description: Latitude on mass grid
##             stagger: M
##             sr_x: 1
##             sr_y: 1
## 
##      13 dimensions:
##         Time  Size:1   *** is unlimited ***
##         DateStrLen  Size:19
##         west_east  Size:51
##         south_north  Size:51
##         num_metgrid_levels  Size:27
##         num_st_layers  Size:4
##         num_sm_layers  Size:4
##         south_north_stag  Size:52
##         west_east_stag  Size:52
##         z-dimension0132  Size:132
##         z-dimension0012  Size:12
##         z-dimension0016  Size:16
##         z-dimension0024  Size:24
## 
##     76 global attributes:
##         TITLE: OUTPUT FROM METGRID V3.9.1
##         SIMULATION_START_DATE: 2016-01-10_00:00:00
##         WEST-EAST_GRID_DIMENSION: 52
##         SOUTH-NORTH_GRID_DIMENSION: 52
##         BOTTOM-TOP_GRID_DIMENSION: 27
##         WEST-EAST_PATCH_START_UNSTAG: 1
##         WEST-EAST_PATCH_END_UNSTAG: 51
##         WEST-EAST_PATCH_START_STAG: 1
##         WEST-EAST_PATCH_END_STAG: 52
##         SOUTH-NORTH_PATCH_START_UNSTAG: 1
##         SOUTH-NORTH_PATCH_END_UNSTAG: 51
##         SOUTH-NORTH_PATCH_START_STAG: 1
##         SOUTH-NORTH_PATCH_END_STAG: 52
##         GRIDTYPE: C
##         DX: 1000
##         DY: 1000
##         DYN_OPT: 2
##         CEN_LAT: -23.5996932983398
##         CEN_LON: -46.6294555664062
##         TRUELAT1: -23
##         TRUELAT2: -24
##         MOAD_CEN_LAT: -23.6000061035156
##         STAND_LON: -45
##         POLE_LAT: 90
##         POLE_LON: 0
##         corner_lats: -23.8218078613281
##          corner_lats: -23.3720855712891
##          corner_lats: -23.3771743774414
##          corner_lats: -23.826904296875
##          corner_lats: -23.8217391967773
##          corner_lats: -23.3720245361328
##          corner_lats: -23.3772277832031
##          corner_lats: -23.8269424438477
##          corner_lats: -23.826286315918
##          corner_lats: -23.3675918579102
##          corner_lats: -23.372673034668
##          corner_lats: -23.8314056396484
##          corner_lats: -23.8262329101562
##          corner_lats: -23.3675231933594
##          corner_lats: -23.3727111816406
##          corner_lats: -23.8314437866211
##         corner_lons: -46.8780517578125
##          corner_lons: -46.8716430664062
##          corner_lons: -46.3817138671875
##          corner_lons: -46.3864440917969
##          corner_lons: -46.8829650878906
##          corner_lons: -46.8765258789062
##          corner_lons: -46.3768005371094
##          corner_lons: -46.3815307617188
##          corner_lons: -46.8781127929688
##          corner_lons: -46.87158203125
##          corner_lons: -46.3816528320312
##          corner_lons: -46.386474609375
##          corner_lons: -46.8830261230469
##          corner_lons: -46.87646484375
##          corner_lons: -46.3767700195312
##          corner_lons: -46.3815612792969
##         MAP_PROJ: 1
##         MMINLU: USGS
##         NUM_LAND_CAT: 24
##         ISWATER: 16
##         ISLAKE: -1
##         ISICE: 24
##         ISURBAN: 1
##         ISOILWATER: 14
##         grid_id: 3
##         parent_id: 2
##         i_parent_start: 35
##         j_parent_start: 33
##         i_parent_end: 51
##         j_parent_end: 49
##         parent_grid_ratio: 3
##         sr_x: 1
##         sr_y: 1
##         NUM_METGRID_SOIL_LEVELS: 4
##         FLAG_METGRID: 1
##         FLAG_EXCLUDED_MIDDLE: 0
##         FLAG_SOIL_LAYERS: 1
##         FLAG_SNOW: 1
##         FLAG_PSFC: 1
##         FLAG_SM000010: 1
##         FLAG_SM010040: 1
##         FLAG_SM040100: 1
##         FLAG_SM100200: 1
##         FLAG_ST000010: 1
##         FLAG_ST010040: 1
##         FLAG_ST040100: 1
##         FLAG_ST100200: 1
##         FLAG_SLP: 1
##         FLAG_SNOWH: 1
##         FLAG_SOILHGT: 1
##         FLAG_UTROP: 1
##         FLAG_VTROP: 1
##         FLAG_TTROP: 1
##         FLAG_PTROP: 1
##         FLAG_PTROPNN: 1
##         FLAG_HGTTROP: 1
##         FLAG_UMAXW: 1
##         FLAG_VMAXW: 1
##         FLAG_TMAXW: 1
##         FLAG_PMAXW: 1
##         FLAG_PMAXWNN: 1
##         FLAG_HGTMAXW: 1
##         FLAG_MF_XY: 1
##         FLAG_LAI12M: 1
##         FLAG_LAKE_DEPTH: 1
\end{verbatim}

que mostra o nome do arquivo (e versão da biblioteca usada para criar),
número de variáveis (92 no arquivo de exemplo), uma descrição de cada
variável (incluindo atributos) as dimensões (13 para esse arquivo) e os
atributos globais.

Agora vamos abrir alguma variável:

\begin{Shaded}
\begin{Highlighting}[]
\KeywordTok{names}\NormalTok{(wrf}\OperatorTok{$}\NormalTok{var)                }\CommentTok{# print no nome de cada variavel}
\end{Highlighting}
\end{Shaded}

\begin{verbatim}
##  [1] "Times"       "PRES"        "SOIL_LAYERS" "SM"          "ST"         
##  [6] "GHT"         "HGTTROP"     "TTROP"       "PTROPNN"     "PTROP"      
## [11] "VTROP"       "UTROP"       "HGTMAXW"     "TMAXW"       "PMAXWNN"    
## [16] "PMAXW"       "VMAXW"       "UMAXW"       "SNOWH"       "SNOW"       
## [21] "SKINTEMP"    "SOILHGT"     "LANDSEA"     "SEAICE"      "ST100200"   
## [26] "ST040100"    "ST010040"    "ST000010"    "SM100200"    "SM040100"   
## [31] "SM010040"    "SM000010"    "PSFC"        "RH"          "VV"         
## [36] "UU"          "TT"          "PMSL"        "URB_PARAM"   "LAKE_DEPTH" 
## [41] "VAR_SSO"     "OL4"         "OL3"         "OL2"         "OL1"        
## [46] "OA4"         "OA3"         "OA2"         "OA1"         "VAR"        
## [51] "CON"         "SLOPECAT"    "SNOALB"      "LAI12M"      "GREENFRAC"  
## [56] "ALBEDO12M"   "SCB_DOM"     "SOILCBOT"    "SCT_DOM"     "SOILCTOP"   
## [61] "SOILTEMP"    "HGT_M"       "LU_INDEX"    "LANDUSEF"    "COSALPHA_V" 
## [66] "SINALPHA_V"  "COSALPHA_U"  "SINALPHA_U"  "XLONG_C"     "XLAT_C"     
## [71] "LANDMASK"    "COSALPHA"    "SINALPHA"    "F"           "E"          
## [76] "MAPFAC_UY"   "MAPFAC_VY"   "MAPFAC_MY"   "MAPFAC_UX"   "MAPFAC_VX"  
## [81] "MAPFAC_MX"   "MAPFAC_U"    "MAPFAC_V"    "MAPFAC_M"    "CLONG"      
## [86] "CLAT"        "XLONG_U"     "XLAT_U"      "XLONG_V"     "XLAT_V"     
## [91] "XLONG_M"     "XLAT_M"
\end{verbatim}

\begin{Shaded}
\begin{Highlighting}[]
\NormalTok{TEMP <-}\StringTok{ }\NormalTok{ncdf4}\OperatorTok{::}\KeywordTok{ncvar_get}\NormalTok{(wrf, }\StringTok{"TT"}\NormalTok{)  }\CommentTok{# escolho você picachu}
\KeywordTok{class}\NormalTok{(TEMP)}
\end{Highlighting}
\end{Shaded}

\begin{verbatim}
## [1] "array"
\end{verbatim}

Como o NetCDF é organizado para guardar matrizes (arrays), só sabemos
que a variável \texttt{ST} é um array

\begin{Shaded}
\begin{Highlighting}[]
\KeywordTok{ncatt_get}\NormalTok{(wrf,}\StringTok{"TT"}\NormalTok{) }\CommentTok{# ou ncatt_get(wrf,"TT",verbose = T)}
\end{Highlighting}
\end{Shaded}

\begin{verbatim}
## $FieldType
## [1] 104
## 
## $MemoryOrder
## [1] "XYZ"
## 
## $units
## [1] "K"
## 
## $description
## [1] "Temperature"
## 
## $stagger
## [1] "M"
## 
## $sr_x
## [1] 1
## 
## $sr_y
## [1] 1
\end{verbatim}

\begin{Shaded}
\begin{Highlighting}[]
\KeywordTok{dim}\NormalTok{(TEMP)}
\end{Highlighting}
\end{Shaded}

\begin{verbatim}
## [1] 51 51 27
\end{verbatim}

praticamente a mesma informação do print anterior:

\begin{verbatim}
float TT[west_east,south_north,num_metgrid_levels,Time]   
FieldType: 104
MemoryOrder: XYZ
units: K
description: Temperature
stagger: M
sr_x: 1
sr_y: 1
\end{verbatim}

como temos apenas 1 tempo essa dimensão é desconsiderada para
simplificar.

A latitude de cada ponto de grade, assim como longitude níveis e tempo
podem ser extraídas:

\begin{Shaded}
\begin{Highlighting}[]
\NormalTok{lat  <-}\StringTok{ }\KeywordTok{ncvar_get}\NormalTok{(wrf, }\StringTok{"XLAT_M"}\NormalTok{)}
\NormalTok{lon  <-}\StringTok{ }\KeywordTok{ncvar_get}\NormalTok{(wrf, }\StringTok{"XLONG_M"}\NormalTok{)}
\NormalTok{time <-}\StringTok{ }\KeywordTok{ncvar_get}\NormalTok{(wrf, }\StringTok{"Times"}\NormalTok{)}
\end{Highlighting}
\end{Shaded}

O metadado de Longitude:

\begin{verbatim}
float XLONG_M[west_east,south_north,Time]   
FieldType: 104
MemoryOrder: XY 
units: degrees longitude
description: Longitude on mass grid
stagger: M
sr_x: 1
sr_y: 1
\end{verbatim}

Latitude:

\begin{verbatim}
float XLAT_M[west_east,south_north,Time]   
FieldType: 104
MemoryOrder: XY 
units: degrees latitude
description: Latitude on mass grid
stagger: M
sr_x: 1
sr_y: 1
\end{verbatim}

e a altura:

\begin{verbatim}
float GHT[west_east,south_north,num_metgrid_levels,Time]   
FieldType: 104
MemoryOrder: XYZ
units: m
description: Height
stagger: M
sr_x: 1
sr_y: 1
\end{verbatim}

Da mesma forma com que podemos acessar variáveis e atributos com
\texttt{ncvar\_get} e \texttt{ncatt\_get}, podemos modificar estes
valores com \texttt{ncvar\_put} e \texttt{ncatt\_put}. Outras operações
como renomear (\texttt{ncvar\_rename}) e trocar o valor de missval
(\texttt{ncvar\_change\_missval}) também estão disponíveis.

\emph{DICA}: \texttt{ncatt\_get} e \texttt{ncatt\_put} acessam e alteram
os atributos de váriaveis e também atributos globais do NetCDF usando o
argumento \texttt{varid=0}.

Para salvar as alterações e/ou liberar o acesso ao arquivo use a função
\texttt{nc\_close} (ou a função \texttt{nc\_sync} que sincroniza o
NetCDF mas não fecha a conexão com o arquivo).

\begin{Shaded}
\begin{Highlighting}[]
\KeywordTok{nc_close}\NormalTok{(wrf) }\CommentTok{# ou nc_sync(wrf)}
\end{Highlighting}
\end{Shaded}

Novas dimensões e também novas variáveis podem ser criadas com
\texttt{ncvar\_def} e \texttt{ncvar\_add} em um arquivo aberto com
permissão de leitura, como por exemplo:

\begin{Shaded}
\begin{Highlighting}[]
\NormalTok{wrf     <-}\StringTok{ }\KeywordTok{nc_open}\NormalTok{(}\StringTok{"~/met_em.d03.2016-01-10.nc"}\NormalTok{, }\DataTypeTok{write=}\OtherTok{TRUE}\NormalTok{)}
\NormalTok{extrema <-}\StringTok{ }\KeywordTok{ncvar_def}\NormalTok{(}\DataTypeTok{name =} \StringTok{"Tex"}\NormalTok{,}
                     \DataTypeTok{units =} \StringTok{"K"}\NormalTok{,}
                     \DataTypeTok{dim =} \KeywordTok{list}\NormalTok{(wrf}\OperatorTok{$}\NormalTok{dim}\OperatorTok{$}\NormalTok{west_east,}
\NormalTok{                                wrf}\OperatorTok{$}\NormalTok{dim}\OperatorTok{$}\NormalTok{south_north,}
\NormalTok{                                wrf}\OperatorTok{$}\NormalTok{dim}\OperatorTok{$}\NormalTok{Time),}
                     \DataTypeTok{missval =} \DecValTok{-999}\NormalTok{,}
                     \DataTypeTok{longname =} \StringTok{"temperatura extrema"}\NormalTok{)}
\KeywordTok{ncvar_add}\NormalTok{(wrf, extrema)}
\KeywordTok{names}\NormalTok{(wrf}\OperatorTok{$}\NormalTok{var)}
\KeywordTok{nc_close}\NormalTok{(wrf)}
\end{Highlighting}
\end{Shaded}

Se esse arquivo for aberto novamente vai conter 93 variáveis junto com a
variável \texttt{Tex} da forma que definimos, caso queria os mesmos
atributos que as demais é só usar a função \texttt{ncatt\_get} na
variável.

\begin{Shaded}
\begin{Highlighting}[]
\NormalTok{wrf     <-}\StringTok{ }\KeywordTok{nc_open}\NormalTok{(}\StringTok{"~/met_em.d03.2016-01-10.nc"}\NormalTok{,}\DataTypeTok{write=}\NormalTok{T)}
\KeywordTok{print}\NormalTok{(wrf)}
\end{Highlighting}
\end{Shaded}

\begin{verbatim}
## File ~/met_em.d03.2016-01-10.nc (NC_FORMAT_64BIT):
## 
##      92 variables (excluding dimension variables):
##         char Times[DateStrLen,Time]   
##         float PRES[west_east,south_north,num_metgrid_levels,Time]   
##             FieldType: 104
##             MemoryOrder: XYZ
##             units: 
##             description: 
##             stagger: M
##             sr_x: 1
##             sr_y: 1
##         float SOIL_LAYERS[west_east,south_north,num_st_layers,Time]   
##             FieldType: 104
##             MemoryOrder: XYZ
##             units: 
##             description: 
##             stagger: M
##             sr_x: 1
##             sr_y: 1
##         float SM[west_east,south_north,num_sm_layers,Time]   
##             FieldType: 104
##             MemoryOrder: XYZ
##             units: 
##             description: 
##             stagger: M
##             sr_x: 1
##             sr_y: 1
##         float ST[west_east,south_north,num_st_layers,Time]   
##             FieldType: 104
##             MemoryOrder: XYZ
##             units: 
##             description: 
##             stagger: M
##             sr_x: 1
##             sr_y: 1
##         float GHT[west_east,south_north,num_metgrid_levels,Time]   
##             FieldType: 104
##             MemoryOrder: XYZ
##             units: m
##             description: Height
##             stagger: M
##             sr_x: 1
##             sr_y: 1
##         float HGTTROP[west_east,south_north,Time]   
##             FieldType: 104
##             MemoryOrder: XY 
##             units: m
##             description: Height of tropopause
##             stagger: M
##             sr_x: 1
##             sr_y: 1
##         float TTROP[west_east,south_north,Time]   
##             FieldType: 104
##             MemoryOrder: XY 
##             units: K
##             description: Temperature at tropopause
##             stagger: M
##             sr_x: 1
##             sr_y: 1
##         float PTROPNN[west_east,south_north,Time]   
##             FieldType: 104
##             MemoryOrder: XY 
##             units: Pa
##             description: PTROP, used for nearest neighbor interp
##             stagger: M
##             sr_x: 1
##             sr_y: 1
##         float PTROP[west_east,south_north,Time]   
##             FieldType: 104
##             MemoryOrder: XY 
##             units: Pa
##             description: Pressure of tropopause
##             stagger: M
##             sr_x: 1
##             sr_y: 1
##         float VTROP[west_east,south_north_stag,Time]   
##             FieldType: 104
##             MemoryOrder: XY 
##             units: m s-1
##             description: V                 at tropopause
##             stagger: V
##             sr_x: 1
##             sr_y: 1
##         float UTROP[west_east_stag,south_north,Time]   
##             FieldType: 104
##             MemoryOrder: XY 
##             units: m s-1
##             description: U                 at tropopause
##             stagger: U
##             sr_x: 1
##             sr_y: 1
##         float HGTMAXW[west_east,south_north,Time]   
##             FieldType: 104
##             MemoryOrder: XY 
##             units: m
##             description: Height of max wind level
##             stagger: M
##             sr_x: 1
##             sr_y: 1
##         float TMAXW[west_east,south_north,Time]   
##             FieldType: 104
##             MemoryOrder: XY 
##             units: K
##             description: Temperature at max wind level
##             stagger: M
##             sr_x: 1
##             sr_y: 1
##         float PMAXWNN[west_east,south_north,Time]   
##             FieldType: 104
##             MemoryOrder: XY 
##             units: Pa
##             description: PMAXW, used for nearest neighbor interp
##             stagger: M
##             sr_x: 1
##             sr_y: 1
##         float PMAXW[west_east,south_north,Time]   
##             FieldType: 104
##             MemoryOrder: XY 
##             units: Pa
##             description: Pressure of max wind level
##             stagger: M
##             sr_x: 1
##             sr_y: 1
##         float VMAXW[west_east,south_north_stag,Time]   
##             FieldType: 104
##             MemoryOrder: XY 
##             units: m s-1
##             description: V                 at max wind
##             stagger: V
##             sr_x: 1
##             sr_y: 1
##         float UMAXW[west_east_stag,south_north,Time]   
##             FieldType: 104
##             MemoryOrder: XY 
##             units: m s-1
##             description: U                 at max wind
##             stagger: U
##             sr_x: 1
##             sr_y: 1
##         float SNOWH[west_east,south_north,Time]   
##             FieldType: 104
##             MemoryOrder: XY 
##             units: m
##             description: Physical Snow Depth
##             stagger: M
##             sr_x: 1
##             sr_y: 1
##         float SNOW[west_east,south_north,Time]   
##             FieldType: 104
##             MemoryOrder: XY 
##             units: kg m-2
##             description: Water equivalent snow depth
##             stagger: M
##             sr_x: 1
##             sr_y: 1
##         float SKINTEMP[west_east,south_north,Time]   
##             FieldType: 104
##             MemoryOrder: XY 
##             units: K
##             description: Skin temperature
##             stagger: M
##             sr_x: 1
##             sr_y: 1
##         float SOILHGT[west_east,south_north,Time]   
##             FieldType: 104
##             MemoryOrder: XY 
##             units: m
##             description: Terrain field of source analysis
##             stagger: M
##             sr_x: 1
##             sr_y: 1
##         float LANDSEA[west_east,south_north,Time]   
##             FieldType: 104
##             MemoryOrder: XY 
##             units: proprtn
##             description: Land/Sea flag (1=land, 0 or 2=sea)
##             stagger: M
##             sr_x: 1
##             sr_y: 1
##         float SEAICE[west_east,south_north,Time]   
##             FieldType: 104
##             MemoryOrder: XY 
##             units: proprtn
##             description: Ice flag
##             stagger: M
##             sr_x: 1
##             sr_y: 1
##         float ST100200[west_east,south_north,Time]   
##             FieldType: 104
##             MemoryOrder: XY 
##             units: K
##             description: T 100-200 cm below ground layer (Bottom)
##             stagger: M
##             sr_x: 1
##             sr_y: 1
##         float ST040100[west_east,south_north,Time]   
##             FieldType: 104
##             MemoryOrder: XY 
##             units: K
##             description: T 40-100 cm below ground layer (Upper)
##             stagger: M
##             sr_x: 1
##             sr_y: 1
##         float ST010040[west_east,south_north,Time]   
##             FieldType: 104
##             MemoryOrder: XY 
##             units: K
##             description: T 10-40 cm below ground layer (Upper)
##             stagger: M
##             sr_x: 1
##             sr_y: 1
##         float ST000010[west_east,south_north,Time]   
##             FieldType: 104
##             MemoryOrder: XY 
##             units: K
##             description: T 0-10 cm below ground layer (Upper)
##             stagger: M
##             sr_x: 1
##             sr_y: 1
##         float SM100200[west_east,south_north,Time]   
##             FieldType: 104
##             MemoryOrder: XY 
##             units: fraction
##             description: Soil Moist 100-200 cm below gr layer
##             stagger: M
##             sr_x: 1
##             sr_y: 1
##         float SM040100[west_east,south_north,Time]   
##             FieldType: 104
##             MemoryOrder: XY 
##             units: fraction
##             description: Soil Moist 40-100 cm below grn layer
##             stagger: M
##             sr_x: 1
##             sr_y: 1
##         float SM010040[west_east,south_north,Time]   
##             FieldType: 104
##             MemoryOrder: XY 
##             units: fraction
##             description: Soil Moist 10-40 cm below grn layer
##             stagger: M
##             sr_x: 1
##             sr_y: 1
##         float SM000010[west_east,south_north,Time]   
##             FieldType: 104
##             MemoryOrder: XY 
##             units: fraction
##             description: Soil Moist 0-10 cm below grn layer (Up)
##             stagger: M
##             sr_x: 1
##             sr_y: 1
##         float PSFC[west_east,south_north,Time]   
##             FieldType: 104
##             MemoryOrder: XY 
##             units: Pa
##             description: Surface Pressure
##             stagger: M
##             sr_x: 1
##             sr_y: 1
##         float RH[west_east,south_north,num_metgrid_levels,Time]   
##             FieldType: 104
##             MemoryOrder: XYZ
##             units: %
##             description: Relative Humidity
##             stagger: M
##             sr_x: 1
##             sr_y: 1
##         float VV[west_east,south_north_stag,num_metgrid_levels,Time]   
##             FieldType: 104
##             MemoryOrder: XYZ
##             units: m s-1
##             description: V
##             stagger: V
##             sr_x: 1
##             sr_y: 1
##         float UU[west_east_stag,south_north,num_metgrid_levels,Time]   
##             FieldType: 104
##             MemoryOrder: XYZ
##             units: m s-1
##             description: U
##             stagger: U
##             sr_x: 1
##             sr_y: 1
##         float TT[west_east,south_north,num_metgrid_levels,Time]   
##             FieldType: 104
##             MemoryOrder: XYZ
##             units: K
##             description: Temperature
##             stagger: M
##             sr_x: 1
##             sr_y: 1
##         float PMSL[west_east,south_north,Time]   
##             FieldType: 104
##             MemoryOrder: XY 
##             units: Pa
##             description: Sea-level Pressure
##             stagger: M
##             sr_x: 1
##             sr_y: 1
##         float URB_PARAM[west_east,south_north,z-dimension0132,Time]   
##             FieldType: 104
##             MemoryOrder: XYZ
##             units: dimensionless
##             description: Urban_Parameters
##             stagger: M
##             sr_x: 1
##             sr_y: 1
##         float LAKE_DEPTH[west_east,south_north,Time]   
##             FieldType: 104
##             MemoryOrder: XY 
##             units: meters MSL
##             description: Topography height
##             stagger: M
##             sr_x: 1
##             sr_y: 1
##         float VAR_SSO[west_east,south_north,Time]   
##             FieldType: 104
##             MemoryOrder: XY 
##             units: meters2 MSL
##             description: Variance of Subgrid Scale Orography
##             stagger: M
##             sr_x: 1
##             sr_y: 1
##         float OL4[west_east,south_north,Time]   
##             FieldType: 104
##             MemoryOrder: XY 
##             units: whoknows
##             description: something
##             stagger: M
##             sr_x: 1
##             sr_y: 1
##         float OL3[west_east,south_north,Time]   
##             FieldType: 104
##             MemoryOrder: XY 
##             units: whoknows
##             description: something
##             stagger: M
##             sr_x: 1
##             sr_y: 1
##         float OL2[west_east,south_north,Time]   
##             FieldType: 104
##             MemoryOrder: XY 
##             units: whoknows
##             description: something
##             stagger: M
##             sr_x: 1
##             sr_y: 1
##         float OL1[west_east,south_north,Time]   
##             FieldType: 104
##             MemoryOrder: XY 
##             units: whoknows
##             description: something
##             stagger: M
##             sr_x: 1
##             sr_y: 1
##         float OA4[west_east,south_north,Time]   
##             FieldType: 104
##             MemoryOrder: XY 
##             units: whoknows
##             description: something
##             stagger: M
##             sr_x: 1
##             sr_y: 1
##         float OA3[west_east,south_north,Time]   
##             FieldType: 104
##             MemoryOrder: XY 
##             units: whoknows
##             description: something
##             stagger: M
##             sr_x: 1
##             sr_y: 1
##         float OA2[west_east,south_north,Time]   
##             FieldType: 104
##             MemoryOrder: XY 
##             units: whoknows
##             description: something
##             stagger: M
##             sr_x: 1
##             sr_y: 1
##         float OA1[west_east,south_north,Time]   
##             FieldType: 104
##             MemoryOrder: XY 
##             units: whoknows
##             description: something
##             stagger: M
##             sr_x: 1
##             sr_y: 1
##         float VAR[west_east,south_north,Time]   
##             FieldType: 104
##             MemoryOrder: XY 
##             units: whoknows
##             description: something
##             stagger: M
##             sr_x: 1
##             sr_y: 1
##         float CON[west_east,south_north,Time]   
##             FieldType: 104
##             MemoryOrder: XY 
##             units: whoknows
##             description: something
##             stagger: M
##             sr_x: 1
##             sr_y: 1
##         float SLOPECAT[west_east,south_north,Time]   
##             FieldType: 104
##             MemoryOrder: XY 
##             units: category
##             description: Dominant category
##             stagger: M
##             sr_x: 1
##             sr_y: 1
##         float SNOALB[west_east,south_north,Time]   
##             FieldType: 104
##             MemoryOrder: XY 
##             units: percent
##             description: Maximum snow albedo
##             stagger: M
##             sr_x: 1
##             sr_y: 1
##         float LAI12M[west_east,south_north,z-dimension0012,Time]   
##             FieldType: 104
##             MemoryOrder: XYZ
##             units: m^2/m^2
##             description: MODIS LAI
##             stagger: M
##             sr_x: 1
##             sr_y: 1
##         float GREENFRAC[west_east,south_north,z-dimension0012,Time]   
##             FieldType: 104
##             MemoryOrder: XYZ
##             units: fraction
##             description: MODIS FPAR
##             stagger: M
##             sr_x: 1
##             sr_y: 1
##         float ALBEDO12M[west_east,south_north,z-dimension0012,Time]   
##             FieldType: 104
##             MemoryOrder: XYZ
##             units: percent
##             description: Monthly surface albedo
##             stagger: M
##             sr_x: 1
##             sr_y: 1
##         float SCB_DOM[west_east,south_north,Time]   
##             FieldType: 104
##             MemoryOrder: XY 
##             units: category
##             description: Dominant category
##             stagger: M
##             sr_x: 1
##             sr_y: 1
##         float SOILCBOT[west_east,south_north,z-dimension0016,Time]   
##             FieldType: 104
##             MemoryOrder: XYZ
##             units: category
##             description: 16-category bottom-layer soil type
##             stagger: M
##             sr_x: 1
##             sr_y: 1
##         float SCT_DOM[west_east,south_north,Time]   
##             FieldType: 104
##             MemoryOrder: XY 
##             units: category
##             description: Dominant category
##             stagger: M
##             sr_x: 1
##             sr_y: 1
##         float SOILCTOP[west_east,south_north,z-dimension0016,Time]   
##             FieldType: 104
##             MemoryOrder: XYZ
##             units: category
##             description: 16-category top-layer soil type
##             stagger: M
##             sr_x: 1
##             sr_y: 1
##         float SOILTEMP[west_east,south_north,Time]   
##             FieldType: 104
##             MemoryOrder: XY 
##             units: Kelvin
##             description: Annual mean deep soil temperature
##             stagger: M
##             sr_x: 1
##             sr_y: 1
##         float HGT_M[west_east,south_north,Time]   
##             FieldType: 104
##             MemoryOrder: XY 
##             units: meters MSL
##             description: GMTED2010 30-arc-second topography height
##             stagger: M
##             sr_x: 1
##             sr_y: 1
##         float LU_INDEX[west_east,south_north,Time]   
##             FieldType: 104
##             MemoryOrder: XY 
##             units: category
##             description: Dominant category
##             stagger: M
##             sr_x: 1
##             sr_y: 1
##         float LANDUSEF[west_east,south_north,z-dimension0024,Time]   
##             FieldType: 104
##             MemoryOrder: XYZ
##             units: category
##             description: 24-category USGS landuse
##             stagger: M
##             sr_x: 1
##             sr_y: 1
##         float COSALPHA_V[west_east,south_north_stag,Time]   
##             FieldType: 104
##             MemoryOrder: XY 
##             units: none
##             description: Cosine of rotation angle on V grid
##             stagger: V
##             sr_x: 1
##             sr_y: 1
##         float SINALPHA_V[west_east,south_north_stag,Time]   
##             FieldType: 104
##             MemoryOrder: XY 
##             units: none
##             description: Sine of rotation angle on V grid
##             stagger: V
##             sr_x: 1
##             sr_y: 1
##         float COSALPHA_U[west_east_stag,south_north,Time]   
##             FieldType: 104
##             MemoryOrder: XY 
##             units: none
##             description: Cosine of rotation angle on U grid
##             stagger: U
##             sr_x: 1
##             sr_y: 1
##         float SINALPHA_U[west_east_stag,south_north,Time]   
##             FieldType: 104
##             MemoryOrder: XY 
##             units: none
##             description: Sine of rotation angle on U grid
##             stagger: U
##             sr_x: 1
##             sr_y: 1
##         float XLONG_C[west_east_stag,south_north_stag,Time]   
##             FieldType: 104
##             MemoryOrder: XY 
##             units: degrees longitude
##             description: Longitude at grid cell corners
##             stagger: CORNER
##             sr_x: 1
##             sr_y: 1
##         float XLAT_C[west_east_stag,south_north_stag,Time]   
##             FieldType: 104
##             MemoryOrder: XY 
##             units: degrees latitude
##             description: Latitude at grid cell corners
##             stagger: CORNER
##             sr_x: 1
##             sr_y: 1
##         float LANDMASK[west_east,south_north,Time]   
##             FieldType: 104
##             MemoryOrder: XY 
##             units: none
##             description: Landmask : 1=land, 0=water
##             stagger: M
##             sr_x: 1
##             sr_y: 1
##         float COSALPHA[west_east,south_north,Time]   
##             FieldType: 104
##             MemoryOrder: XY 
##             units: none
##             description: Cosine of rotation angle
##             stagger: M
##             sr_x: 1
##             sr_y: 1
##         float SINALPHA[west_east,south_north,Time]   
##             FieldType: 104
##             MemoryOrder: XY 
##             units: none
##             description: Sine of rotation angle
##             stagger: M
##             sr_x: 1
##             sr_y: 1
##         float F[west_east,south_north,Time]   
##             FieldType: 104
##             MemoryOrder: XY 
##             units: -
##             description: Coriolis F parameter
##             stagger: M
##             sr_x: 1
##             sr_y: 1
##         float E[west_east,south_north,Time]   
##             FieldType: 104
##             MemoryOrder: XY 
##             units: -
##             description: Coriolis E parameter
##             stagger: M
##             sr_x: 1
##             sr_y: 1
##         float MAPFAC_UY[west_east_stag,south_north,Time]   
##             FieldType: 104
##             MemoryOrder: XY 
##             units: none
##             description: Mapfactor (y-dir) on U grid
##             stagger: U
##             sr_x: 1
##             sr_y: 1
##         float MAPFAC_VY[west_east,south_north_stag,Time]   
##             FieldType: 104
##             MemoryOrder: XY 
##             units: none
##             description: Mapfactor (y-dir) on V grid
##             stagger: V
##             sr_x: 1
##             sr_y: 1
##         float MAPFAC_MY[west_east,south_north,Time]   
##             FieldType: 104
##             MemoryOrder: XY 
##             units: none
##             description: Mapfactor (y-dir) on mass grid
##             stagger: M
##             sr_x: 1
##             sr_y: 1
##         float MAPFAC_UX[west_east_stag,south_north,Time]   
##             FieldType: 104
##             MemoryOrder: XY 
##             units: none
##             description: Mapfactor (x-dir) on U grid
##             stagger: U
##             sr_x: 1
##             sr_y: 1
##         float MAPFAC_VX[west_east,south_north_stag,Time]   
##             FieldType: 104
##             MemoryOrder: XY 
##             units: none
##             description: Mapfactor (x-dir) on V grid
##             stagger: V
##             sr_x: 1
##             sr_y: 1
##         float MAPFAC_MX[west_east,south_north,Time]   
##             FieldType: 104
##             MemoryOrder: XY 
##             units: none
##             description: Mapfactor (x-dir) on mass grid
##             stagger: M
##             sr_x: 1
##             sr_y: 1
##         float MAPFAC_U[west_east_stag,south_north,Time]   
##             FieldType: 104
##             MemoryOrder: XY 
##             units: none
##             description: Mapfactor on U grid
##             stagger: U
##             sr_x: 1
##             sr_y: 1
##         float MAPFAC_V[west_east,south_north_stag,Time]   
##             FieldType: 104
##             MemoryOrder: XY 
##             units: none
##             description: Mapfactor on V grid
##             stagger: V
##             sr_x: 1
##             sr_y: 1
##         float MAPFAC_M[west_east,south_north,Time]   
##             FieldType: 104
##             MemoryOrder: XY 
##             units: none
##             description: Mapfactor on mass grid
##             stagger: M
##             sr_x: 1
##             sr_y: 1
##         float CLONG[west_east,south_north,Time]   
##             FieldType: 104
##             MemoryOrder: XY 
##             units: degrees longitude
##             description: Computational longitude on mass grid
##             stagger: M
##             sr_x: 1
##             sr_y: 1
##         float CLAT[west_east,south_north,Time]   
##             FieldType: 104
##             MemoryOrder: XY 
##             units: degrees latitude
##             description: Computational latitude on mass grid
##             stagger: M
##             sr_x: 1
##             sr_y: 1
##         float XLONG_U[west_east_stag,south_north,Time]   
##             FieldType: 104
##             MemoryOrder: XY 
##             units: degrees longitude
##             description: Longitude on U grid
##             stagger: U
##             sr_x: 1
##             sr_y: 1
##         float XLAT_U[west_east_stag,south_north,Time]   
##             FieldType: 104
##             MemoryOrder: XY 
##             units: degrees latitude
##             description: Latitude on U grid
##             stagger: U
##             sr_x: 1
##             sr_y: 1
##         float XLONG_V[west_east,south_north_stag,Time]   
##             FieldType: 104
##             MemoryOrder: XY 
##             units: degrees longitude
##             description: Longitude on V grid
##             stagger: V
##             sr_x: 1
##             sr_y: 1
##         float XLAT_V[west_east,south_north_stag,Time]   
##             FieldType: 104
##             MemoryOrder: XY 
##             units: degrees latitude
##             description: Latitude on V grid
##             stagger: V
##             sr_x: 1
##             sr_y: 1
##         float XLONG_M[west_east,south_north,Time]   
##             FieldType: 104
##             MemoryOrder: XY 
##             units: degrees longitude
##             description: Longitude on mass grid
##             stagger: M
##             sr_x: 1
##             sr_y: 1
##         float XLAT_M[west_east,south_north,Time]   
##             FieldType: 104
##             MemoryOrder: XY 
##             units: degrees latitude
##             description: Latitude on mass grid
##             stagger: M
##             sr_x: 1
##             sr_y: 1
## 
##      13 dimensions:
##         Time  Size:1   *** is unlimited ***
##         DateStrLen  Size:19
##         west_east  Size:51
##         south_north  Size:51
##         num_metgrid_levels  Size:27
##         num_st_layers  Size:4
##         num_sm_layers  Size:4
##         south_north_stag  Size:52
##         west_east_stag  Size:52
##         z-dimension0132  Size:132
##         z-dimension0012  Size:12
##         z-dimension0016  Size:16
##         z-dimension0024  Size:24
## 
##     76 global attributes:
##         TITLE: OUTPUT FROM METGRID V3.9.1
##         SIMULATION_START_DATE: 2016-01-10_00:00:00
##         WEST-EAST_GRID_DIMENSION: 52
##         SOUTH-NORTH_GRID_DIMENSION: 52
##         BOTTOM-TOP_GRID_DIMENSION: 27
##         WEST-EAST_PATCH_START_UNSTAG: 1
##         WEST-EAST_PATCH_END_UNSTAG: 51
##         WEST-EAST_PATCH_START_STAG: 1
##         WEST-EAST_PATCH_END_STAG: 52
##         SOUTH-NORTH_PATCH_START_UNSTAG: 1
##         SOUTH-NORTH_PATCH_END_UNSTAG: 51
##         SOUTH-NORTH_PATCH_START_STAG: 1
##         SOUTH-NORTH_PATCH_END_STAG: 52
##         GRIDTYPE: C
##         DX: 1000
##         DY: 1000
##         DYN_OPT: 2
##         CEN_LAT: -23.5996932983398
##         CEN_LON: -46.6294555664062
##         TRUELAT1: -23
##         TRUELAT2: -24
##         MOAD_CEN_LAT: -23.6000061035156
##         STAND_LON: -45
##         POLE_LAT: 90
##         POLE_LON: 0
##         corner_lats: -23.8218078613281
##          corner_lats: -23.3720855712891
##          corner_lats: -23.3771743774414
##          corner_lats: -23.826904296875
##          corner_lats: -23.8217391967773
##          corner_lats: -23.3720245361328
##          corner_lats: -23.3772277832031
##          corner_lats: -23.8269424438477
##          corner_lats: -23.826286315918
##          corner_lats: -23.3675918579102
##          corner_lats: -23.372673034668
##          corner_lats: -23.8314056396484
##          corner_lats: -23.8262329101562
##          corner_lats: -23.3675231933594
##          corner_lats: -23.3727111816406
##          corner_lats: -23.8314437866211
##         corner_lons: -46.8780517578125
##          corner_lons: -46.8716430664062
##          corner_lons: -46.3817138671875
##          corner_lons: -46.3864440917969
##          corner_lons: -46.8829650878906
##          corner_lons: -46.8765258789062
##          corner_lons: -46.3768005371094
##          corner_lons: -46.3815307617188
##          corner_lons: -46.8781127929688
##          corner_lons: -46.87158203125
##          corner_lons: -46.3816528320312
##          corner_lons: -46.386474609375
##          corner_lons: -46.8830261230469
##          corner_lons: -46.87646484375
##          corner_lons: -46.3767700195312
##          corner_lons: -46.3815612792969
##         MAP_PROJ: 1
##         MMINLU: USGS
##         NUM_LAND_CAT: 24
##         ISWATER: 16
##         ISLAKE: -1
##         ISICE: 24
##         ISURBAN: 1
##         ISOILWATER: 14
##         grid_id: 3
##         parent_id: 2
##         i_parent_start: 35
##         j_parent_start: 33
##         i_parent_end: 51
##         j_parent_end: 49
##         parent_grid_ratio: 3
##         sr_x: 1
##         sr_y: 1
##         NUM_METGRID_SOIL_LEVELS: 4
##         FLAG_METGRID: 1
##         FLAG_EXCLUDED_MIDDLE: 0
##         FLAG_SOIL_LAYERS: 1
##         FLAG_SNOW: 1
##         FLAG_PSFC: 1
##         FLAG_SM000010: 1
##         FLAG_SM010040: 1
##         FLAG_SM040100: 1
##         FLAG_SM100200: 1
##         FLAG_ST000010: 1
##         FLAG_ST010040: 1
##         FLAG_ST040100: 1
##         FLAG_ST100200: 1
##         FLAG_SLP: 1
##         FLAG_SNOWH: 1
##         FLAG_SOILHGT: 1
##         FLAG_UTROP: 1
##         FLAG_VTROP: 1
##         FLAG_TTROP: 1
##         FLAG_PTROP: 1
##         FLAG_PTROPNN: 1
##         FLAG_HGTTROP: 1
##         FLAG_UMAXW: 1
##         FLAG_VMAXW: 1
##         FLAG_TMAXW: 1
##         FLAG_PMAXW: 1
##         FLAG_PMAXWNN: 1
##         FLAG_HGTMAXW: 1
##         FLAG_MF_XY: 1
##         FLAG_LAI12M: 1
##         FLAG_LAKE_DEPTH: 1
\end{verbatim}

O pacote possue ainda funções mais específicas para a criação de
arquivos em NetCDF como \texttt{nc\_create}, funções que definem
dimenções como \texttt{ncdim\_def} e funções para colocar e tirar o
arquivo de modo de definição \texttt{nc\_redef} e \texttt{nc\_enddef}.

\emph{DICA}: o NetCDF no R funciona de forma parecida com ouma lista ou
data frame, podemos ``ver'' ou selecionar suas sub-partes
(sub-sub-partes\ldots{}) com ``\$'' e TAB.

\chapter{Plotando}\label{plotando}

\section{plot}\label{plot}

exemplo

\begin{Shaded}
\begin{Highlighting}[]
\NormalTok{df <-}\StringTok{ }\KeywordTok{readRDS}\NormalTok{(}\StringTok{"df.rds"}\NormalTok{)}
\KeywordTok{head}\NormalTok{(df)}
\end{Highlighting}
\end{Shaded}

\begin{verbatim}
##   TipodeRede TipodeMonitoramento            Tipo       Data  Hora
## 2 Automático              CETESB Dados Primários 01/01/2014 01:00
## 3 Automático              CETESB Dados Primários 01/01/2014 02:00
## 4 Automático              CETESB Dados Primários 01/01/2014 03:00
## 5 Automático              CETESB Dados Primários 01/01/2014 04:00
## 6 Automático              CETESB Dados Primários 01/01/2014 05:00
## 7 Automático              CETESB Dados Primários 01/01/2014 06:00
##   CodigoEstação                NomeEstação              NomeParâmetro
## 2            95 Cid.Universitária-USP-Ipen NOx (Óxidos de Nitrogênio)
## 3            95 Cid.Universitária-USP-Ipen NOx (Óxidos de Nitrogênio)
## 4            95 Cid.Universitária-USP-Ipen NOx (Óxidos de Nitrogênio)
## 5            95 Cid.Universitária-USP-Ipen NOx (Óxidos de Nitrogênio)
## 6            95 Cid.Universitária-USP-Ipen NOx (Óxidos de Nitrogênio)
## 7            95 Cid.Universitária-USP-Ipen NOx (Óxidos de Nitrogênio)
##   UnidadedeMedida MediaHoraria MediaMovel Valido       tempo_char
## 2             ppb            9          -    Não 01/01/2014 01:00
## 3             ppb            9          -    Sim 01/01/2014 02:00
## 4             ppb            5          -    Sim 01/01/2014 03:00
## 5             ppb            4          -    Sim 01/01/2014 04:00
## 6             ppb            5          -    Sim 01/01/2014 05:00
## 7             ppb            5          -    Sim 01/01/2014 06:00
##                 tempo  weekdays     mes    diajuliano  ano
## 2 2014-01-01 01:00:00 Wednesday January 16071.04 days 2014
## 3 2014-01-01 02:00:00 Wednesday January 16071.08 days 2014
## 4 2014-01-01 03:00:00 Wednesday January 16071.12 days 2014
## 5 2014-01-01 04:00:00 Wednesday January 16071.17 days 2014
## 6 2014-01-01 05:00:00 Wednesday January 16071.21 days 2014
## 7 2014-01-01 06:00:00 Wednesday January 16071.25 days 2014
\end{verbatim}

plot basico

\begin{Shaded}
\begin{Highlighting}[]
\KeywordTok{args}\NormalTok{(plot)}
\end{Highlighting}
\end{Shaded}

\begin{verbatim}
## function (x, y, ...) 
## NULL
\end{verbatim}

então

\begin{Shaded}
\begin{Highlighting}[]
\KeywordTok{plot}\NormalTok{(}\DataTypeTok{x =}\NormalTok{ df}\OperatorTok{$}\NormalTok{tempo, }\DataTypeTok{y =}\NormalTok{ df}\OperatorTok{$}\NormalTok{MediaHoraria)}
\end{Highlighting}
\end{Shaded}

\includegraphics{cursoR_files/figure-latex/unnamed-chunk-68-1.pdf}

feio, ne?

\begin{Shaded}
\begin{Highlighting}[]
\KeywordTok{plot}\NormalTok{(}\DataTypeTok{x =}\NormalTok{ df}\OperatorTok{$}\NormalTok{tempo[}\DecValTok{1}\OperatorTok{:}\DecValTok{100}\NormalTok{], }\DataTypeTok{y =}\NormalTok{ df}\OperatorTok{$}\NormalTok{MediaHoraria[}\DecValTok{1}\OperatorTok{:}\DecValTok{100}\NormalTok{], }
     \DataTypeTok{pch =} \DecValTok{16}\NormalTok{, }\DataTypeTok{type =} \StringTok{"b"}\NormalTok{, }\DataTypeTok{col =} \StringTok{"blue"}\NormalTok{,}
     \DataTypeTok{xlab =} \StringTok{"horas"}\NormalTok{, }\DataTypeTok{ylab =} \StringTok{"NO2[ppb]"}\NormalTok{,}
     \DataTypeTok{main =} \StringTok{"Plot menos feio"}\NormalTok{)}
\end{Highlighting}
\end{Shaded}

\includegraphics{cursoR_files/figure-latex/unnamed-chunk-69-1.pdf}

Vamos a colocar \textbf{DOIS} plots

\begin{Shaded}
\begin{Highlighting}[]
\KeywordTok{plot}\NormalTok{(}\DataTypeTok{x =}\NormalTok{ df}\OperatorTok{$}\NormalTok{tempo[}\DecValTok{1}\OperatorTok{:}\DecValTok{100}\NormalTok{], }\DataTypeTok{y =}\NormalTok{ df}\OperatorTok{$}\NormalTok{MediaHoraria[}\DecValTok{101}\OperatorTok{:}\DecValTok{200}\NormalTok{], }
     \DataTypeTok{pch =} \DecValTok{16}\NormalTok{, }\DataTypeTok{type =} \StringTok{"b"}\NormalTok{, }\DataTypeTok{col =} \StringTok{"blue"}\NormalTok{,}
     \DataTypeTok{xlab =} \StringTok{"horas"}\NormalTok{, }\DataTypeTok{ylab =} \StringTok{"NO2[ppb]"}\NormalTok{,}
     \DataTypeTok{main =} \StringTok{"Plot estranho"}\NormalTok{)}
\KeywordTok{lines}\NormalTok{(}\DataTypeTok{x =}\NormalTok{ df}\OperatorTok{$}\NormalTok{tempo[}\DecValTok{1}\OperatorTok{:}\DecValTok{100}\NormalTok{], }\DataTypeTok{y =}\NormalTok{ df}\OperatorTok{$}\NormalTok{MediaHoraria[}\DecValTok{1}\OperatorTok{:}\DecValTok{100}\NormalTok{], }
     \DataTypeTok{pch =} \DecValTok{15}\NormalTok{, }\DataTypeTok{type =} \StringTok{"b"}\NormalTok{, }\DataTypeTok{col =} \StringTok{"red"}\NormalTok{)}
\end{Highlighting}
\end{Shaded}

\includegraphics{cursoR_files/figure-latex/unnamed-chunk-70-1.pdf}

Se tu é fan de BASE PLOT, tudo bem :)

\section{ggplot}\label{ggplot}

Tem um monte de recursos para ggplot na web

\begin{Shaded}
\begin{Highlighting}[]
\KeywordTok{library}\NormalTok{(ggplot2)}
\KeywordTok{ggplot}\NormalTok{(df)}
\end{Highlighting}
\end{Shaded}

\includegraphics{cursoR_files/figure-latex/unnamed-chunk-71-1.pdf}

\begin{Shaded}
\begin{Highlighting}[]
\KeywordTok{ggplot}\NormalTok{(df, }\KeywordTok{aes}\NormalTok{(}\DataTypeTok{x =}\NormalTok{ tempo, }\DataTypeTok{y =}\NormalTok{ MediaHoraria))}
\end{Highlighting}
\end{Shaded}

\includegraphics{cursoR_files/figure-latex/unnamed-chunk-72-1.pdf}

\begin{Shaded}
\begin{Highlighting}[]
\KeywordTok{ggplot}\NormalTok{(df, }\KeywordTok{aes}\NormalTok{(}\DataTypeTok{x =}\NormalTok{ tempo, }\DataTypeTok{y =}\NormalTok{ MediaHoraria)) }\OperatorTok{+}\StringTok{ }
\StringTok{  }\KeywordTok{geom_line}\NormalTok{()}
\end{Highlighting}
\end{Shaded}

\begin{verbatim}
## Warning: Removed 1 rows containing missing values (geom_path).
\end{verbatim}

\includegraphics{cursoR_files/figure-latex/unnamed-chunk-73-1.pdf}

opa

\begin{Shaded}
\begin{Highlighting}[]
\KeywordTok{ggplot}\NormalTok{(df, }\KeywordTok{aes}\NormalTok{(}\DataTypeTok{x =}\NormalTok{ tempo, }\DataTypeTok{y =}\NormalTok{ MediaHoraria, }\DataTypeTok{colour =}\NormalTok{ mes)) }\OperatorTok{+}\StringTok{ }
\StringTok{  }\KeywordTok{geom_line}\NormalTok{() }
\end{Highlighting}
\end{Shaded}

\begin{verbatim}
## Warning: Removed 1 rows containing missing values (geom_path).
\end{verbatim}

\includegraphics{cursoR_files/figure-latex/unnamed-chunk-74-1.pdf}

deixando so 2014

\begin{Shaded}
\begin{Highlighting}[]
\NormalTok{df <-}\StringTok{ }\NormalTok{df[df}\OperatorTok{$}\NormalTok{ano }\OperatorTok{==}\StringTok{ }\DecValTok{2014}\NormalTok{,]}
\end{Highlighting}
\end{Shaded}

\begin{Shaded}
\begin{Highlighting}[]
\KeywordTok{ggplot}\NormalTok{(df, }\KeywordTok{aes}\NormalTok{(}\DataTypeTok{x =}\NormalTok{ tempo, }\DataTypeTok{y =}\NormalTok{ MediaHoraria, }\DataTypeTok{colour =}\NormalTok{ mes)) }\OperatorTok{+}\StringTok{ }
\StringTok{  }\KeywordTok{geom_line}\NormalTok{() }\OperatorTok{+}
\StringTok{  }\KeywordTok{theme_bw}\NormalTok{()}
\end{Highlighting}
\end{Shaded}

\includegraphics{cursoR_files/figure-latex/unnamed-chunk-76-1.pdf}

\begin{Shaded}
\begin{Highlighting}[]
\KeywordTok{ggplot}\NormalTok{(df, }\KeywordTok{aes}\NormalTok{(}\DataTypeTok{x =}\NormalTok{ tempo, }\DataTypeTok{y =}\NormalTok{ MediaHoraria, }\DataTypeTok{colour =}\NormalTok{ MediaHoraria)) }\OperatorTok{+}\StringTok{ }
\StringTok{  }\KeywordTok{geom_line}\NormalTok{() }
\end{Highlighting}
\end{Shaded}

\includegraphics{cursoR_files/figure-latex/unnamed-chunk-77-1.pdf}

\begin{Shaded}
\begin{Highlighting}[]
\KeywordTok{ggplot}\NormalTok{(df, }\KeywordTok{aes}\NormalTok{(}\DataTypeTok{x =}\NormalTok{ tempo, }\DataTypeTok{y =}\NormalTok{ MediaHoraria, }\DataTypeTok{colour =}\NormalTok{ MediaHoraria)) }\OperatorTok{+}\StringTok{ }
\StringTok{  }\KeywordTok{geom_line}\NormalTok{() }\OperatorTok{+}
\StringTok{  }\KeywordTok{facet_wrap}\NormalTok{(}\OperatorTok{~}\NormalTok{mes)}
\end{Highlighting}
\end{Shaded}

\includegraphics{cursoR_files/figure-latex/unnamed-chunk-78-1.pdf}

\begin{Shaded}
\begin{Highlighting}[]
\KeywordTok{ggplot}\NormalTok{(df, }\KeywordTok{aes}\NormalTok{(}\DataTypeTok{x =}\NormalTok{ tempo, }\DataTypeTok{y =}\NormalTok{ MediaHoraria, }\DataTypeTok{colour =}\NormalTok{ MediaHoraria)) }\OperatorTok{+}\StringTok{ }
\StringTok{  }\KeywordTok{geom_line}\NormalTok{() }\OperatorTok{+}
\StringTok{  }\KeywordTok{facet_wrap}\NormalTok{(}\OperatorTok{~}\NormalTok{mes, }\DataTypeTok{scales =} \StringTok{"free"}\NormalTok{)}
\end{Highlighting}
\end{Shaded}

\includegraphics{cursoR_files/figure-latex/unnamed-chunk-79-1.pdf}

Y para terminar, meu theme favorito

\begin{Shaded}
\begin{Highlighting}[]
\NormalTok{devtools}\OperatorTok{::}\KeywordTok{install_github}\NormalTok{(}\StringTok{"atmoschem/veinreport"}\NormalTok{)}
\end{Highlighting}
\end{Shaded}

e logo

\begin{Shaded}
\begin{Highlighting}[]
\KeywordTok{library}\NormalTok{(veinreport)}
\KeywordTok{library}\NormalTok{(cptcity)}
\end{Highlighting}
\end{Shaded}

\begin{Shaded}
\begin{Highlighting}[]
\KeywordTok{ggplot}\NormalTok{(df, }\KeywordTok{aes}\NormalTok{(}\DataTypeTok{x =}\NormalTok{ tempo, }\DataTypeTok{y =}\NormalTok{ MediaHoraria, }\DataTypeTok{colour =}\NormalTok{ MediaHoraria)) }\OperatorTok{+}\StringTok{ }
\StringTok{  }\KeywordTok{geom_line}\NormalTok{()}\OperatorTok{+}
\StringTok{  }\KeywordTok{theme_black}\NormalTok{() }\OperatorTok{+}
\StringTok{  }\KeywordTok{scale_color_gradientn}\NormalTok{(}\DataTypeTok{colours =} \KeywordTok{cpt}\NormalTok{())}
\end{Highlighting}
\end{Shaded}

\includegraphics{cursoR_files/figure-latex/unnamed-chunk-82-1.pdf}

Pode revertir a escala de cores

\begin{Shaded}
\begin{Highlighting}[]
\KeywordTok{ggplot}\NormalTok{(df, }\KeywordTok{aes}\NormalTok{(}\DataTypeTok{x =}\NormalTok{ tempo, }\DataTypeTok{y =}\NormalTok{ MediaHoraria, }\DataTypeTok{colour =}\NormalTok{ MediaHoraria)) }\OperatorTok{+}\StringTok{ }
\StringTok{  }\KeywordTok{geom_line}\NormalTok{()}\OperatorTok{+}
\StringTok{  }\KeywordTok{theme_black}\NormalTok{() }\OperatorTok{+}
\StringTok{  }\KeywordTok{scale_color_gradientn}\NormalTok{(}\DataTypeTok{colours =} \KeywordTok{rev}\NormalTok{(}\KeywordTok{cpt}\NormalTok{())) }\OperatorTok{+}\StringTok{ }
\StringTok{  }\KeywordTok{facet_wrap}\NormalTok{(}\OperatorTok{~}\NormalTok{mes, }\DataTypeTok{scales =} \StringTok{"free"}\NormalTok{)}
\end{Highlighting}
\end{Shaded}

\includegraphics{cursoR_files/figure-latex/unnamed-chunk-83-1.pdf}

não gostou, tenta com a sorte

\begin{Shaded}
\begin{Highlighting}[]
\KeywordTok{ggplot}\NormalTok{(df, }\KeywordTok{aes}\NormalTok{(}\DataTypeTok{x =}\NormalTok{ tempo, }\DataTypeTok{y =}\NormalTok{ MediaHoraria, }\DataTypeTok{colour =}\NormalTok{ MediaHoraria)) }\OperatorTok{+}\StringTok{ }
\StringTok{  }\KeywordTok{geom_line}\NormalTok{()}\OperatorTok{+}
\StringTok{  }\KeywordTok{theme_black}\NormalTok{() }\OperatorTok{+}
\StringTok{    }\KeywordTok{facet_wrap}\NormalTok{(}\OperatorTok{~}\NormalTok{mes, }\DataTypeTok{scales =} \StringTok{"free"}\NormalTok{) }\OperatorTok{+}
\StringTok{  }\KeywordTok{scale_color_gradientn}\NormalTok{(}\DataTypeTok{colours =} \KeywordTok{lucky}\NormalTok{())}
\end{Highlighting}
\end{Shaded}

\begin{verbatim}
## Colour gradient: idv_radar_dbz, number: 3449
\end{verbatim}

\includegraphics{cursoR_files/figure-latex/unnamed-chunk-84-1.pdf}

\begin{Shaded}
\begin{Highlighting}[]
\KeywordTok{ggplot}\NormalTok{(df, }\KeywordTok{aes}\NormalTok{(}\DataTypeTok{x =}\NormalTok{ tempo, }\DataTypeTok{y =}\NormalTok{ MediaHoraria, }\DataTypeTok{colour =}\NormalTok{ MediaHoraria)) }\OperatorTok{+}\StringTok{ }
\StringTok{  }\KeywordTok{geom_line}\NormalTok{()}\OperatorTok{+}
\StringTok{  }\KeywordTok{theme_black}\NormalTok{() }\OperatorTok{+}
\StringTok{    }\KeywordTok{facet_wrap}\NormalTok{(}\OperatorTok{~}\NormalTok{mes, }\DataTypeTok{scales =} \StringTok{"free"}\NormalTok{) }\OperatorTok{+}
\StringTok{  }\KeywordTok{scale_color_gradientn}\NormalTok{(}\DataTypeTok{colours =} \KeywordTok{lucky}\NormalTok{())}
\end{Highlighting}
\end{Shaded}

\begin{verbatim}
## Colour gradient: ing_xmas_ib_jul13, number: 3608
\end{verbatim}

\includegraphics{cursoR_files/figure-latex/unnamed-chunk-85-1.pdf}

\begin{Shaded}
\begin{Highlighting}[]
\KeywordTok{ggplot}\NormalTok{(df, }\KeywordTok{aes}\NormalTok{(}\DataTypeTok{x =}\NormalTok{ tempo, }\DataTypeTok{y =}\NormalTok{ MediaHoraria, }\DataTypeTok{colour =}\NormalTok{ MediaHoraria)) }\OperatorTok{+}\StringTok{ }
\StringTok{  }\KeywordTok{geom_line}\NormalTok{()}\OperatorTok{+}
\StringTok{  }\KeywordTok{theme_black}\NormalTok{() }\OperatorTok{+}
\StringTok{  }\KeywordTok{scale_color_gradientn}\NormalTok{(}\DataTypeTok{colours =} \KeywordTok{lucky}\NormalTok{())}
\end{Highlighting}
\end{Shaded}

\begin{verbatim}
## Colour gradient: go2_webtwo_green_g2_1, number: 3257
\end{verbatim}

\includegraphics{cursoR_files/figure-latex/unnamed-chunk-86-1.pdf}

\chapter{Estruturas de control}\label{loop}

\section{if-else}\label{if-else}

\section{for}\label{for}

\section{while}\label{while}

\section{repeat}\label{repeat}

\section{lapply}\label{lapply}

\section{sapply}\label{sapply}

\section{split}\label{split}

\section{tapply}\label{tapply}

\section{apply}\label{apply}

\section{mapply}\label{mapply}

\chapter{De scripts a funções e de funções a pacotes}\label{fx}

Coming soon

\chapter{\texorpdfstring{Geo Spatial: \texttt{raster}, \texttt{sf} e
\texttt{stars}}{Geo Spatial: raster, sf e stars}}\label{geo}

Coming soon

\bibliography{book.bib,packages.bib}


\end{document}
