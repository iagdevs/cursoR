\documentclass[]{book}
\usepackage{lmodern}
\usepackage{amssymb,amsmath}
\usepackage{ifxetex,ifluatex}
\usepackage{fixltx2e} % provides \textsubscript
\ifnum 0\ifxetex 1\fi\ifluatex 1\fi=0 % if pdftex
  \usepackage[T1]{fontenc}
  \usepackage[utf8]{inputenc}
\else % if luatex or xelatex
  \ifxetex
    \usepackage{mathspec}
  \else
    \usepackage{fontspec}
  \fi
  \defaultfontfeatures{Ligatures=TeX,Scale=MatchLowercase}
\fi
% use upquote if available, for straight quotes in verbatim environments
\IfFileExists{upquote.sty}{\usepackage{upquote}}{}
% use microtype if available
\IfFileExists{microtype.sty}{%
\usepackage{microtype}
\UseMicrotypeSet[protrusion]{basicmath} % disable protrusion for tt fonts
}{}
\usepackage[margin=1in]{geometry}
\usepackage{hyperref}
\hypersetup{unicode=true,
            pdftitle={Curso de R para meteorologia IAG/USP},
            pdfauthor={Sergio Ibarra-Espinosa, Amanda Rehbein e possivelmente outros (u r invited to collaborate)},
            pdfborder={0 0 0},
            breaklinks=true}
\urlstyle{same}  % don't use monospace font for urls
\usepackage{natbib}
\bibliographystyle{apalike}
\usepackage{color}
\usepackage{fancyvrb}
\newcommand{\VerbBar}{|}
\newcommand{\VERB}{\Verb[commandchars=\\\{\}]}
\DefineVerbatimEnvironment{Highlighting}{Verbatim}{commandchars=\\\{\}}
% Add ',fontsize=\small' for more characters per line
\usepackage{framed}
\definecolor{shadecolor}{RGB}{248,248,248}
\newenvironment{Shaded}{\begin{snugshade}}{\end{snugshade}}
\newcommand{\KeywordTok}[1]{\textcolor[rgb]{0.13,0.29,0.53}{\textbf{#1}}}
\newcommand{\DataTypeTok}[1]{\textcolor[rgb]{0.13,0.29,0.53}{#1}}
\newcommand{\DecValTok}[1]{\textcolor[rgb]{0.00,0.00,0.81}{#1}}
\newcommand{\BaseNTok}[1]{\textcolor[rgb]{0.00,0.00,0.81}{#1}}
\newcommand{\FloatTok}[1]{\textcolor[rgb]{0.00,0.00,0.81}{#1}}
\newcommand{\ConstantTok}[1]{\textcolor[rgb]{0.00,0.00,0.00}{#1}}
\newcommand{\CharTok}[1]{\textcolor[rgb]{0.31,0.60,0.02}{#1}}
\newcommand{\SpecialCharTok}[1]{\textcolor[rgb]{0.00,0.00,0.00}{#1}}
\newcommand{\StringTok}[1]{\textcolor[rgb]{0.31,0.60,0.02}{#1}}
\newcommand{\VerbatimStringTok}[1]{\textcolor[rgb]{0.31,0.60,0.02}{#1}}
\newcommand{\SpecialStringTok}[1]{\textcolor[rgb]{0.31,0.60,0.02}{#1}}
\newcommand{\ImportTok}[1]{#1}
\newcommand{\CommentTok}[1]{\textcolor[rgb]{0.56,0.35,0.01}{\textit{#1}}}
\newcommand{\DocumentationTok}[1]{\textcolor[rgb]{0.56,0.35,0.01}{\textbf{\textit{#1}}}}
\newcommand{\AnnotationTok}[1]{\textcolor[rgb]{0.56,0.35,0.01}{\textbf{\textit{#1}}}}
\newcommand{\CommentVarTok}[1]{\textcolor[rgb]{0.56,0.35,0.01}{\textbf{\textit{#1}}}}
\newcommand{\OtherTok}[1]{\textcolor[rgb]{0.56,0.35,0.01}{#1}}
\newcommand{\FunctionTok}[1]{\textcolor[rgb]{0.00,0.00,0.00}{#1}}
\newcommand{\VariableTok}[1]{\textcolor[rgb]{0.00,0.00,0.00}{#1}}
\newcommand{\ControlFlowTok}[1]{\textcolor[rgb]{0.13,0.29,0.53}{\textbf{#1}}}
\newcommand{\OperatorTok}[1]{\textcolor[rgb]{0.81,0.36,0.00}{\textbf{#1}}}
\newcommand{\BuiltInTok}[1]{#1}
\newcommand{\ExtensionTok}[1]{#1}
\newcommand{\PreprocessorTok}[1]{\textcolor[rgb]{0.56,0.35,0.01}{\textit{#1}}}
\newcommand{\AttributeTok}[1]{\textcolor[rgb]{0.77,0.63,0.00}{#1}}
\newcommand{\RegionMarkerTok}[1]{#1}
\newcommand{\InformationTok}[1]{\textcolor[rgb]{0.56,0.35,0.01}{\textbf{\textit{#1}}}}
\newcommand{\WarningTok}[1]{\textcolor[rgb]{0.56,0.35,0.01}{\textbf{\textit{#1}}}}
\newcommand{\AlertTok}[1]{\textcolor[rgb]{0.94,0.16,0.16}{#1}}
\newcommand{\ErrorTok}[1]{\textcolor[rgb]{0.64,0.00,0.00}{\textbf{#1}}}
\newcommand{\NormalTok}[1]{#1}
\usepackage{longtable,booktabs}
\usepackage{graphicx,grffile}
\makeatletter
\def\maxwidth{\ifdim\Gin@nat@width>\linewidth\linewidth\else\Gin@nat@width\fi}
\def\maxheight{\ifdim\Gin@nat@height>\textheight\textheight\else\Gin@nat@height\fi}
\makeatother
% Scale images if necessary, so that they will not overflow the page
% margins by default, and it is still possible to overwrite the defaults
% using explicit options in \includegraphics[width, height, ...]{}
\setkeys{Gin}{width=\maxwidth,height=\maxheight,keepaspectratio}
\IfFileExists{parskip.sty}{%
\usepackage{parskip}
}{% else
\setlength{\parindent}{0pt}
\setlength{\parskip}{6pt plus 2pt minus 1pt}
}
\setlength{\emergencystretch}{3em}  % prevent overfull lines
\providecommand{\tightlist}{%
  \setlength{\itemsep}{0pt}\setlength{\parskip}{0pt}}
\setcounter{secnumdepth}{5}
% Redefines (sub)paragraphs to behave more like sections
\ifx\paragraph\undefined\else
\let\oldparagraph\paragraph
\renewcommand{\paragraph}[1]{\oldparagraph{#1}\mbox{}}
\fi
\ifx\subparagraph\undefined\else
\let\oldsubparagraph\subparagraph
\renewcommand{\subparagraph}[1]{\oldsubparagraph{#1}\mbox{}}
\fi

%%% Use protect on footnotes to avoid problems with footnotes in titles
\let\rmarkdownfootnote\footnote%
\def\footnote{\protect\rmarkdownfootnote}

%%% Change title format to be more compact
\usepackage{titling}

% Create subtitle command for use in maketitle
\newcommand{\subtitle}[1]{
  \posttitle{
    \begin{center}\large#1\end{center}
    }
}

\setlength{\droptitle}{-2em}
  \title{Curso de R para meteorologia IAG/USP}
  \pretitle{\vspace{\droptitle}\centering\huge}
  \posttitle{\par}
  \author{Sergio Ibarra-Espinosa, Amanda Rehbein e possivelmente outros (u r
invited to collaborate)}
  \preauthor{\centering\large\emph}
  \postauthor{\par}
  \predate{\centering\large\emph}
  \postdate{\par}
  \date{2018-04-29}

\usepackage{booktabs}
\usepackage{amsthm}
\makeatletter
\def\thm@space@setup{%
  \thm@preskip=8pt plus 2pt minus 4pt
  \thm@postskip=\thm@preskip
}
\makeatother

\usepackage{amsthm}
\newtheorem{theorem}{Theorem}[chapter]
\newtheorem{lemma}{Lemma}[chapter]
\theoremstyle{definition}
\newtheorem{definition}{Definition}[chapter]
\newtheorem{corollary}{Corollary}[chapter]
\newtheorem{proposition}{Proposition}[chapter]
\theoremstyle{definition}
\newtheorem{example}{Example}[chapter]
\theoremstyle{definition}
\newtheorem{exercise}{Exercise}[chapter]
\theoremstyle{remark}
\newtheorem*{remark}{Remark}
\newtheorem*{solution}{Solution}
\begin{document}
\maketitle

{
\setcounter{tocdepth}{1}
\tableofcontents
}
\chapter{Pre-requisitos do sistema}\label{primero}

Em Windows, instale além do R, Rtools
\url{https://cran.r-project.org/bin/windows/Rtools/}

Em MAC instale netcdf e:

\begin{Shaded}
\begin{Highlighting}[]
\ExtensionTok{brew}\NormalTok{ unlink gdal}
\ExtensionTok{brew}\NormalTok{ tap osgeo/osgeo4mac }\KeywordTok{&&} \ExtensionTok{brew}\NormalTok{ tap --repair}
\ExtensionTok{brew}\NormalTok{ install proj}
\ExtensionTok{brew}\NormalTok{ install geos}
\ExtensionTok{brew}\NormalTok{ install udunits}
\ExtensionTok{brew}\NormalTok{ install gdal2 --with-armadillo --with-complete --with-libkml --with-unsupported}
\ExtensionTok{brew}\NormalTok{ link --force gdal2}
\end{Highlighting}
\end{Shaded}

Em Ubuntu:

\begin{Shaded}
\begin{Highlighting}[]
  \ExtensionTok{-}\NormalTok{ sudo add-apt-repository ppa:ubuntugis/ubuntugis-unstable --yes}
  \ExtensionTok{-}\NormalTok{ sudo apt-get --yes --force-yes update -qq}
  \CommentTok{# install tmap dependencies}
  \ExtensionTok{-}\NormalTok{ sudo apt-get install --yes libprotobuf-dev protobuf-compiler libv8-3.14-dev}
  \CommentTok{# install tmap dependencies; for 16.04 libjq-dev this ppa is needed:}
  \ExtensionTok{-}\NormalTok{ sudo add-apt-repository -y ppa:opencpu/jq}
  \ExtensionTok{-}\NormalTok{ sudo apt-get --yes --force-yes update -qq}
  \ExtensionTok{-}\NormalTok{ sudo apt-get install libjq-dev}
  \CommentTok{# units/udunits2 dependency:}
  \ExtensionTok{-}\NormalTok{ sudo apt-get install --yes libudunits2-dev}
  \CommentTok{# sf dependencies:}
  \ExtensionTok{-}\NormalTok{ sudo apt-get install --yes libproj-dev libgeos-dev libgdal-dev libnetcdf-dev  netcdf-bin gdal-bin}
\end{Highlighting}
\end{Shaded}

\section{Pacotes usados neste curso}\label{pacotes-usados-neste-curso}

Para fazer este curso instale os seguintes pacotes como indicado:

\begin{Shaded}
\begin{Highlighting}[]
\KeywordTok{install.packages}\NormalTok{(}\StringTok{"devtools"}\NormalTok{)}
\NormalTok{devtools}\OperatorTok{::}\KeywordTok{install_github}\NormalTok{(}\StringTok{"tidyverse/tidyverse"}\NormalTok{)}
\NormalTok{devtools}\OperatorTok{::}\KeywordTok{install_github}\NormalTok{(}\StringTok{"r-spatial/sf"}\NormalTok{)}
\NormalTok{devtools}\OperatorTok{::}\KeywordTok{install_github}\NormalTok{(}\StringTok{"r-spatial/mapview"}\NormalTok{)}
\NormalTok{devtools}\OperatorTok{::}\KeywordTok{install_github}\NormalTok{(}\StringTok{"r-spatial/stars"}\NormalTok{)}
\KeywordTok{install.packages}\NormalTok{(}\KeywordTok{c}\NormalTok{(}\StringTok{"raster"}\NormalTok{, }\StringTok{"sp"}\NormalTok{, }\StringTok{"rgdal"}\NormalTok{, }\StringTok{"maptools"}\NormalTok{, }\StringTok{"ncdf4"}\NormalTok{))}
\KeywordTok{install.packages}\NormalTok{(}\KeywordTok{c}\NormalTok{(}\StringTok{"cptcity"}\NormalTok{, }\StringTok{"data.table"}\NormalTok{, }\StringTok{"openair"}\NormalTok{))}
\end{Highlighting}
\end{Shaded}

\begin{itemize}
\tightlist
\item
  \href{https://CRAN.R-project.org/package=devtools}{devtools} é um
  pacote para instalar pacotes de diferentes repositórios
\item
  \href{https://github.com/tidyverse}{tidyverse} é o universo de pacotes
  do Hadley Wickham. A instalação tem que ser usando devtools, pois
  precisamos plotar os objetos espacias sf usando
  \href{https://www.isgeomsfinggplot2yet.site/}{geom\_sf}.
\item
  \href{https://github.com/r-spatial/sf}{sf} e
  \href{https://github.com/r-spatial/mapbiew}{mapview},
  \href{https://github.com/r-spatial/stars}{stars}, raster, sp, rgdal e
  maptools são para a parte espacial. Lembrar que os objetos em
  meteorologias são espaço-temporais.
\item
  ncdf4 é um pacote para manipular arquivos NetCDF.
\item
  \href{https://ibarraespinosa.github.io/cptcity/}{cptcity} é um pacote
  que tem 7140 paletas de cores do arquivo web cpt-city
  (\url{http://soliton.vm.bytemark.co.uk/pub/cpt-city/index.html}).
\item
  \href{http://davidcarslaw.github.io/openair/}{openair} é um pacote
  para trabalhar com dados de qualidade do ar e meteorologia.
\end{itemize}

Se faltarem dependencias de sistema, instale elas e instale os pacotes.

\section{Colaborar}\label{colaborar}

A forma preferida de colaboração é com pull-requests em
\url{https://github.com/ibarraespinosa/cursoR/pull/new/master}. Lembre
de aplicar a Guia de Estilo de R de Google
(\url{https://google.github.io/styleguide/Rguide.xml}) ou com o formato
de formatR \url{https://yihui.name/formatr/}. Em poucas palavras, lembre
que seu código vai ser lido por seres humanos. Se quiser tem acesso no
repositório deste curso, me contate. Tem um botão para editar qualquer
página.

\section{Aportar com dados}\label{aportar-com-dados}

Se você tem dados para fazer este curso mais legal, por favor, edite
este aquivo e com pull request, eu vou fazer um merge para poder.

\begin{enumerate}
\def\labelenumi{\arabic{enumi}.}
\item
  NCEP: \url{ftp://nomads.ncdc.noaa.gov/GFS/analysis_only/}
\item
\item
\end{enumerate}

\chapter{Intro}\label{intro}

Este curso é para pos, então vamos ver conteúdo rapidamente e se não da
tempo, este curso esta online no sitio
\url{https://github.com/atmoschem/cursorIAG}.

Eu tento usar
\href{http://stat.ethz.ch/R-manual/R-devel/library/base/html/00Index.html}{BASE}
sempre que posso, e se não da ai vou para outros paradigmas.

Outros pacotes de BASE:
\href{http://stat.ethz.ch/R-manual/R-devel/library/utils/html/00Index.html}{utils},
\href{http://stat.ethz.ch/R-manual/R-devel/library/stats/html/00Index.html}{stats},
\href{http://stat.ethz.ch/R-manual/R-devel/library/datasets/html/00Index.html}{datasets},
\href{http://stat.ethz.ch/R-manual/R-devel/library/graphics/html/00Index.html}{graphics},
\href{https://stat.ethz.ch/R-manual/R-devel/library/grDevices/html/00Index.html}{grDevices},
\href{https://stat.ethz.ch/R-manual/R-devel/library/grid/html/00Index.html}{grid},
\href{https://stat.ethz.ch/R-manual/R-devel/library/methods/html/00Index.html}{methods},
\href{https://stat.ethz.ch/R-manual/R-devel/library/tools/html/00Index.html}{tools},
\href{https://stat.ethz.ch/R-manual/R-devel/library/parallel/html/00Index.html}{parallel},
\href{https://stat.ethz.ch/R-manual/R-devel/library/compiler/html/00Index.html}{compiler},
\href{https://stat.ethz.ch/R-manual/R-devel/library/splines/html/00Index.html}{splines},
\href{https://stat.ethz.ch/R-manual/R-devel/library/tcltk/html/00Index.html}{tcltk}
,
\href{https://stat.ethz.ch/R-manual/R-devel/library/stats4/html/00Index.html}{stats4}.

Veja
\href{https://cran.r-project.org/web/packages/available_packages_by_name.html}{outros}
pacotes.

Este curso esta baseado no livro
\href{https://leanpub.com/rprogramming}{R Programming for Data Science}.

Vamos usar \href{https://www.rstudio.com/}{Rstudio}

\textbf{Dica:}

\begin{itemize}
\tightlist
\item
  Se não sabe como usar uma função, escreva: \texttt{?função}.
\item
  As funções tem argumentos, use \textbf{TAB} para ver eles numa função.
\end{itemize}

Vamos lá!

\chapter{R!}\label{r}

\begin{itemize}
\tightlist
\item
  Quase em qualquer sistema operacional mas eu vou focar em Linux.
\item
  Muita documentação:
\item
  \href{http://cran.r-project.org/doc/manuals/r-release/R-intro.html}{Intro}.
\item
  \href{http://cran.r-project.org/doc/manuals/r-release/R-data.html}{I/O}.
\item
  Quer fazer um pacote?
  \href{http://cran.r-project.org/doc/manuals/r-release/R-exts.html}{Veja},
  \href{http://cran.r-project.org/doc/manuals/r-release/R-ints.html}{aqui}
  e
  \href{http://cran.r-project.org/doc/manuals/r-release/R-lang.html}{aqui}.
\item
  \href{https://stackoverflow.com/questions/tagged/r}{Stackoverflow}
  provides a great source of resources.
\end{itemize}

\section{Objetos de R}\label{objetos-de-r}

\begin{itemize}
\tightlist
\item
  Character a
\item
  numeric 1
\item
  integer 1
\item
  complex 0+1i
\item
  logical TRUE
\end{itemize}

\section{Classe}\label{classe}

\texttt{class} função permite ver a classe dos objetos

\section{Vetores}\label{vetores}

\begin{itemize}
\tightlist
\item
  c(``A'', ``C'', ``D'')
\item
  1:5 = c(1, 2, 3, 4, 5)
\item
  c(TRUE, FALSE)
\item
  c(1i, -1i)
\item
  c(1, ``C'', ``D'') qual é a classe???
\item
  c(1, NA, ``D'') qual é a classe???
\item
  c(1, NA, NaN) qual é a classe???
\end{itemize}

\section{\texorpdfstring{Convertir objetos com
\texttt{as}}{Convertir objetos com as}}\label{convertir-objetos-com-as}

\begin{Shaded}
\begin{Highlighting}[]
\KeywordTok{as.numeric}\NormalTok{(}\KeywordTok{c}\NormalTok{(}\DecValTok{1}\NormalTok{, }\StringTok{"C"}\NormalTok{, }\StringTok{"D"}\NormalTok{))}
\end{Highlighting}
\end{Shaded}

\begin{verbatim}
## Warning: NAs introduzidos por coerção
\end{verbatim}

\begin{verbatim}
## [1]  1 NA NA
\end{verbatim}

\section{\texorpdfstring{Matrices e a função
\texttt{matrix}}{Matrices e a função matrix}}\label{matrices-e-a-funcao-matrix}

\textbf{{[}linhas, colunas{]}}

\begin{itemize}
\tightlist
\item
  permitidos elementos \textbf{da mesma clase}!
\end{itemize}

vamos ver os argumentos da função \texttt{matrix}

\begin{Shaded}
\begin{Highlighting}[]
\KeywordTok{args}\NormalTok{(matrix)}
\end{Highlighting}
\end{Shaded}

\begin{verbatim}
## function (data = NA, nrow = 1, ncol = 1, byrow = FALSE, dimnames = NULL) 
## NULL
\end{verbatim}

usando TAB

\begin{Shaded}
\begin{Highlighting}[]
\NormalTok{(m <-}\StringTok{ }\KeywordTok{matrix}\NormalTok{(}\DataTypeTok{data =} \DecValTok{0}\NormalTok{, }\DataTypeTok{nrow =} \DecValTok{4}\NormalTok{, }\DataTypeTok{ncol =} \DecValTok{4}\NormalTok{))}
\end{Highlighting}
\end{Shaded}

\begin{verbatim}
##      [,1] [,2] [,3] [,4]
## [1,]    0    0    0    0
## [2,]    0    0    0    0
## [3,]    0    0    0    0
## [4,]    0    0    0    0
\end{verbatim}

\begin{Shaded}
\begin{Highlighting}[]
\NormalTok{(m1 <-}\StringTok{ }\KeywordTok{matrix}\NormalTok{(}\DataTypeTok{data =} \DecValTok{1}\OperatorTok{:}\NormalTok{(}\DecValTok{4}\OperatorTok{*}\DecValTok{4}\NormalTok{), }\DataTypeTok{nrow =} \DecValTok{4}\NormalTok{, }\DataTypeTok{ncol =} \DecValTok{4}\NormalTok{))}
\end{Highlighting}
\end{Shaded}

\begin{verbatim}
##      [,1] [,2] [,3] [,4]
## [1,]    1    5    9   13
## [2,]    2    6   10   14
## [3,]    3    7   11   15
## [4,]    4    8   12   16
\end{verbatim}

\begin{Shaded}
\begin{Highlighting}[]
\KeywordTok{dim}\NormalTok{(m1)}
\end{Highlighting}
\end{Shaded}

\begin{verbatim}
## [1] 4 4
\end{verbatim}

\begin{Shaded}
\begin{Highlighting}[]
\NormalTok{(m2 <-}\StringTok{ }\KeywordTok{matrix}\NormalTok{(}\DataTypeTok{data =} \DecValTok{1}\OperatorTok{:}\NormalTok{(}\DecValTok{4}\OperatorTok{*}\DecValTok{4}\NormalTok{), }\DataTypeTok{nrow =} \DecValTok{4}\NormalTok{, }\DataTypeTok{ncol =} \DecValTok{4}\NormalTok{, }\DataTypeTok{byrow =} \OtherTok{TRUE}\NormalTok{))}
\end{Highlighting}
\end{Shaded}

\begin{verbatim}
##      [,1] [,2] [,3] [,4]
## [1,]    1    2    3    4
## [2,]    5    6    7    8
## [3,]    9   10   11   12
## [4,]   13   14   15   16
\end{verbatim}

\section{Array}\label{array}

É como uma matriz de matrizes de matrizes de matrizes\ldots{}\ldots{}
and so on.

\begin{Shaded}
\begin{Highlighting}[]
\KeywordTok{args}\NormalTok{(array)}
\end{Highlighting}
\end{Shaded}

\begin{verbatim}
## function (data = NA, dim = length(data), dimnames = NULL) 
## NULL
\end{verbatim}

lembre usar TAB

\begin{Shaded}
\begin{Highlighting}[]
\NormalTok{(a <-}\StringTok{ }\KeywordTok{array}\NormalTok{(}\DataTypeTok{data =} \DecValTok{0}\NormalTok{, }\DataTypeTok{dim =} \KeywordTok{c}\NormalTok{(}\DecValTok{1}\NormalTok{,}\DecValTok{1}\NormalTok{)))}
\end{Highlighting}
\end{Shaded}

\begin{verbatim}
##      [,1]
## [1,]    0
\end{verbatim}

\begin{Shaded}
\begin{Highlighting}[]
\KeywordTok{class}\NormalTok{(a)}
\end{Highlighting}
\end{Shaded}

\begin{verbatim}
## [1] "matrix"
\end{verbatim}

\begin{Shaded}
\begin{Highlighting}[]
\NormalTok{(a <-}\StringTok{ }\KeywordTok{array}\NormalTok{(}\DataTypeTok{data =} \DecValTok{0}\NormalTok{, }\DataTypeTok{dim =} \KeywordTok{c}\NormalTok{(}\DecValTok{1}\NormalTok{,}\DecValTok{1}\NormalTok{,}\DecValTok{1}\NormalTok{)))}
\end{Highlighting}
\end{Shaded}

\begin{verbatim}
## , , 1
## 
##      [,1]
## [1,]    0
\end{verbatim}

\begin{Shaded}
\begin{Highlighting}[]
\KeywordTok{class}\NormalTok{(a)}
\end{Highlighting}
\end{Shaded}

\begin{verbatim}
## [1] "array"
\end{verbatim}

\begin{Shaded}
\begin{Highlighting}[]
\NormalTok{(a <-}\StringTok{ }\KeywordTok{array}\NormalTok{(}\DataTypeTok{data =} \DecValTok{0}\NormalTok{, }\DataTypeTok{dim =} \KeywordTok{c}\NormalTok{(}\DecValTok{2}\NormalTok{,}\DecValTok{2}\NormalTok{,}\DecValTok{2}\NormalTok{)))}
\end{Highlighting}
\end{Shaded}

\begin{verbatim}
## , , 1
## 
##      [,1] [,2]
## [1,]    0    0
## [2,]    0    0
## 
## , , 2
## 
##      [,1] [,2]
## [1,]    0    0
## [2,]    0    0
\end{verbatim}

\begin{Shaded}
\begin{Highlighting}[]
\NormalTok{(a <-}\StringTok{ }\KeywordTok{array}\NormalTok{(}\DataTypeTok{data =} \DecValTok{0}\NormalTok{, }\DataTypeTok{dim =} \KeywordTok{c}\NormalTok{(}\DecValTok{2}\NormalTok{,}\DecValTok{4}\NormalTok{,}\DecValTok{4}\NormalTok{)))}
\end{Highlighting}
\end{Shaded}

\begin{verbatim}
## , , 1
## 
##      [,1] [,2] [,3] [,4]
## [1,]    0    0    0    0
## [2,]    0    0    0    0
## 
## , , 2
## 
##      [,1] [,2] [,3] [,4]
## [1,]    0    0    0    0
## [2,]    0    0    0    0
## 
## , , 3
## 
##      [,1] [,2] [,3] [,4]
## [1,]    0    0    0    0
## [2,]    0    0    0    0
## 
## , , 4
## 
##      [,1] [,2] [,3] [,4]
## [1,]    0    0    0    0
## [2,]    0    0    0    0
\end{verbatim}

\begin{Shaded}
\begin{Highlighting}[]
\KeywordTok{dim}\NormalTok{(a)}
\end{Highlighting}
\end{Shaded}

\begin{verbatim}
## [1] 2 4 4
\end{verbatim}

\begin{Shaded}
\begin{Highlighting}[]
\NormalTok{(a <-}\StringTok{ }\KeywordTok{array}\NormalTok{(}\DataTypeTok{data =} \DecValTok{0}\NormalTok{, }\DataTypeTok{dim =} \KeywordTok{c}\NormalTok{(}\DecValTok{2}\NormalTok{, }\DecValTok{2}\NormalTok{,}\DecValTok{2}\NormalTok{,}\DecValTok{2}\NormalTok{)))}
\end{Highlighting}
\end{Shaded}

\section{\texorpdfstring{\texttt{list}}{list}}\label{list}

As listas são como sacolas, e dentro delas, tu pode colocar mais
sacolas\ldots{} então, tu pode ter sacolas, dentro de sacolas, dentro de
sacolas\ldots{} ou

\begin{Shaded}
\begin{Highlighting}[]
\KeywordTok{list}\NormalTok{(}\KeywordTok{list}\NormalTok{(}\KeywordTok{list}\NormalTok{(}\KeywordTok{list}\NormalTok{(}\DecValTok{1}\NormalTok{))))}
\end{Highlighting}
\end{Shaded}

\begin{verbatim}
## [[1]]
## [[1]][[1]]
## [[1]][[1]][[1]]
## [[1]][[1]][[1]][[1]]
## [1] 1
\end{verbatim}

a diferença das matrices, tu pode colocar cualquer coisa nas listas, por
exemplo: funções, characters, etc.

\begin{Shaded}
\begin{Highlighting}[]
\NormalTok{(x <-}\StringTok{ }\KeywordTok{list}\NormalTok{(}\DecValTok{1}\NormalTok{, }\StringTok{"a"}\NormalTok{, }\OtherTok{TRUE}\NormalTok{, }\DecValTok{1} \OperatorTok{+}\StringTok{ }\NormalTok{4i))}
\end{Highlighting}
\end{Shaded}

\begin{verbatim}
## [[1]]
## [1] 1
## 
## [[2]]
## [1] "a"
## 
## [[3]]
## [1] TRUE
## 
## [[4]]
## [1] 1+4i
\end{verbatim}

\section{Tempo e Data}\label{tempo-e-data}

R tem classes de tempo e data:

\begin{Shaded}
\begin{Highlighting}[]
\NormalTok{(a <-}\StringTok{ }\KeywordTok{ISOdate}\NormalTok{(}\DataTypeTok{year =} \DecValTok{2018}\NormalTok{, }\DataTypeTok{month =} \DecValTok{4}\NormalTok{, }\DataTypeTok{day =} \DecValTok{5}\NormalTok{))}
\end{Highlighting}
\end{Shaded}

\begin{verbatim}
## [1] "2018-04-05 12:00:00 GMT"
\end{verbatim}

\begin{Shaded}
\begin{Highlighting}[]
\KeywordTok{class}\NormalTok{(a)}
\end{Highlighting}
\end{Shaded}

\begin{verbatim}
## [1] "POSIXct" "POSIXt"
\end{verbatim}

\begin{Shaded}
\begin{Highlighting}[]
\NormalTok{(b <-}\StringTok{ }\KeywordTok{ISOdate}\NormalTok{(}\DataTypeTok{year =} \DecValTok{2018}\NormalTok{, }\DataTypeTok{month =} \DecValTok{4}\NormalTok{, }\DataTypeTok{day =} \DecValTok{5}\NormalTok{, }\DataTypeTok{tz =} \StringTok{"Americas/Sao_Paulo"}\NormalTok{))}
\end{Highlighting}
\end{Shaded}

\begin{verbatim}
## [1] "2018-04-05 12:00:00 Americas"
\end{verbatim}

tempo

\begin{Shaded}
\begin{Highlighting}[]
\NormalTok{(d <-}\StringTok{ }\KeywordTok{ISOdatetime}\NormalTok{(}\DataTypeTok{year =} \DecValTok{2018}\NormalTok{, }\DataTypeTok{month =} \DecValTok{4}\NormalTok{, }\DataTypeTok{day =} \DecValTok{5}\NormalTok{, }\DataTypeTok{hour =} \DecValTok{0}\NormalTok{, }\DataTypeTok{min =} \DecValTok{0}\NormalTok{, }\DataTypeTok{sec =} \DecValTok{0}\NormalTok{,}
                  \DataTypeTok{tz =} \StringTok{"Americas/Sao_Paulo"}\NormalTok{))}
\end{Highlighting}
\end{Shaded}

\begin{verbatim}
## [1] "2018-04-05 Americas"
\end{verbatim}

O pacote \href{https://github.com/eddelbuettel/nanotime}{nanotime}
permite trabalhar com nano segundos.

Da pra fazer secuencias:

\begin{Shaded}
\begin{Highlighting}[]
\NormalTok{hoje <-}\StringTok{ }\KeywordTok{Sys.time}\NormalTok{()}
\NormalTok{(a <-}\StringTok{ }\KeywordTok{seq.POSIXt}\NormalTok{(}\DataTypeTok{from =}\NormalTok{ hoje, }\DataTypeTok{by =} \DecValTok{3600}\NormalTok{, }\DataTypeTok{length.out =} \DecValTok{24}\NormalTok{))}
\end{Highlighting}
\end{Shaded}

\begin{verbatim}
##  [1] "2018-04-29 18:46:07 -03" "2018-04-29 19:46:07 -03"
##  [3] "2018-04-29 20:46:07 -03" "2018-04-29 21:46:07 -03"
##  [5] "2018-04-29 22:46:07 -03" "2018-04-29 23:46:07 -03"
##  [7] "2018-04-30 00:46:07 -03" "2018-04-30 01:46:07 -03"
##  [9] "2018-04-30 02:46:07 -03" "2018-04-30 03:46:07 -03"
## [11] "2018-04-30 04:46:07 -03" "2018-04-30 05:46:07 -03"
## [13] "2018-04-30 06:46:07 -03" "2018-04-30 07:46:07 -03"
## [15] "2018-04-30 08:46:07 -03" "2018-04-30 09:46:07 -03"
## [17] "2018-04-30 10:46:07 -03" "2018-04-30 11:46:07 -03"
## [19] "2018-04-30 12:46:07 -03" "2018-04-30 13:46:07 -03"
## [21] "2018-04-30 14:46:07 -03" "2018-04-30 15:46:07 -03"
## [23] "2018-04-30 16:46:07 -03" "2018-04-30 17:46:07 -03"
\end{verbatim}

funções bacana: \textbf{weekdays}, \textbf{month}, \textbf{julian}

\begin{Shaded}
\begin{Highlighting}[]
\KeywordTok{weekdays}\NormalTok{(a)}
\end{Highlighting}
\end{Shaded}

\begin{verbatim}
##  [1] "domingo" "domingo" "domingo" "domingo" "domingo" "domingo" "segunda"
##  [8] "segunda" "segunda" "segunda" "segunda" "segunda" "segunda" "segunda"
## [15] "segunda" "segunda" "segunda" "segunda" "segunda" "segunda" "segunda"
## [22] "segunda" "segunda" "segunda"
\end{verbatim}

\begin{Shaded}
\begin{Highlighting}[]
\KeywordTok{months}\NormalTok{(a)}
\end{Highlighting}
\end{Shaded}

\begin{verbatim}
##  [1] "abril" "abril" "abril" "abril" "abril" "abril" "abril" "abril"
##  [9] "abril" "abril" "abril" "abril" "abril" "abril" "abril" "abril"
## [17] "abril" "abril" "abril" "abril" "abril" "abril" "abril" "abril"
\end{verbatim}

\begin{Shaded}
\begin{Highlighting}[]
\KeywordTok{julian}\NormalTok{(a) }\CommentTok{#olha ?julian... dias desde origin}
\end{Highlighting}
\end{Shaded}

\begin{verbatim}
## Time differences in days
##  [1] 17650.91 17650.95 17650.99 17651.03 17651.07 17651.12 17651.16
##  [8] 17651.20 17651.24 17651.28 17651.32 17651.37 17651.41 17651.45
## [15] 17651.49 17651.53 17651.57 17651.62 17651.66 17651.70 17651.74
## [22] 17651.78 17651.82 17651.87
## attr(,"origin")
## [1] "1970-01-01 GMT"
\end{verbatim}

olha \url{https://en.wikipedia.org/wiki/Julian_day}:

\section{Fatores}\label{fatores}

Os \texttt{factors} podem ser um pouco infernais. Olha
\href{http://www.burns-stat.com/documents/books/the-r-inferno/}{R
INFERNO}

Usados para representar categorias, ejemplo clasico para nos, dias da
semana.

\begin{Shaded}
\begin{Highlighting}[]
\NormalTok{a <-}\StringTok{ }\KeywordTok{seq.POSIXt}\NormalTok{(}\DataTypeTok{from =}\NormalTok{ hoje, }\DataTypeTok{by =} \DecValTok{3600}\NormalTok{, }\DataTypeTok{length.out =} \DecValTok{24}\OperatorTok{*}\DecValTok{7}\NormalTok{)}
\NormalTok{aa <-}\StringTok{ }\KeywordTok{weekdays}\NormalTok{(a)}
\KeywordTok{class}\NormalTok{(aa)}
\end{Highlighting}
\end{Shaded}

\begin{verbatim}
## [1] "character"
\end{verbatim}

\begin{Shaded}
\begin{Highlighting}[]
\KeywordTok{factor}\NormalTok{(aa)}
\end{Highlighting}
\end{Shaded}

\begin{verbatim}
##   [1] domingo domingo domingo domingo domingo domingo segunda segunda
##   [9] segunda segunda segunda segunda segunda segunda segunda segunda
##  [17] segunda segunda segunda segunda segunda segunda segunda segunda
##  [25] segunda segunda segunda segunda segunda segunda terça   terça  
##  [33] terça   terça   terça   terça   terça   terça   terça   terça  
##  [41] terça   terça   terça   terça   terça   terça   terça   terça  
##  [49] terça   terça   terça   terça   terça   terça   quarta  quarta 
##  [57] quarta  quarta  quarta  quarta  quarta  quarta  quarta  quarta 
##  [65] quarta  quarta  quarta  quarta  quarta  quarta  quarta  quarta 
##  [73] quarta  quarta  quarta  quarta  quarta  quarta  quinta  quinta 
##  [81] quinta  quinta  quinta  quinta  quinta  quinta  quinta  quinta 
##  [89] quinta  quinta  quinta  quinta  quinta  quinta  quinta  quinta 
##  [97] quinta  quinta  quinta  quinta  quinta  quinta  sexta   sexta  
## [105] sexta   sexta   sexta   sexta   sexta   sexta   sexta   sexta  
## [113] sexta   sexta   sexta   sexta   sexta   sexta   sexta   sexta  
## [121] sexta   sexta   sexta   sexta   sexta   sexta   sábado  sábado 
## [129] sábado  sábado  sábado  sábado  sábado  sábado  sábado  sábado 
## [137] sábado  sábado  sábado  sábado  sábado  sábado  sábado  sábado 
## [145] sábado  sábado  sábado  sábado  sábado  sábado  domingo domingo
## [153] domingo domingo domingo domingo domingo domingo domingo domingo
## [161] domingo domingo domingo domingo domingo domingo domingo domingo
## Levels: domingo quarta quinta sábado segunda sexta terça
\end{verbatim}

olha os \textbf{Levels}

Então:

\begin{Shaded}
\begin{Highlighting}[]
\NormalTok{ab <-}\StringTok{ }\KeywordTok{factor}\NormalTok{(}\DataTypeTok{x =}\NormalTok{ aa,}
             \DataTypeTok{levels =} \KeywordTok{c}\NormalTok{(}\StringTok{"Monday"}\NormalTok{, }\StringTok{"Tuesday"}\NormalTok{,  }\StringTok{"Wednesday"}\NormalTok{,  }\StringTok{"Thursday"}\NormalTok{,}
                        \StringTok{"Friday"}\NormalTok{, }\StringTok{"Saturday"}\NormalTok{, }\StringTok{"Sunday"}\NormalTok{))}
\KeywordTok{levels}\NormalTok{(ab)}
\end{Highlighting}
\end{Shaded}

\begin{verbatim}
## [1] "Monday"    "Tuesday"   "Wednesday" "Thursday"  "Friday"    "Saturday" 
## [7] "Sunday"
\end{verbatim}

\section{\texorpdfstring{\texttt{data.frames}}{data.frames}}\label{data.frames}

\emph{lembre ?data.frame}

São como planilha EXCEL\ldots{}. mais o menos

É uma classe bem especial, tem elementos de matriz mas o modo é lista

\begin{Shaded}
\begin{Highlighting}[]
\NormalTok{(df <-}\StringTok{ }\KeywordTok{data.frame}\NormalTok{(}\DataTypeTok{a =} \DecValTok{1}\OperatorTok{:}\DecValTok{3}\NormalTok{))}
\end{Highlighting}
\end{Shaded}

\begin{verbatim}
##   a
## 1 1
## 2 2
## 3 3
\end{verbatim}

\begin{Shaded}
\begin{Highlighting}[]
\KeywordTok{names}\NormalTok{(df)}
\end{Highlighting}
\end{Shaded}

\begin{verbatim}
## [1] "a"
\end{verbatim}

\begin{Shaded}
\begin{Highlighting}[]
\KeywordTok{class}\NormalTok{(df)}
\end{Highlighting}
\end{Shaded}

\begin{verbatim}
## [1] "data.frame"
\end{verbatim}

\begin{Shaded}
\begin{Highlighting}[]
\KeywordTok{mode}\NormalTok{(df)}
\end{Highlighting}
\end{Shaded}

\begin{verbatim}
## [1] "list"
\end{verbatim}

Então

\begin{Shaded}
\begin{Highlighting}[]
\KeywordTok{nrow}\NormalTok{(df)}
\end{Highlighting}
\end{Shaded}

\begin{verbatim}
## [1] 3
\end{verbatim}

\begin{Shaded}
\begin{Highlighting}[]
\KeywordTok{ncol}\NormalTok{(df)}
\end{Highlighting}
\end{Shaded}

\begin{verbatim}
## [1] 1
\end{verbatim}

\begin{Shaded}
\begin{Highlighting}[]
\KeywordTok{dim}\NormalTok{(df)}
\end{Highlighting}
\end{Shaded}

\begin{verbatim}
## [1] 3 1
\end{verbatim}

\chapter{Importando e exportando dados em
R}\label{importando-e-exportando-dados-em-r}

\section{data-frames}\label{data-frames}

Probabelmente um dos promeiros objetos que vamos usar quando começamos
usar R. Pensa num data-frame como uma planilha de Libreoffice (o excel).
Os data-frame pode ser criaos como foi visto na seção anterior. O
principal, é que temos varias funções para ler data-frames no R, entre
elas

\begin{itemize}
\tightlist
\item
  read.csv
\item
  read.csv2
\item
  read.table
\end{itemize}

Agora vamos a ler dados do repositorio usando read.table, mas primeiro
vamos lembrar que se tu precisar ver a ajuda da função, tem que escrever
no R \texttt{?read.table}. Então, agora vamos ver os argumentos da
função:

\begin{Shaded}
\begin{Highlighting}[]
\KeywordTok{args}\NormalTok{(read.table)}
\end{Highlighting}
\end{Shaded}

\begin{verbatim}
## function (file, header = FALSE, sep = "", quote = "\"'", dec = ".", 
##     numerals = c("allow.loss", "warn.loss", "no.loss"), row.names, 
##     col.names, as.is = !stringsAsFactors, na.strings = "NA", 
##     colClasses = NA, nrows = -1, skip = 0, check.names = TRUE, 
##     fill = !blank.lines.skip, strip.white = FALSE, blank.lines.skip = TRUE, 
##     comment.char = "#", allowEscapes = FALSE, flush = FALSE, 
##     stringsAsFactors = default.stringsAsFactors(), fileEncoding = "", 
##     encoding = "unknown", text, skipNul = FALSE) 
## NULL
\end{verbatim}

Aqui vem-se os valores default dos argumentos da função
\texttt{read.table}. O terceiro argumento é sep, com valores por default
= ``''.

\begin{Shaded}
\begin{Highlighting}[]
\NormalTok{df <-}\StringTok{ }\KeywordTok{read.table}\NormalTok{(}\StringTok{"https://raw.githubusercontent.com/ibarraespinosa/cursoR/master/dados/NOXIPEN2014.txt"}\NormalTok{)}
\end{Highlighting}
\end{Shaded}

Agora vamos usar a funções \texttt{head} and \texttt{tail} para ver as
primeiras e as ultimas 6 linhas do data-frame.

\begin{Shaded}
\begin{Highlighting}[]
\KeywordTok{head}\NormalTok{(df)}
\end{Highlighting}
\end{Shaded}

\begin{verbatim}
##   TipodeRede TipodeMonitoramento            Tipo       Data  Hora
## 2 Automático              CETESB Dados Primários 01/01/2014 01:00
## 3 Automático              CETESB Dados Primários 01/01/2014 02:00
## 4 Automático              CETESB Dados Primários 01/01/2014 03:00
## 5 Automático              CETESB Dados Primários 01/01/2014 04:00
## 6 Automático              CETESB Dados Primários 01/01/2014 05:00
## 7 Automático              CETESB Dados Primários 01/01/2014 06:00
##   CodigoEstação                NomeEstação              NomeParâmetro
## 2            95 Cid.Universitária-USP-Ipen NOx (Óxidos de Nitrogênio)
## 3            95 Cid.Universitária-USP-Ipen NOx (Óxidos de Nitrogênio)
## 4            95 Cid.Universitária-USP-Ipen NOx (Óxidos de Nitrogênio)
## 5            95 Cid.Universitária-USP-Ipen NOx (Óxidos de Nitrogênio)
## 6            95 Cid.Universitária-USP-Ipen NOx (Óxidos de Nitrogênio)
## 7            95 Cid.Universitária-USP-Ipen NOx (Óxidos de Nitrogênio)
##   UnidadedeMedida MediaHoraria MediaMovel Valido
## 2             ppb            9          -    Não
## 3             ppb            9          -    Sim
## 4             ppb            5          -    Sim
## 5             ppb            4          -    Sim
## 6             ppb            5          -    Sim
## 7             ppb            5          -    Sim
\end{verbatim}

\begin{Shaded}
\begin{Highlighting}[]
\KeywordTok{tail}\NormalTok{(df)}
\end{Highlighting}
\end{Shaded}

\begin{verbatim}
##      TipodeRede TipodeMonitoramento            Tipo       Data  Hora
## 8577 Automático              CETESB Dados Primários 01/01/2015 19:00
## 8578 Automático              CETESB Dados Primários 01/01/2015 20:00
## 8579 Automático              CETESB Dados Primários 01/01/2015 21:00
## 8580 Automático              CETESB Dados Primários 01/01/2015 22:00
## 8581 Automático              CETESB Dados Primários 01/01/2015 23:00
## 8582 Automático              CETESB Dados Primários 01/01/2015 24:00
##      CodigoEstação                NomeEstação              NomeParâmetro
## 8577            95 Cid.Universitária-USP-Ipen NOx (Óxidos de Nitrogênio)
## 8578            95 Cid.Universitária-USP-Ipen NOx (Óxidos de Nitrogênio)
## 8579            95 Cid.Universitária-USP-Ipen NOx (Óxidos de Nitrogênio)
## 8580            95 Cid.Universitária-USP-Ipen NOx (Óxidos de Nitrogênio)
## 8581            95 Cid.Universitária-USP-Ipen NOx (Óxidos de Nitrogênio)
## 8582            95 Cid.Universitária-USP-Ipen NOx (Óxidos de Nitrogênio)
##      UnidadedeMedida MediaHoraria MediaMovel Valido
## 8577             ppb            3          -    Sim
## 8578             ppb            8          -    Sim
## 8579             ppb           11          -    Sim
## 8580             ppb           11          -    Sim
## 8581             ppb           16          -    Sim
## 8582             ppb           NA          -    Sim
\end{verbatim}

Agora vamos ler os mesmos dados com outro formato e testar e read.table
funciona do mesmo jeito

\begin{Shaded}
\begin{Highlighting}[]
\NormalTok{df2 <-}\StringTok{ }\KeywordTok{read.table}\NormalTok{(}\StringTok{"https://raw.githubusercontent.com/ibarraespinosa/cursoR/master/dados/NOXIPEN2014v2.txt"}\NormalTok{)}
\CommentTok{# Error in scan(file = file, what = what, sep = sep, quote = quote, dec = dec, : }
\CommentTok{# linha 1 não tinha 6 elementos}
\end{Highlighting}
\end{Shaded}

Vemos a mensagem de error, mas o que quer dizer.

\textbf{Se tu recever um banco de dados tipo .txt e quer abrir no
R\ldots{} ABRE ELE COM BLOCO DE NOTAS PRIMEIRO!!!}

O primeiro arquivo:

\includegraphics[width=18.47in]{figuras/f1}

O segundo arquivo é:

\includegraphics[width=15.33in]{figuras/f2}

qual é a diferença?

Como vemos o segundo arquivo tem separação de ``;'', entao, temos que
lero arquivo assim:

\begin{Shaded}
\begin{Highlighting}[]
\NormalTok{df2 <-}\StringTok{ }\KeywordTok{read.table}\NormalTok{(}\StringTok{"https://raw.githubusercontent.com/ibarraespinosa/cursoR/master/dados/NOXIPEN2014v2.txt"}\NormalTok{, }\DataTypeTok{sep =} \StringTok{";"}\NormalTok{)}
\KeywordTok{head}\NormalTok{(df2)}
\end{Highlighting}
\end{Shaded}

\begin{verbatim}
##   TipodeRede TipodeMonitoramento            Tipo       Data  Hora
## 2 Automático              CETESB Dados Primários 01/01/2014 01:00
## 3 Automático              CETESB Dados Primários 01/01/2014 02:00
## 4 Automático              CETESB Dados Primários 01/01/2014 03:00
## 5 Automático              CETESB Dados Primários 01/01/2014 04:00
## 6 Automático              CETESB Dados Primários 01/01/2014 05:00
## 7 Automático              CETESB Dados Primários 01/01/2014 06:00
##   CodigoEstação                NomeEstação              NomeParâmetro
## 2            95 Cid.Universitária-USP-Ipen NOx (Óxidos de Nitrogênio)
## 3            95 Cid.Universitária-USP-Ipen NOx (Óxidos de Nitrogênio)
## 4            95 Cid.Universitária-USP-Ipen NOx (Óxidos de Nitrogênio)
## 5            95 Cid.Universitária-USP-Ipen NOx (Óxidos de Nitrogênio)
## 6            95 Cid.Universitária-USP-Ipen NOx (Óxidos de Nitrogênio)
## 7            95 Cid.Universitária-USP-Ipen NOx (Óxidos de Nitrogênio)
##   UnidadedeMedida MediaHoraria MediaMovel Valido
## 2             ppb            9          -    Não
## 3             ppb            9          -    Sim
## 4             ppb            5          -    Sim
## 5             ppb            4          -    Sim
## 6             ppb            5          -    Sim
## 7             ppb            5          -    Sim
\end{verbatim}

\begin{Shaded}
\begin{Highlighting}[]
\KeywordTok{tail}\NormalTok{(df2)}
\end{Highlighting}
\end{Shaded}

\begin{verbatim}
##      TipodeRede TipodeMonitoramento            Tipo       Data  Hora
## 8577 Automático              CETESB Dados Primários 01/01/2015 19:00
## 8578 Automático              CETESB Dados Primários 01/01/2015 20:00
## 8579 Automático              CETESB Dados Primários 01/01/2015 21:00
## 8580 Automático              CETESB Dados Primários 01/01/2015 22:00
## 8581 Automático              CETESB Dados Primários 01/01/2015 23:00
## 8582 Automático              CETESB Dados Primários 01/01/2015 24:00
##      CodigoEstação                NomeEstação              NomeParâmetro
## 8577            95 Cid.Universitária-USP-Ipen NOx (Óxidos de Nitrogênio)
## 8578            95 Cid.Universitária-USP-Ipen NOx (Óxidos de Nitrogênio)
## 8579            95 Cid.Universitária-USP-Ipen NOx (Óxidos de Nitrogênio)
## 8580            95 Cid.Universitária-USP-Ipen NOx (Óxidos de Nitrogênio)
## 8581            95 Cid.Universitária-USP-Ipen NOx (Óxidos de Nitrogênio)
## 8582            95 Cid.Universitária-USP-Ipen NOx (Óxidos de Nitrogênio)
##      UnidadedeMedida MediaHoraria MediaMovel Valido
## 8577             ppb            3          -    Sim
## 8578             ppb            8          -    Sim
## 8579             ppb           11          -    Sim
## 8580             ppb           11          -    Sim
## 8581             ppb           16          -    Sim
## 8582             ppb           NA          -    Sim
\end{verbatim}

\section{data.table}\label{data.table}

data.table é um pacote que apresenta a classe \texttt{data.table}, que é
como uma versão melhorada da classe \texttt{data-frame} O termo
especifico é que \texttt{data-table} tem herencia (inherits) da classe
\texttt{data.frame}

\chapter{Applications}\label{applications}

Some \emph{significant} applications are demonstrated in this chapter.

\section{Example one}\label{example-one}

\section{Example two}\label{example-two}

\chapter{Final Words}\label{final-words}

We have finished a nice book.

\bibliography{book.bib,packages.bib}


\end{document}
